\section{The Conceptualisations}
\label{sec:conceptualisations}

The conceptualisations to be developed follow the basic assumptions and
positings that are elaborated in more detail in \cite{Graebe2021}. In
particular, the following namespace prefixes are used:
\begin{itemize}[noitemsep]
  \item \texttt{ex:} -- the namespace of a special system to be modelled. 
  \item \texttt{tc:} -- the namespace of the TRIZ concepts (RDF subjects).
  \item \texttt{od:} -- the namespace of WUMM's own concepts (RDF predicates, 
  	general concepts). 
\end{itemize}

The concept will be to gather various definitions from the mentioned sources.
With these sources the important TRIZ triple will be implemented.

From there the necessary connection will be created for the various triples.
Furthermore missing concepts will be implemented to fully show the Functional Analysis and the belonging concepts.

For the organizing of the turtle file and to structure the building process of the turtle file, first of all the Functional Model will be built. 
This is done by following the five steps, which will be explained in \ref{subsec:functional_model}.
With the given model the Functional Analysis components can be added to attach all important concepts.

In the end the implementation will be supported by different example, which are going to use the implemented concepts.
In figure \ref{fig:example_conceptionalism} an implemented function of an example is shown.

\begin{figure}[H]
    \centering
    \begin{code}\tt
        ex:Steer\\
        \> a tc:function ;\\
        \> tc:SubjectActionObject ex:DriverTurnSteeringwheel ;\\
        \> tc:QualityOfFunction tc:UsefulFunction ;\\
        \> skos:prefLabel "Steer"@en ;\\
        \> skos:Definition "Turning the steering wheel by the\\
        \> \> driver to change the direction of the vehicle."@en .
    \end{code}
    \caption{Example For Function \textit{Steer}}
    \label{fig:example_conceptionalism}
\end{figure}
