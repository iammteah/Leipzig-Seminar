\section{Creating And Modeling the Turtle-File}
\label{sec:turtle_file}

As explained in the book "Systematische Innovationsmethoden" the Functional Model for the Functional Analysis is built in five steps.
These steps have been explained in \ref{subsec:functional_model}.
The explanation of the implementation will also be structured in these five steps.

Each of these steps is also a new WUMM concept in the RDF graph. 

\begin{enumerate}[noitemsep]
    \item od:ListOfComponents
    \item od:ListOfInteractions
    \item od:FunctionForComponents
    \item od:DirectionOfFunction
    \item od:DefiningQualityOfFunction
\end{enumerate}

For each step the necessary implementation steps will be explained. 

In the following all implementation examples will only include the English language tags.
In the Turtle File itself there are also tags in Russian and German.


\subsection{Making A List Of Components}

Within the first step for building the Functional Model all components which are part of the system are listed.
Hence the system concept is implemented.

A system has different objectives, mentioned in \ref{subsubsec:objectives_system}. 
These objectives are implemented as TRIZ components. 
The \textit{hasPrimaryObjective}, \textit{hasSecondaryObjective} and \textit{hasAuxiliaryObjective} WUMM concepts link a system to the belonging objective.

The system consists of subsystems and components. 
As a system can itself be a subsystem of another system, also an upper system is implemented. 

The relations between these are created via new WUMM triples.
These are \textit{hasSubsystem}, \textit{hasUppersystem}, \textit{hasComponent} and \textit{hasComponent}.

The objective of this step is to create a \textit{Component Model}. 
In the component model all component and subsystems of the system are listed.
As this is a Functional Model without the functions, it is implemented as a sub-concept of the Functional Model.
The additions to the Functional Model will be added in the next steps.

Creating the Component Model is done with the Component Analysis. 
This is implemented as a sub-concept of the first step of Building the Functional Model.

The link to the Component Model is done via the \textit{domain} and \textit{range} properties.
As the Component Analysis starts with a system and returns a Component Model.

The implementation of the Component Analysis is shown in figure \ref{fig:implementation_component_analysis}.

\begin{figure}[H]
    \centering
    \begin{code}\tt
        tc:ComponentAnalysis\\
        \> a skos:Concept ;\\
        \> od:subConceptOf od:ListOfComponents ;\\
        \> od:SouchkovDefinition "A step in Function Analysis that identifies\\
        \> \> components of a technical system being analyzed and its supersystem."@en ;\\
        \> rdfs:domain tc:System ;\\
        \> rdfs:range tc:ComponentModel ;\\
        \> skos:prefLabel "Component Analysis"@en ;\\
        \> skos:altLabel "Component and Structural Analysis"@en .
    \end{code}
    \caption{Implementation Of Component Analysis}
    \label{fig:implementation_component_analysis}
\end{figure}

\subsection{Determine Interactions}

In the next step components, which in some kind interact with each other are listed. 
These are called Interactions and are implemented as a TRIZ concept. 

The linking is done with a WUMM concept called \textit{hasInteraction}. 
The RDFS property domain and range are both Components and hence an interaction is created.

All the interactions can be shown with an Interaction Matrix. 
This is mentioned in the implementation as a TRIZ concept. 

The Interaction Matrix can be created with an Interaction Analysis. 
The Interaction Analysis is implemented, see figure \ref{fig:implementation_interaction_analysis}, as a TRIZ concept.
And as this leads to the result of the belonging interactions it is described as a sub-concept of the second step.

\begin{figure}[h]
    \centering
    \begin{code}\tt
        tc:InteractionAnalysis\\
        \> a skos:Concept ;\\
        \> rdfs:domain tc:System ;\\
        \> rdfs:range tc:InteractionMatrix ;\\
        \> od:subConceptOf od:ListOfInteractions ;\\
        \> od:SouchkovDefinition "A part of Function Analysis that identifies\\
        \> \> interactions between components included in a Component Model."@en ;\\
        \> skos:prefLabel "Interaction Analysis"@en ;\\
        \> skos:altLabel "Structure Analysis"@en ;\\
        \> skos:altLabel "Funtion Analysis of the Process"@en ;\\
        \> skos:altLabel "Process analysis"@en ;\\
        \> skos:example: "The coffee production process is technically implemented\\
        \> \> by the components of the coffee machine."@en .\\
    \end{code}
    \caption{Implementation Of Interaction Analysis}
    \label{fig:implementation_interaction_analysis}
\end{figure}

\subsection{Linking Functions To Components}

In the next step functions will be linked to the belonging components. 

Therefore a action is defined. 
Furthermore a WUMM concept is added which attaches an action to an interaction.

A function is defined as having a direction, but as this is going to be added in the next step the term function is going to be used without these functionality in this section.

The allowed values for constructing a function will be the TRIZ concept Subject-Action-Object.
As this already includes the direction, this will be explained in the following section (\ref{subsec:direction_function}).

The sub-concepts of a function are the various specifications of a function. 
These are all summed up in the TRIZ concept \textit{QualityOfFunction}, which is going to be explained in section \ref{subsec:quality_function}.

The following three functions are also implemented as a sub-concept, but are not representing a quality of the function. 
These three ordering where introduced by Nikolai Shchedrin and have been explained in section \ref{subsubsec:primary_function}.

\begin{itemize}[noitemsep]
    \item Main Function
    \item Auxiliary Function
    \item Core Function
\end{itemize}

Nikolai Shchedrin introduced the Model Of A Function, which will also be implemented.
This model represents the function with two components, an action, a parameter, the type of the function and the degree of execution. 
The not already implemented \textit{Parameter} and \textit{Degree Of Execution} will be added as TRIZ concept. 
Furthermore WUMM concepts will be added, which attach these TRIZ concepts to a function.

As mentioned in section \ref{subsubsec:function_classification} Nikolai Shchedrin also added three new types for classifying a function.
These types were also added as a TRIZ concept. 
With the WUMM concept \textit{hasFunctionType} a function is connected with the according function type.

\subsection{Determining The Direction Of The Function}
\label{subsec:direction_function}

The representation for the function is done with the concept of Subject-Action-Object. 
This is implemented as a TRIZ concept. 
Therefore the \textit{Function Carrier} and the \textit{Object of the Function} is needed.
Both are also embedded as a TRIZ concept.

To connect an object of a function with a function carrier, the new WUMM concept \textit{hasFunctionCarrier} is introduced.
This is done similar for the opposite direction.

An action can be attached to a direction with \textit{hasDirection}.
This maps a action to a Subject-Action-Object triple.

\begin{figure}[h]
    \centering
    \begin{code}\tt
        tc:SubjectActionObject\\
        \> a skos:Concept ;\\
        \> od:subConceptOf tc:FunctionModeling ;\\
        \> od:hasSubConcept tc:FunctionCarrier, tc:ObjectOfFunction, tc:Predicate,\\
        \> \> tc:DirectionOfFunction ;\\
        \> od:SouchkovDefinition "A triad which identifies a Function Carrier,\\
        \> \> its specific Function, and Target Object."@en ;\\
        \> skos:prefLabel "Subject - Action - Object"@en ;\\
        \> skos:altLabel "Tool - action - product"@en .
    \end{code}
    \caption{Implementation Of Subject-Action-Object}
    \label{fig:implementation_subject_action_object}
\end{figure}

\subsection{Define The Quality Of The Function}
\label{subsec:quality_function}

The fifth and last step in building the Functional Model within the Functional Analysis defines the quality of the given function.

Therefore, as mentioned before, the TRIZ concept \textit{QualityOfFunction} is introduced. 
This is a sub-concept of the function and in that way linked to the function.

The following items define the quality of the function and are thus allowed values in the \textit{QualityOfFunction}.

\begin{itemize}[noitemsep]
    \item Additional Function
    \item Basic Function
    \item Excessive Function
    \item Harmful Function
    \item Ideal Function
    \item Insufficient Function
    \item Neutral Function
    \item Providing Function
    \item Technical Function
    \item Useful Function
    \item Transport Function
    \item Bad Controllable Function
    \item Redundant Function
    \item Missing Function
\end{itemize}

The explanation of those can be found in the book "Systematische Innovationsmethoden" \cite{KS} and in the paper from Nikolai Shchedrin \cite{WebinarFunctionAnalysis}.

The following two are introduced by Nikolaid Shchedrin and are a aggregation of quality of functions from the above list.
These are also implmented as TRIZ concept. 

\begin{itemize}[noitemsep]
    \item Function Disadvantage
    \item Useful Function Disadvantage
\end{itemize}

After running all five steps on the system the Functional Model is created. 
As the Component Model is small part of the Functional Model, this is mentioned as sub-concept. 
Furthermore the Function Modeling describes a process and rules for building the Function Model. 
Hence this is also a sub-concept of Function Model. 

Furthermore for completion \textit{process} is added as a TRIZ concept. 

\subsection{Optimizing The Function}

After creating the Functional Model a Principle has to be applied to the function which need optimization. 
This is done by choosing an increasing and decreasing parameter from the chosen Matrix. 
These parameters have been implemented as \textit{rdf:property}.

In a next step these properties are added to a chosen function and form a new concept, named \textit{functionWithMatrixParameter}.

With the concept \textit{ChooseMatrixEntryByFunction} the parameters are then checked and the suitable matrix entry is returned.
As there are various Matrix, which differ slightly in their entries a new property \textit{chosenMatrix} is added as an allowed value to \textit{ChooseMatrixEntryByFunction}. 
With this the chosen Matrix can be mentioned. 

This results in the \textit{MatrixEntryByFunctionParameter} concept. 
With this and the implemented WUMM concept \textit{ChoosePrincipleFromMatrixEntry} a Principle from the matrix entry is chosen. 
Finally the chosen principle is connected with the function by \textit{PrincipleByMatrixEntryByFunctionParameter}.

\subsection{The Matrix 2003}

The Matrix 2003 has been created with the help of the python script, which has been explained in \ref{subsubsec:creating_matrix}.
To connect the principle id's from the Matrix 2003 with the implemented principles in the file \textit{Principles.ttl} an id has been added.
The property has been called \textbf{od:Matrix2003Id}.

\subsection{The Parameters}

As for the created Matrix 2003 more parameters and upper-parameter were added, the already created \textit{Parameters.ttl} was updated.

The nine new parameters were created as TRIZ concepts.

Furthermore the six upper-parameters were also added as TRIZ concepts. 
For those the new WUMM concept \textit{od:UpperParameter} was implemented.