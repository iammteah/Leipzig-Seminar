\section{Functional Analysis}
\label{sec:functional_analysis}

In this section a summary about the terms and definitions explained in \cite{KS} will be given. 
This is going to help explain in the following sections, why the \textit{turtle}-file was modeled in that way.

\subsection{Concept}

In Functional Analysis the idea is to represent a system by various functions.
This representation helps in organizing and structuring the system.
To tackle the optimization problem a more precise look is taken at the non-useful and contradictory functions in the system. 

The two main objectives of Functional Analysis in TRIZ are formulated in the following.

\begin{enumerate}
    \item Recognizing Of Problems \newline
        Hereby the objective is to find as many as possible non-useful or contradictory functions in the system.
    \item Trimming Of Components \newline
        To optimize the system some components have to be redesigned. 
        During this process the functionality of the component must not be changed.  
\end{enumerate}



With these two main objectives there are five tasks to handle in Functional Analysis.

\begin{enumerate}[noitemsep]
	\item Recognizing Interactions Between Objects
	\item Recognizing Problems Within The System
	\item Formulating Open Problems
	\item Innovative Redesign
	\item Optimizing Systems By Trimming
	\item Bypassing Patents
\end{enumerate}


\subsection{Quality Of A Function}

To further analyze the functions it is recommended to give each function a level of quality.
This quality of a function makes it easier to find problems within the system.

Accorind to the book there are five different quality levels.

\begin{enumerate}
    \item Useful Function: \newline
        A function works as intended and the result is only positive.
    \item Useful, But Insufficient Function \newline
        A function has a positive impact but the result is not satisfieing.
    \item Useful, But Bad Controllable Function \newline
        A function with a positive impact but is not satisfieing as the result cannot be controlled.
        Hence it is wrongly timed.
    \item Useful, But Excessive Function \newline
        A function with a positive impact but a bigger result than necessary.
    \item Harmful Function \newline
        A function which has a negative impact.
\end{enumerate}

More precise definitions can be found in the book \cite{KS} or in the Glossary of Souchkov \cite{SouchkovGlossary}.
As both of these are merged into the \textit{turtle}-file the defintions can now also be found there.

Nikolai Shchedrin mentioned also a new function quality in his web-seminar. 
This is called \textit{Useful Function With Disadvanteges}.
As mentioned on the website \cite{ShchedrinUsefulFunctionDisadvantage} this class includes Redundant Functions, Insufficient Functions, Bad Controllable Functions and Missing Functions.

As there is no source mentining the Definition of a Redundant Function and a Missing Function, these will be interpreted in the following way.

\begin{enumerate}
    \item Redundant Function: \newline
        A function with a positive result, which is implemented a second time in the system.
    \item Missing Function: \newline
        A useful function which is not implemented in the system and therefore missing.
\end{enumerate}

\subsection{Functional Model}
\label{subsec:functional_model}

The Functional Model structures the function and the components in the system.
In the book the following steps are recommended when building a Functional Model. 
It is also mentioned that steps two to four can be made in one step.

\begin{enumerate}[noitemsep]
	\item Making A List Of Components
	\item Determine Interactions
	\item Linking Functions To Components (Subject)
	\item Determining The Direction Of Function (Arrow)
	\item Define The Quality Of The Function
\end{enumerate}

With this a graph-like structure is built. 
Every node in this graph represents a component or subsystem within the system.
An edge represents a function. 
Different styles and forms of an edge represent the quality of the function.

For easier reading and analyzing the Functional Model, it is possible to group components according to self-defined properties.

\subsection{Positive And Negative Aspects}

Using the Functional Analysis helps you to fully understand the system you are working with.
Furthermore you can exchange knowledge between colleagues and clarify misunderstandings.

A Functional Analysis makes also sense, when no problem is known.
This is good if you want to improve your system without knowing specific complications.

A negative attitude of the Functional Analysis, is that only known systems can be analyzed.
