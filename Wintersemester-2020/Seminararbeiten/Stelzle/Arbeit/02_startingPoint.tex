\section{Starting point} 
\label{sec:starting_point}

The starting point for the building the ontolog for Functional Analysis is going to be the book \textit{Systematische Innovationsmethoden} \cite{KS}.
The definitions, demenstrations and explanations will be used to implement the ontology.

Other definitions for Functional Analyis will be picked from \textit{The Glossary Of Tritz and TRIZ-Related Terms} by Valeri Souchkov \cite{SouchkovGlossary}. 
As this is a great summary with good definitions.

These two information sources and the following more detailed explained Webinar from Nikolai Shchedrin are the starting point in building the ontology for Functional Analysis.


\subsection{Webinar of Nikolai Shchedrin}

Nikolai Shchedrin made a talk about building an ontology for \textit{Function} and \textit{Function Analysis} \cite{WebinarFunctionAnalysis}.
He has presented some insights on how to structure the ontolgy.

\subsubsection{Function Classification}
\label{subsubsec:function_classification}

First he introduced, that there should a be a new classification for the functions. 
There should be three types: 

\begin{enumerate}[noitemsep]
	\item Function of the subsystem
	\item Function of the upper system
	\item Function of the surrounding objects
\end{enumerate}

\subsubsection{Model Of A Function}

Furthermore he introduced the \textit{Model of a function}.
With this a function is further described. 
Not only does it show the Action and the two components which interact, but it furthermore shows the parameter, the type of the function and the degree of execution.

The \textit{Functional Model} is a graph representation of the system which is analyzed. 
Every node in the graph is a component or a subsystem of the system. 
The functions are represented by edges. 
This will be further explained in \ref{subsec:functional_model}.

The \textit{Model of a function} can be helpful for building the \textit{Functional Model}.

\subsubsection{Objectives Of A System}
\label{subsubsec:objectives_system}

As explained by Nikolai Shchedrin the objective of a system can be divided into three objectives.

\begin{enumerate}[noitemsep]
	\item Primary Objective
	\item Secondary Objective
	\item Auxiliary Objective
\end{enumerate}

The Primary Objective is the objective, which the system was build for. 

Secondary Objectives of the system are functionalities which are offered by the system, but for which it was not mainly built.

And the Auxiliary Objectives fulfill the purpose to get the Primary Objective to work.

\subsubsection{Primary Function}
\label{subsubsec:primary_function}

Furthermore a function can be classified in one of these three functions.

\begin{enumerate}[noitemsep]
	\item Primary Function
	\item Core Function
	\item Auxiliary Function
\end{enumerate}

The Primary Function is the Primary Objective and its technical execution. 

The Core Function represents a function, which directly helps executing the Primary Function.

The Auxiliary Functions describe the set of functions which help run other subsystems.
