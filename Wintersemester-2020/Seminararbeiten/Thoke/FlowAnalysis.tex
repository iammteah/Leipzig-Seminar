% Einfache LaTeX-Vorlage für Texte
\documentclass[a4paper,11pt]{article}

\usepackage{ls}
\usepackage[main=english,russian]{babel}

\newenvironment{code}{\tt \begin{tabbing}
\hskip12pt\=\hskip12pt\=\hskip12pt\=\hskip12pt\=\hskip5cm\=\hskip5cm\=\kill}
{\end{tabbing}}
\def\dq{{\char34}}

\parindent0pt
\parskip4pt

\author{Immanuel Thoke}

\title{Modelling TRIZ Flow Analysis with RDF}

\begin{document}
    \maketitle
    \tableofcontents
    \newpage
    \section{Abbreviations}
    FAM - Flow Analysis Model

    ICE - internal combustion engine

    LETS - Laws of Evolution of Technical Systems

    MINT - german acronym for Mathematik, Informatik, Naturwissenschaft und Technik

    TOP - TRIZ Ontology Project

    TRIZ - russian acronym for \foreignlanguage{russian}{теория решения изобретательских задач}, see TIPS

    TIPS - Theory of Inventive Problem Solving 

    RDF - Resource Description Framework 

    WUMM - Widersprüche Und Management-Methoden

    ZRTS - russian acronym for \foreignlanguage{russian}{Комплекс законов и тенденций развития технических систем}, see LETS

    \newpage

    \section{Aim of the work}

    The aim of this paper is to elaborate a proposal for an ontological modelling
    of the areas of \emph{TRIZ Flows} and \emph{TRIZ Flow Analysis} based on the
    materials of the TRIZ Ontology Project (TOP) of the TRIZ Developer Summit
    \cite{TOP} and further own investigations. The work fits into the activities
    of the \emph{WUMM Project} \cite{WUMM} to accompany the TOP project and to
    transfer it into the language of modern semantic concepts
    \cite{WUMM-Ontology}.  The work therefore consists of two parts -- a
    \emph{turtle file}, in which the semantic modelling is performed based on the
    SKOS framework \cite{SKOS}, and \emph{this elaboration}, in which the
    backgrounds and motivations of the concrete modelling decisions are detailed.

    \section{Starting point} 

    Flows and flow analysis play a rather peripheral role in TRIZ.  In the
    standard reference \cite{Petrov2007} on a \emph{TRIZ Body of Knowledge} is
    listed as item 2.5.8, but in the 7 references to the literature listed there
    are no systematic explanations about the role of flows in TRIZ theory, but
    only individual examples in which concrete flows (such as magnetic flow)
    played a role in concrete modelling.

    The most comprehensive source on issues of TRIZ modelling of flows and
    methodological issues of flow analysis is the thesis \cite{Lebedyev2015} by
    Yuri Lebedyev, which he submitted in 2015 under the supervision of S.A. 
    Logvinov for graduation as TRIZ Master.

    Such a limitation is justified, since the aim of this seminar paper is not to
    completely cover the TRIZ theory of flow analysis, but only to identify its
    essential concepts and describe their connection with modern semantic RDF
    means based on the SKOS ontology. The basis for this is above all the
    presentation of Olga Eckardt in the webinar of the TOP project in October 2020
    \cite{Eckardt2020}, in which basic approaches of the Lebedyev's work have
    obviously been incorporated.

    We start from the definition of both concepts in the TOP Glossary
    \cite{TOP-Glossary}: 
    \begin{quote}
        \textbf{Flow} is the directional movement in space of particles of mass of
        matter, as well as the directional movement of energy or information. Flow
        has dual properties: the properties of a substance of which the flow
        consists, and the properties of a field, which is formed as a result of the
        directional movement of particles of matter. Flows can be useful, harmful
        and parasitic. The flow model contains static components: source, channel,
        receiver, control system. Flow is a special case of a process in which
        directional movement occurs in physical space.

        \textbf{Flow analysis} is a method of systems analysis to establish
        relationships in a system to flow, find resources, and determine compliance
        with existing system requirements. The result of a flow analysis is a model
        of the flow as is and a model of the flow to be. Flow modification
        techniques are used to develop systems with flows and to solve inventive
        problems in them.
    \end{quote}
    \newpage
    The use of these concepts is further included there in a single place in the
    context of cause-effect analysis (ibid.):
    \begin{quote}
        Cause-effect analysis is performed: 
    \begin{itemize}[noitemsep]
        \item When the causes of an undesirable effect are not clear (when we cannot
        go from an administrative contradiction to a technical one $AC\to TC$.
        \item When it is necessary to clarify the causes of an undesirable effect
        (to deepen the understanding of the causes of an undesirable effect, e.g.
        after a functional or a flow analysis).
    \end{itemize}
    \end{quote}
    Other sources are much more sparing in their statements on these two
    individual terms, but list other flow-related concepts, for example in
    \cite{Souchkov2018}:
    \begin{itemize}[noitemsep]
        \item Delay Zone -- A location in a flow in which the integral flow speed is
        significantly lower than local flow speed. A Delay Zone is a typical
        disadvantage identified by Flow Analysis. 
        \item Flow -- A sequence of events that have the same common feature.
        \item  Flow Analysis-- An analytical method and a tool which identifies
        disadvantages in flows of energy, substances, and information in a technical
        system.
        \item  Flow Disadvantage -- A disadvantage of a technical system being
        analyzed identified during Flow Analysis. Examples: Bottlenecks, «Gray
        Zones», «Stagnation Zones», etc.
        \item Flow Distribution Analysis-- A part of Flow Analysis that identifies
        distribution of flows and their disadvantages. 
        \item Stagnation Zone -- A part of a flow in which the flow stops temporarily
        or permanently. A Stagnation Zone is a typical disadvantage identified by
        Flow Analysis.
        \item Transmission -- One of the key components (subsystems) of a Complete
        Technical System which according to the Law of System Completeness of a
        technical system transmits a flow of energy required to operate a working
        unit from an engine to the working unit.
    \end{itemize}

    \section{The Conceptualisations}
    Being more than just a static methodology, there have been several approaches
    to standardize the syntaxes and semantics of the conceptual 
    and methodological knowledge of TRIZ. With launching the TRIZ Ontology Project
    (TOP) several main contributors of the MA TRIZ school are pushing forward this 
    longterm goal from a very core institution of the TRIZ history.
    In awareness of the importance to establish uniform language models within 
    human-machine-interface environments the WUMM\cite{WUMM} TOP Companion Project\cite{WUMMTOP}
    is pushing even further 'to transform that information[...]',
    which has been provided by the TOP,'[...]into RDF as 
    the machine readable standard of the Web 3.0 (also called 'Semantic Web').'\cite{WUMMTOP}.     
    
    'RDF is a standard model for data interchange on the Web. [...] [It] 
    extends the linking structure of the Web to use URIs to name the relationship 
    between things as well as the two ends of the link 
    (this is usually referred to as a “triple”). Using this simple model, it allows 
    structured and semi-structured data to be mixed, exposed, and shared across 
    different applications. \cite{RDF}

    Given that framework, we can map the concepts of TRIZ onto a structure which can
    be read and processed by computers that accept hyper text protocols and link them
    to substantial resources creating an environment to collaborate 
    in innovative projects across the world using semantic technologies.
    
    To collect and manage the development of the RDF source code of WUMMTOP, a git 
    repository \cite{RDFData} is used and has been forked for developing the 
    TRIZ Analysis Ontocard feature as the main goal of this work. The README manual
    of the repository provides necessary preliminaries for using and continue developing 
    the project.
 
    The main part of the modelling workflow is the appropriate design how to translate 
    the ontological structure of the given topic into the RDF scope of three-word-sentences. 
    Therefore the use of several namespaces to map each ontological aspect onto the 
    architecture of subjects, predicates and objetcs is key to deliver a well structured 
    RDF Model. 

    The conceptualisations to be developed follow the basic assumptions and
    positings, that are elaborated in more detail in \cite{Graebe2021}. In
    particular, the following namespace prefixes are used:
    \begin{itemize}[noitemsep]
    \item \texttt{ex:} -- the namespace of a special system to be modelled. 
    \item \texttt{tc:} -- the namespace of the TRIZ concepts (RDF subjects).
    \item \texttt{od:} -- the namespace of WUMM's own concepts (RDF predicates,
    general concepts).
    \item \texttt{skos:} -- the Simple Knowledge Organization System Namespace (mainly for labelling).
    \item \texttt{foaf} -- the namespace of Friend of a Friend to describe real world objects.
    \item \texttt{rdfs:} -- the namespace of RDF Schema (RDF properties and their domains).
    \item \texttt{cc:} -- the namespace of the Creative Commons License (to address copy left approach).
    \item \texttt{owl:} -- the namespace of the Web Ontology Language (to address web ontology conventions).
    \end{itemize}

    As in Souchkov's Glossary we use URIs as 
    \begin{center}\tt
    tc:HarmfulAction, tc:HarmfulFunction, tc:HarmfulInteraction,
    tc:HarmfulMachine
    \end{center}
    that combine the property and the subject of the property except for
    \texttt{tc:FlowSubstance}.

    As you can see, the WUMM project introduced the \texttt{od:} namespace to
    extend the given ontological base of the TOP by its own concepts and 
    provide needed explicit definitions of certain abstract models, which
    are yet unclarified.

    To see which predicates are used in the \texttt{od:} namespace, you can
    use the provided SPARQL Query Editor endpoint \cite{SPARQL} and query

    \begin{code}
        SELECT DISTINCT ?p WHERE {?s ?p ?o FILTER regex(?p, 'od')}.
    \end{code}
    
    \section{Modelling the Flow Analysis Ontocard}
    Flow Analysis, as a rather niche method in TRIZ is grounded by two
    major contributions: While Lyubomirsky \cite{Lyubomirsky2006} started
    with the introduction of a broad set of methods to fix certain
    problems assigned to flow as a functional magnitude, Lebedev 
    \cite{Lebedev2011} \cite{Lebedyev2015} catched up on it and developed a 
    systematic approach of analysis of flows based on the well known principles
    of TRIZ. Finally Eckardt \cite{Eckardt2020} started to adopt this work
    integrating it as a part in the TOP.

    As mentioned before and to rely on proper methods of the design of 
    software components a few small changes to the taxonomy of Eckardt's 
    ontology needed to be induced with several layers of abstraction 
    (which are existing prototypically but are not described explicitely) and 
    specification of the used predicates. The description of the modelling 
    will mainly focus on these parts, while explaining the structure of the 
    ontology itself remains exemplarily, as it has always been specified in 
    the quoted material.

    «An ontology is about «modelling of models», because the clarification
    of terms and concepts aimed at with an ontology is intended to be
    practically used in real-world modelling contexts. This «modelling of 
    models» references a typical engineering context, in which the 
    \emph{modelling} of systems plays a central role and serves as basis of
    further planned action (including project planning, implementation,
    operation maintenance, further development of the system).
    In this process, \emph{several levels of abstraction} are to be 
    distinguished.»\cite{Graebe2021}
    
    \begin{itemize}
        \item[0.] The level of the \emph{real-world system} 
        \item[1.] The level of \emph{modelling the real world system} 
        \item[2.] The \emph{level of the meta-model} as the actual (TRIZ) ontology
        on which the systemic concepts are \emph{defined} 
        \item[3.] The \emph{modelling meta-level 2} at which the methodological
        concepts are defined \cite{Graebe2021} 
        \footnote{see source for more details.} 
    \end{quote}


    The methodological practice described in this work is mainly based at 
    level 2, meaning the modelled objects and subjects in terms of RDF refer to 
    the concept of classes used in object-oriented programming. 
    Therefore all properties of RDF objects described here are generic class
    properties, which have to be instanciated separately as a RDF triplet of 
    rdfs:type|s referring a certain object of class properties

    The Flow Analysis Model can be separated in two parts. The 
    \emph{Flow Model} refers to the descriptive and functional properties
    of a flow, its (static) components as abstract or concrete real world 
    objects, that define its properties and description methods, like
    textual or graphical representations. The \emph{Flow Analysis (Model)}
    describes the given context of a flow analysis, e.g. its requirements,
    rules that derive from general TRIZ principles or the LETS, which form
    an overall solution process containing methods how to solve problems using 
    specific flow innovation methods. 

    \subsection{\texttt{tc:Flow}}

    A flow is, on the one hand, a component within a system but on the other 
    hand can describe a system as a dynamic entity in the flow of time. The 
    descriptive and functional properties are modeled along the rules for a 
    morphological table, i.e. defining a PropertyDomain and a list of allowed 
    values, modelled with \texttt{od:allowedValues}, respective \texttt{valueOf} 
    as listed below. The model of Eckardt is slightly extended to cover the 
    full model used by Lebedev. That is \texttt{tc:FlowDelegation} and the
    \emph{Carrier Flow Model}, which is basically obligatory to model the flow 
    of flows and modeled here as a special type of \texttt{tc:FlowFunctionality}.

    \begin{center}
    \begin{tabular}{|c|c|}\hline
        Property & PropertyDomain \\\hline
        od:belongsTo & tc:StageValue \\
        od:hasFlowType & tc:FlowType  \\
        od:hasFlowSubstance & tc:FlowSubstance  \\
        od:hasFlowFunctionality & tc:FlowFunctionality  \\
        od:hasFlowDefect & tc:FlowDefect  \\
        od:hasFlowSource & tc:FlowSource  \\
        od:hasMethodOfDescription & tc:MethodOfDescription \\\hline 
    \end{tabular}
    \end{center}

    \begin{center}
    \begin{tabular}{|l|p{10cm}|}\hline
        PropertyDomain & PropertyValues \\\hline
        tc:FlowType & tc:ComplexFlow, tc:DiscreteFlow, tc:ContinuousFlow \\
        tc:FlowDelegation & tc:OpenFlow, tc:ClosedFlow \\
        tc:FlowSubstance & tc:Information, tc:Energy, tc:Substance \\
        tc:FlowFunctionality & tc:UsefulFlow, tc:HarmfulFlow, tc:ParasiticFlow,
        tc:CarrierFlow  \\
        tc:FlowDefect & tc:InsufficientFlow, tc:ExcessiveFlow,
        tc:BadControllableFlow, tc:AbsentFlow \\ 
        tc:FlowSource & tc:ExternalSource, tc:InternalSource  \\
        tc:MethodOfDescription & tc:TextDescription, tc:ParametricDescription,
        tc:GraphicDescription \\\hline 
    \end{tabular}
    \end{center}

    Here you can see the triples associated with the flow subject:
    \begin{code}\tt
    tc:Flow \\
    \> a skos:Concept, od:SouchkovGlossaryEntry ; \\
    \> od:WOPCategory od:Component; \\
    \> od:SouchkovCategory tc:FlowAnalysis ; \\
    \> od:SouchkovDefinition "A sequence of events that have the same common \\
    \> feature."@en ; \\
    \> od:LebedevDefinition "motion of material, energetic or informational \\
    \> objects within a system" ; \\
    \> skos:prefLabel "Flow"@en, "Fluss"@de, "Поток"@ru ; \\
    \> skos:altLabel "Stream"@en, "Strom"@de . \\
    \end{code}

    \texttt{tc:StageValue} refers to the general concept of \texttt{tc:ModeValue}|s 
    we'll see in action in the example, which is introduced later on. It defines
    different models of flows, which are used along the flow analysis 
    transformation process. \texttt{od:Component} with its predicate
    \texttt{od:WOPCategory} defines whether a component of the analyzed system
    is covered or with \texttt{od:PropertyDomain} properties of the flow as a
    systematic approach is described. Each subject, that refers to a TRIZ 
    concept has a \texttt{skos:Concept}, as well as a 
    \texttt{od:AdditionalConcept}. Beside that we clarify if a subject is defined
    by a certain author, like \texttt{od:LebedevDefinition} or 
    \texttt{od:SouchkovDefinition}. The \texttt{od:SouchkovCategory} refers to a
    specific categorial object defined by the
    \texttt{<http:\\opendiscovery.org/rdf/Ontologies/Souchkov-Glossary/>}.

    \subsection{Subconcepts of \texttt{tc:Flow}}

    A flow consists of several physical parts (source, channel, receiver, 
    control unit) that are components by their own and hence modeled with the 
    predicate \texttt{od:subConceptOf}. For the moment they are attached to the
    flow with a single predicate \texttt{od:hasStaticFlowComponent}. Yet, they 
    have several subcomponents, which are listed below.

    \begin{center}
        \begin{tabular}{|c|c|}\hline
            Property & PropertyDomain \\\hline
            od:hasStaticFlowComponent & tc:StaticFlowComponent \\\hline 
        \end{tabular}
        \end{center}
    
        \begin{center}
        \begin{tabular}{|l|p{10cm}|}\hline
            Components & SubComponents \\\hline
            tc:ControlUnit & tc:Pump, tc:Valve \\
            tc:Receiver & (has no subcomponents yet) \\
            tc:Source & tc:Current, tc:Potential \\
            tc:Channel & (has no subcomponents yet)\\\hline 
        \end{tabular}
        \end{center}
    

    \subsection{\texttt{tc:FlowAnalysis}}
    Eckardt's model of flow analysis is separated into two levels of abstraction.
    The \emph{Top Level View} contains a self descriptive methodological approach,
    which parts of the flow analysis are relevant to carry out a flow transformation
    process. The \emph{Technical View} describes how a transformation is prescribed
    by its compliance rules and its LETS and how requirements are met by 
    solving the contradictions with several \texttt{tc:TechniqesofModification}. 

    The ontological representation of such a process is extremely difficult and is
    not considered here. The ontology of the flow analysis is restricted to the
    description of the \emph{formative elements of the analysis} and the creation
    of a manually generated \emph{FlowTransformation} document.

    \subsubsection{Top Level View}
    A Flow Analysis is an methodological process. Its \emph{Top Level View}
    conncects the given context and a set of rules, that define procedural policies
    to create a project plan, the \texttt{tc:FlowTransformation} (a 
    \texttt{foaf:Document}), in a yet hypothetic Flow Transformation Description
    Language) with the defined \texttt{tc:Flow} model in several stages to define 
    steps of evolution of the technical system and how to process them to accomplish
    the desired transformative action.

    \begin{center}
        \begin{tabular}{|c|c|}\hline
            Property & PropertyRange \\\hline
            od:describesFlow & tc:Flow \\
            od:analysesFlow & tc:Flow \\
            od:inContext & tc:Context \\
            od:followsRules & tc:RulesOfFlowTransformation \\
            od:usesDescriptionMethods & tc:Flow, ex:FlowModelAsIs, \\
            & ex:FlowModelAsRequired, tc:FlowTransformation \\
            od:generatesFlowTransformation & tc:FlowTransformation \\
            od:applyChanges & tc:Flow \\\hline 
        \end{tabular}
        \end{center}
    
    While these properties usually operate from the Domain of 
    \texttt{tc:FlowAnalysis}, with \newline
    \texttt{od:applyChanges} all approaches of the Flow Analysis are finally
    gathered in the \newline \texttt{tc:FlowTransformation} and can be applied 
    to the real system (level 0).

    \texttt{tc:Context} encapsulates all information about the exact conditions
    of the analysis and \newline thus are modelled as \texttt{od:subConceptOf}.

    \begin{center}
        \begin{tabular}{|c|c|}\hline
            Concept & hasSubConcept \\\hline
            tc:Context & tc:Goal, tc:Requirement \\
            tc:Requirement & tc:IncreseUsefulFlow, tc:ReduceHarmfulFlow, \\
            & tc:ReduceParasiticFlow \\\hline 
        \end{tabular}
        \end{center}

    \subsubsection{Technical View}
    The Technical View combines two central concepts how and why methods are applied
    in TRIZ. It connects the \texttt{tc:RulesOfFlowTransformation} with the 
    \emph{ModelAsIs} and the \emph{ModelAsRequired}. It contains their rules of 
    construction, integration of evaluation, their requirements and rules when to use
    which \texttt{tc:TechniquesofModification} to establish a target transformation. 

    With \texttt{od:buildModelAsIs} and \texttt{od:buildModelAsRequired} we define the
    predicates to describe the process how specific instances of a \emph{Flow Model}
    are built to define starting and desired target point of the flow transformation.
    Here, the \texttt{tc:RulesOfFlowTransformation} are attached, that build a
    framework of design directives, not only how the models are developed but also
    how these models are connected to establish a valid solution out of the issued
    circumstances of flow analysis. While \texttt{tc:RulesOfConstruction} and
    \texttt{tc:RulesofEvalution} refer to a certain \texttt{ex:FlowModelAsIs}, 
    the \texttt{tc:RulesOfConstructionOfInconformRequirements} and the
    \texttt{tc:RulesOfApplyingLawsOfDevelopmentOfTechnicalSystems} describe
    how the evaluated contradictions in the \texttt{ex:FlowModelAsIs} can
    be resolved via the knwon LETS to develop a \texttt{ex:FlowModelAsRequired}.
    
    The \texttt{tc:TechniquesofModification} is the last core piece of this 
    ontologization of Flow Analysis. It's a set of methods that are derived 
    from certain LETS or so called trends occuring in technical systems 
    that prescribe certain proceedings. \cite{KoltzeSouchkov2017} \emph{They} 
    are \texttt{od:issuedAt} the \texttt{tc:ModelAsIs} to solve a certain
    \texttt{tc:FlowDefect}. Each of these methods are modeled as
    \texttt{od:subConceptOf tc:TechniquesofModification} and are categorized
    by the static components they're intended to change. Additionally,
    they're categorized by the overall requirement set in context. For example:

    \begin{code}\tt
        tc:TechniquesOfModification \\
        \> od:issuedAt tc:ModelAsIs ; \\
        \> od:forms tc:ModelAsRequired ; \\
        \> od:hasSubConcept tc:ModificationOfReceiver, tc:ModificationOfSource, \\
        \> tc:ModificationOfChannel, tc:ModificationOfControlUnit ; \\
        \> skos:prefLabel "techniques of modification"@en, "Techniken zur  \\
        \> Modifikation"@de ; \\
        \> skos:altLabel "techniques of transformation"@en, "Techniken zur \\
        \> Transformation"@de ; \\
        \> a skos:Concept, od:AdditionalConcept . \\
        \\
        \\
        tc:ModificationOfChannel \\
        \> od:subConceptOf tc:TechniquesOfModification ; \\
        \> od:hasSubConcept tc:removeBottleNeck, tc:removeDeadZone,  \\
        \> tc:removeGreyZone ,tc:changeFlowMedium, tc:outsourceFlow,  \\
        \> tc:reduceAmountOfTransformations, tc:improvePermeability,  \\
        \> tc:reduceLength, tc:ModificationOfControlUnit, tc:combineUsefulFlows,  \\
        \> tc:channelFlow, tc:implementBottleNeck, tc:implementDeadZone,  \\
        \> tc:reducePermeability, tc:extendLength ; \\
        \> skos:prefLabel "modification of the channel"@en, "Modifikation des  \\
        \> Kanals"@de ; \\
        \> skos:altLabel "transformation of the channel"@en, "Transformation  \\
        \> des Kanals"@de ; \\
        \> a skos:Concept, od:AdditionalConcept . \\
        \\
        tc:removeBottleNeck \\
        \> od:subConceptOf tc:ModificationOfChannel, tc:IncreaseUsefulFlow ; \\
        \> skos:prefLabel "remove bottle neck"@en, "Flaschenhals entfernen"@de ; \\
        \> a skos:Concept, od:AdditionalConcept . \\
        \end{code}

    \section{Example for Flow Analysis}
    \subsection{Model of an Internal Combustion Engine}

    To illustrate how this machine readable ontology can be used to model real
    world problems, we introduce an example on how to improve the exhaust
    flow of an internal combustion engine (ICE), which have been proposed by 
    Lebedev \cite{Lebedyev2015}. We show how the flow model is implemented,
    address some weaknesses that occured using the given Ontocard and 
    translating it into RDF and show how the Flow Analysis is applicated.

    We used \emph{Wikipedia}[source!] to adjust the given example and meet the
    typical terminology. You will see, that the granularity stays on the level
    of the generality of the used terms. This is an important property of 
    ontological modelling. You can go up and down that ladder as the used
    analytical methods and vocabulary allow you to. In the example this can be
    seen in the specific categorizations and assignments to each level of 
    abstraction (see 6.), which change when you enter the terminology of
    an ICE or are abstracted, when only the general categories of the 
    flow analysis concepts are met. This depends a lot on the given context,
    which is why there is yet no general solution when to use which level of 
    abstraction is sufficient. Intuitively, the more the problem is located
    in specific components, that refer to a special context in that domain,
    the more concrete the terminology needs to get.
    \newpage
    First we show you the basic model of an ICE:
    \begin{code}\tt
        ex:BasicInternalCombustionEngine \\
        \> a tc:Flow ; \\
        \> od:hasFlowType tc:ComplexFlow ; \\
        \> od:hasFlowSubstance ex:Fuel, ex:Transmission, ex:Heat, ex:Work, \\
        \> ex:Cooling, ex:Exhaust ; \\
        \> od:hasFlowFunctionality tc:CarrierFlow ; \\
        \> od:hasFlowDelegation tc:OpenFlow ; \\
        \> od:hasFlowSource tc:InternalSource ; \\
        \> od:hasStaticFlowComponent ex:Crankshaft, ex:ExhaustCamshaft, \\
        \> ex:InletCamshaft, ex:Piston, ex:ConnectionRod, ex:SparkPlug, \\
        \> ex:CombustionChamber, ex:IntakeValve, ex:ExhaustValve, \\
        \> ex:CoolingWaterJacket ; \\
        \> od:hasMethodOfDescription tc:ParametricDescription, \\
        \> ex:ICE3DModel ; \\
        \> skos:prefLabel "basic internal combustion engine"@en, \\
        \> "grundlegendes Modell eines Verbrennungsmotors"@de  ; \\
        \> skos:note "basic construction model from wikipedia \\
        \> <https://en.wikipedia.org/wiki/Internal\_combustion\_engine>" . \\
    \end{code}

    Here you can see some decisions on abstraction and ascertainment. We've
    focused on concrete descriptions of \texttt{tc:FlowSubstance} and
    \texttt{tc:StaticFlowComponent}s as they are sufficient to show the 
    problems of an exhaust flow and how to model an easy improvement.

    Thus, the ICE is already a complex machine, where you have at least
    two interdependent system levels, i. e. a flow of flows. But the given 
    Ontocard doesn't define rules how systems of systems, respectively flows 
    of flows, are modelled, because the Flow Model lacks a model of self 
    description. Thats why this modelling is still experimentally. Therefore we
    made some assumptions:
    - What's an \texttt{tc:FlowSubstance} on the second level is a 
    \texttt{tc:Flow} on the first level
    - The \texttt{tc:FlowFunctionality} of a second order flow is always a 
    \texttt{tc:CarrierFlow} \footnote{see chapter 7 in Lebedyev2015 
    \cite{Lebedyev2015} }

    So let's have a look at the \texttt{ex:Exhaust} flow:
    \begin{code}\tt
        ex:Exhaust \\
        \> a tc:Flow ; \\
        \> od:hasFlowType tc:DiscreteFlow ; \\
        \> od:hasFlowSubstance tc:Substance ; \\
        \> od:hasFlowFunctionality tc:HarmfulFlow ; \\
        \> od:hasFlowDelegation tc:OpenFlow ; \\
        \> od:hasFlowSource tc:InternalSource ; \\
        \> od:hasStaticFlowComponent ex:CombustionChamber, ex:ExhaustPipe, \\
        \> ex:Atmosphere, ex:ExhaustValve ; \\
        \> skos:prefLabel "the exhaust flow"@en, "Abgasstrom"@de . \\
    \end{code}

    \texttt{ex:Exhaust} is a \texttt{tc:HarmfulFlow} with several static 
    components. These components are not exclusive for the exhaust flow.
    They're used in other flow systems as different flow components. For
    example the \texttt{ex:CombustionChamber}, here modelled as
    \texttt{a tc:Potential}, is \texttt{a tc:Channel} in the system of the
    \texttt{tc:Heat} flow. To inherit context sensitivity to these 
    subjects, we use \texttt{od:specialCase}: \cite{RDFC}
    
    \begin{code}
    ex:CombustionChamber \\
    \> od:specialCase ex:Heat ; \\
    \> a tc:Channel ; \\
    \> skos:prefLabel "combustion chamber"@en, "Verbrennungskammer"@de . \\
    \\
    ex:CombustionChamber \\
    \> od:specialCase ex:Exhaust ; \\
    \> a tc:Potential ; \\
    \> skos:prefLabel "combustion chamber"@en, "Verbrennungskammer"@de . \\
    \end{code}

    \subsection{Improvement of the ExhaustFlow}
    We now have a model of a \emph{system as is}, which can be used for
    improvement. As already mentioned, we want to describe the process 
    of exhaust flow innovation. Given the context of 
    \texttt{tc:PollutionRestrictions}, 
    we define the goal \texttt{ex:SatisfactionOfPollutionRestrictions} with the
    requirements to reduce the exhaust flow while the work flow needs to be 
    preserved as good as possible (\texttt{ex:InnovationOfExhaustFlow} 
    \texttt{od:hasRequirement} \texttt{ex:ReduceExhaustFlow}, 
    \texttt{ex:PreserveWorkFlow}). We define a specific
    \texttt{foaf:Document} to hold all information about the innovation
    project, describe which flows need to be described and analyzed,
    following a \emph{ModelAsIs}, as well as a \emph{ModelAsRequired},
    (called \texttt{ex:ICEExhaustionModel} and \newline
    \texttt{ex:buildICEExhaustionInnovationModel}).

    \begin{code}
    ex:InnovationOfExhaustFlow \\
    \> a tc:FlowAnalysis ; \\
    \> od:generatesFlowTransformation ex:ExhaustionFlowTransformation ; \\
    \> od:describesFlow ex:Exhaust, ex:ICEExhaustionModel,  \\
    \> ex:ICEExhaustionInnovationModel ; \\
    \> od:analysesFlow ex:Exhaust, ex:Fuel, ex:Air, ex:AirFuelMixture, ex:Heat, \\
    \> ex:Work ; \\
    \> od:inContext ex:PollutionRestrictions ; \\
    \> od:hasGoal ex:SatisfactionOfPollutionRestrictions ; \\
    \> od:hasRequirement ex:ReduceExhaustFlow, ex:PreserveWorkFlow ; \\
    \> ex:buildICEExhaustionModel ex:ICEExhaustionModel ; \\
    \> ex:buildICEExhaustionInnovationModel ex:ICEExhaustionInnovationModel ; \\
    \> tc:applyRulesofApplyingLawsOfDevelopmentOfTechnicalSystems  \\
    \> ex:ICEExhaustionInnovationModel ; \\
    \> skos:prefLabel "innovation of the exhaust flow"@en, "Innovation des  \\
    \> Abgassstroms"@de . \\
    \end{code}

    Using the \texttt{tc:RulesofEvalution} we can define an 
    \texttt{ex:EvaluationOfExhaustFlow}, that detetcs a 
    \texttt{tc:FlowDefect} which needs to be fixed, which can be found in the 
    \newline
    \texttt{ex:buildICEExhaustionModel}:

    \begin{code}    
    ex:EvaluationOfExhaustFlow \\
    \> a tc:RulesOfEvaluation ; \\
    \> od:detects ex:HighDispersedExhaust ; \\
    \> skos:prefLabel "evaluation of the exhaust flow"@en, "Evaluation des  \\
    \> Abgassstroms"@de . \\
    \\
    ex:HighDispersedExhaust \\
    \> a tc:ExcessiveFlow ; \\
    \> skos:prefLabel "high dispersed exhaust"@en, "Abgasstrom mit (zu) vielen  \\
    \> Schwebstoffen"@de . \\
    \\
    ex:ICEExhaustionModel \\
    \> a tc:Flow, tc:ModelAsIs ; \\
    \> od:hasFlowType tc:ComplexFlow ; \\
    \> od:hasFlowSubstance ex:Fuel, ex:Transmission, ex:Heat, ex:Work,  \\
    \> ex:Cooling, ex:Exhaust ; \\
    \> od:hasFlowFunctionality tc:CarrierFlow ; \\
    \> od:hasFlowDefect ex:HighDispersedExhaust ; \\
    \> od:hasFlowDelegation tc:OpenFlow ; \\
    \> od:hasFlowSource tc:InternalSource ; \\
    \> od:hasStaticFlowComponent ex:Crankshaft, ex:ExhaustCamshaft,  \\
    \> ex:InletCamshaft, ex:Piston, ex:ConnectionRod, ex:SparkPlug,  \\
    \> ex:CombustionChamber, ex:IntakeValve, ex:ExhaustValve,  \\
    \> ex:CoolingWaterJacket ; \\
    \> od:hasMethodOfDescription tc:ParametricDescription, ex:ICE3DModel ; \\
    \> od:generatesFlowTransformation ex:ExhaustionFlowTransformation ; \\
    \> skos:prefLabel "ICE exhaustion model"@en, "Abgasmodell des Verbrennungsmotors"@de . \\
    \end{code}

    Using these informations, certain \texttt{tc:TechniquesOfModification} can 
    be applied to fix the \texttt{tc:FlowDefect} and fullfill the set 
    \texttt{tc:Requirements},
    
    \begin{code}
    ex:installCatalyticConverter \\
    \> a tc:reducePermeability ; \\
    \> skos:prefLabel "install catalytic converter"@en, "Katalysator  \\
    \> einbauen"@de .  \\
    \\
    ex:addGasolineAdditive \\
    \> a tc:compensateWithSecondFlow ; \\
    \> skos:prefLabel "add gasoline additive flow"@en, "Fluss für  \\
    \> Kraftstoffzusatz hinzufügen" . \\
    \end{code}
    
    leading to an additional flow \texttt{ex:FuelAdditive} and upgraded channel
    component of the \texttt{ex:Exhaust} flow,
    \begin{code}
    ex:FuelAdditive \\
    \> a tc:Flow ; \\
    \> od:hasFlowType tc:DiscreteFlow ; \\
    \> od:hasFlowSubstance tc:Substance ; \\
    \> od:hasFlowFunctionality tc:UsefulFlow ; \\
    \> od:hasFlowDelegation tc:ClosedFlow ; \\
    \> od:hasFlowSource tc:InternalSource ; \\
    \> od:hasStaticFlowComponent ex:IntakeValve, ex:FuelPump, ex:FuelTank,  \\
    \> ex:FuelSuppylPipe, ex:CombustionChamber ; \\
    \> skos:prefLabel "flow of fuel additive"@en, "Fluss des  \\
    \> Kraftstoffzusatzes"@de . \\
    \\
    ex:ExhaustPipeWithCatalyticConverter \\
    \> a tc:Channel ; \\
    \> skos:prefLabel "exhaust pipe with catalytic converter"@en, "Auspuff mit  \\
    \> Katalysator"@de . \\
    \end{code}
    
    resulting in the final \emph{ModelAsRequired} 
    \texttt{ex:buildICEExhaustionInnovationModel}:
    \begin{code}
    ex:ICEExhaustionInnovationModel \\
    \> a tc:Flow, tc:ModelAsRequired ; \\
    \> od:hasFlowType tc:ComplexFlow ; \\
    \> od:hasFlowSubstance ex:Fuel, ex:FuelAdditive, ex:Transmission, ex:Heat,  \\
    \> ex:Work, ex:Cooling, ex:Exhaust ; \\
    \> od:hasFlowFunctionality tc:CarrierFlow ; \\
    \> od:hasFlowDelegation tc:OpenFlow ; \\
    \> od:hasFlowSource tc:InternalSource ; \\
    \> od:hasStaticFlowComponent ex:Crankshaft, ex:ExhaustCamshaft,  \\
    \> ex:InletCamshaft, ex:Piston, ex:ConnectionRod, ex:SparkPlug,  \\
    \> ex:CombustionChamber, ex:IntakeValve, ex:ExhaustValve, \\
    \> ex:CoolingWaterJacket ; \\
    \> od:hasMethodOfDescription tc:ParametricDescription, ex:ICE3DModel ; \\
    \> od:generatesFlowTransformation ex:ExhaustionFlowTransformation ; \\
    \> skos:prefLabel "ICE exhaustion innovation model"@en, "innovatives  \\
    \> Abgasmodell des Verbrennungsmotors"@de . \\
    \end{code}
    \newpage
    \section{Critical review and further development}
        \subsection{Outside view - embedding the Flow Analysis Ontocard}
        For now this TRIZ Ontocard remains isolated from its embedded environment within the ontological hierarchy of the TOP and rules how and when to 
        apply a flow analysis method within the main process of inventive problem solving as it is proposed by the TRIZ model. A long term goal may be,
        that from an autonomous engineers perspective it should be possible to gain a programmatic manual how to solve a specific inventive or systemic
        problem only with the given RDF of the TRIZ Ontology, for which it has to be serializable.

        To embed the Flow Analysis Ontocard either a top down or a bottom up approach is applicable. For top down it needs to be clarified how
        Flow Analysis as one of the Top Level Ontologies fits in the structure with the remaining parts, especially LETS, the TRIZ application areas and
        the TRIZ Model. As it seems clear Eckardt already considered the connection to LETS and the TRIZ Model, for which these areas should be prioritized
        in the further development.
        
        Further, not only in terms of ontologization, it remains unclear where exactly the methodological framework of flow
        analysis is settled, how it can be derived or transformed from other system analysis methods, which methods from different approaches it includes
        or where other techniques may benefit from including flow analysis methods in terms of multiperspective modelling. There are severel concepts, 
        which use similar semantics or patterns to describe a problem, only on different levels of abstraction, especially those, which work on graph
        like structures. First mentioned here should be the process analysis framework and additionally function analysis as they intersect on a general
        conceptual level.

        Looking at processes, functions and flows in TRIZ terminology \cite{Souchkov2018} \cite{Eckardt2020} a flow model is a special process model, where the process is a
        function in time between each of the components of the process. \cite{KoltzeSouchkov2017} In the flow analysis model a flow is also a function, though it derives not
        as a function between components, but as an emergent property of its components properties, which do not have functional properties only, but relational
        properties too, especially the control unit and the channel, which can not be grouped by source or receiver components bijectively in time. 
        So it's neither a classical process model nor a function model in its strict sense, thus combines aspects of both as flow is a function in time between nodes, 
        which combine multiple properties of different components, but with interdependet properties like ratio of flow rate and capacity. In the status
        quo these aspects are not covered explicitely by the flow analysis model.

        Further, the concept of substance field models(SFM) is closely related to the FAM. 'Analog to the function model the components are also connected
        via a function, but in the substance field model the energy or a specific form of energy is added.' \cite{KoltzeSouchkov2017} Thus, the substance field model
        matches the systemic structure of flow composition, leading to its definition as a physical quantity as a product of a field and a face. \cite{Flussdefinition(Physik)}
        Even more, SFM uses the terminology of useful and harmful effects, too. \cite{KoltzeSouchkov2017} Therefore the linkage between those methods should be generalized in the
        future.

        Next mentioned should be its connection to the root conflict analysis. As already scribed in the introduction, cause and effect analysis and the
        identification of cause-effect-chains is a well described and established concept in TRIZ. As many rules of flow analysis and its techniques to
        change flows obviously imply a root conflict analysis, there is no ontological relation between these concepts and should be made explicit. 
        
        Yet, there are several other concepts in TRIZ that can be exploited using the flow analysis. Generally speaking, all models which describe 
        progress in time and map the ability and change of motion of some potential can be analyzed using the FAM. So it maybe should considered to
        review the concept of evolution of technical systems in terms of flow analysis, at least ontologically.
        
        While having focused mainly on the MATRIZ school, there exists a flow model in OTSM-TRIZ as well. Khomenko proposed a 'Problem Flow Approach',
        which uses a more abstract concept of flow. 

        

        \subsection{Inside view - methodological critics}
        Eckardts ontology of flow analysis as it is matches the basic methodological approach of TRIZ on how to change a problematic system in terms 
        of requirement engineering. It has a descriptive flow model and categorial connections what rules are needed to construct and evaluate a 
        flow model as it is and which have to be used to transform it into a flow model as it should be, while including some techniques how to change
        its static components to aim for the prefered requirements. Beside that basic descriptive perspective it does not meet the need on how
        to achieve demanded modifications of flow systems in specific cases, as it has been already noticed by Eckardt(algorithms and a matrix
        of methods of transformation of the model as is to the model as it should be as a possible next step \cite{Eckardt2020}').
        The proposed <<Techniques of Modification>> used by Eckardt for this transformation are based on the work of Lyubomirsky. While Eckardt established
        his categories of techniques how to increase useful flows and reduce of harmful flows by changing different static components, a more detailed analytic 
        point of view needs to adress which problematic effects can be solved using certain techniques. Further more, we need a map which decodes what
        properties of components can cause components to occur harmful properties, like grey zones, stagnation zones and so on. 
        But to establish these performative aspects the descriptive flow model needs to be improved as well. 
        Beside having properties for static components, which has already been regarded by Lyubomirsky and improved by Lebedev (like capacity or conductivity),
        \texttt{dynamic properties of flow}, for example resonance or flow rate, have to be
        defined to model how specific flow properties emerge out of the constellation of certain configuration of its static components and how they can be affected
        changing them.

        which change of components change which property of the flow
        how does the change of a flow property affect certain components
        
        This concludes, that an additionally level of abstraction is needed to account for the interconnection of flow and its components. The question
        is, how it can be modelled that a stagnation zone is only bad for useful flows but can be used to reduce harmful flows?
        At this point, only singular flows are issued by the FAM, meaning that it is not clear, if a global flow system is issued or a specific local flow.
        But it has to be considered, that synergistic effects on the one hand can cause contrary reactions on the other hand. A flow analysis therefore must 
        take the complexity of flow networks into account. Beside extending the descriptive flow model to spotlight on the systemic causation of these problems, 
        this is also a methodological issue. At the moment FAM focuses on methods, that emerge out of experienced based best practices in an engineering context.
        While establishing methodological concepts, that aim for generalization of categorial problem classes,
        they are mainly build for qualitive substance and energy flows only. Not to forget, that there are some blue prints adressing information flows, too, like
        the problem of conversion of flows and the concept of carrier and media flows, but these remain cursory.
        This rises the question: Which domains can be adressed by the actual model and which not? And further more, are the foundations of the TRIZ model sufficient to address complex problems in flow analysis, that cannot clearly distinct between
        useful and harmful flows. If we look at these techniques, they try to fulfill requirements orienting on one parameter at the time.
        But in practice, especially in a network of flows, you need to take multiple parameters into account, respectively distinguish between global and local
        parameterization. Therefore a quantization or numerical framework is needed, like it is applied by graph theoretical concepts and optimization theory. 
        
         



         

        How can the vocabulary be used to apply it to certain domains, especially those which doesn't seem to be intuitively compatible to the generic techniques
        of flow modification. 


        hard to define flows of flows

        flow analysis doesn't show the transformation as a process

        the definition of static component remains unflexible when 

\begin{thebibliography}{...}
    \raggedright
    \bibitem{Eckardt2020} Eckardt, Olga. Ontologische Diagramme. Flussmodell und Flussanalyse. TRIZ Ontology Project - The Webinar 2020. 27.10.2020, 19 Uhr (UTC+3). Translated to German by Hans-Gert Gräbe, Leipzig. 
    \bibitem{Flussdefinition(Physik)} Fluss (Physik).
        \url{https://de.wikipedia.org/wiki/Fluss_(Physik)}. Last visited 03 May 2021.
    \bibitem{Graebe2020} Hans-Gert Gräbe (2020).  Die Menschen und ihre Technischen Systeme. LIFIS Online.
        \url{doi:10.14625/graebe_20200519}.
    \bibitem{Graebe2021} Hans-Gert Gr\"abe (2021). About the WUMM modelling concepts of a TRIZ ontology.
        \url{https:\\github.com/wumm-project/Leipzig-Seminar/blob/master/Wintersemester-2020/Seminararbeiten/Anmerkungen.pdf}.
    \bibitem{KoltzeSouchkov2017} Koltze, Souchkov (2018). Systematische Innovation. TRIZ-Anwendungen in der Produkt- und Prozessentwicklung. Carl Hanser Verlag. München. 2017.
    \bibitem{Kuryan et al. 2020} Kuryan, Andrey. Rubin, Mikhail. Shchedrin, Nikolai. Eckardt, Olga. Rubina, Natalia. TRIZ Ontology. Currect State and Perspectives. Translated to English by Hans-Gert Gräbe, Leipzig, who aknowledges the support by the free version of deepl.com. Minsk, Belarus. August 21, 2020.
    \bibitem{Litvin et al. 2007} Litvin, Simon. Petrov, Vladimir. Rubin, Mikhail. International TRIZ Association(MA TRIZ). Frey, Victor. Altshuller Institute for TRIZ Studies. TRIZ Body of Knowledge. 2007.
    \bibitem{Lebedev2011} Lebedev, Yuri. Classification of flows in technical systems. 2011.
    \bibitem{Lebedyev2015} Yuri Lebedyev (2015). Tool improvement Flow Analysis. (Russian original translated with DeepL.com)
      \url{https:\\github.com/wumm-project/Leipzig-Seminar/blob/master/Wintersemester-2020/Seminararbeiten/Thoke/Lebedev-2015-en.pdf}.
    \bibitem{Lyubomirsky2006} Lyubomirsky, Alex. Law on efficiency improvement use of substance, energy and information flows. 2006.
    \bibitem{Petrov2007} Vladimir Petrov, Michail Rubin, Simon Litvin (2007). Fundamentals of TRIZ Theory of Inventive Problem Solving (in Russian). Reprinted in 2020. ISBN 978-5-4496-8183-6.
    \bibitem{Petrov2019} Petrov, Vladimir. TRIZ. Theory of Inventive Problem Solving - Level 1. Springer International Publishing. Ra'anana, Israel. 2019.
    \bibitem{Souchkov2018} Valeri Souchkov (2018).  Glossary of TRIZ and TRIZ-related terms. Version 1.2.
        \url{https:\\matriz.org/wp-content/uploads/2016/11/TRIZGlossaryVersion1_2.pdf}. 
    \bibitem{SKOS} SKOS -- The Simple Knowledge Organization System.
        \url{https:\\www.w3.org/TR/skos-reference/}.
    \bibitem{TOP} The TRIZ Ontology Project (TOP) \foreignlanguage{russian}{Онтология ТРИЗ} of the TRIZ Developer Summit.
        \url{https:\\triz-summit.ru/Onto_TRIZ/}.
    \bibitem{TOP-Glossary} TRIZ 100 Glossary. A short glossary of key TRIZ concepts and terms (in Russian).
        \url{https:\\triz-summit.ru/onto_triz/100/}.
    \bibitem{WUMM} The WUMM Project. https:\\wumm-project.github.io/About.html. Last visited 22 February 2021.
    \bibitem{WUMMTOP} The WUMM TOP Companion Project. https:\\wumm-project.github.io/Ontology. Last visited 22 February 2021.
    \bibitem{RDF} The W3C Resource Description Framework (RDF). https:\\www.w3.org/RDF/. Last visited 27 April 2021.
    \bibitem{RDFC} RDFwithContext. A proposal . 
        \url{https:\\www.w3.org/2011/rdf-wg/wiki/RDFwithContexts}. Last visited 02 May 2021.
    \bibitem{RDFData} The RDFData git repository. https:\\github.com/wumm-project/RDFData.  Last visited 22 February 2021.
    \bibitem{SPARQL} The Virtuoso SPARQL Query Editor hosted by the WUMM Project. http:\\wumm.uni-leipzig.de:8891/sparql. Last visited 22 February 2021. 
\end{thebibliography}
\end{document}