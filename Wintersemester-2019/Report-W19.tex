\documentclass[11pt,a4paper]{article}
\usepackage{ls,amsmath,bbm,wrapfig}
\usepackage[utf8]{inputenc}
\usepackage[german]{babel}

\title{Seminarreport zum Seminar Systemtheorie\\[6pt] an der Universität
  Leipzig\\ im Wintersemester 2019/20}

\author{Hans-Gert Gr\"abe, Ken Pierre Kleemann}

\date{2. Dezember 2020}

\begin{document}
\maketitle
\tableofcontents
\clearpage

\section{Ziel und Methodik des Seminars}

\subsection{Zielstellung}

Der Systembegriff spielt in der Informatik eine herausragende Rolle, wenn es
um Datenbanksysteme, Softwaresysteme, Hardwaresysteme, Abrechnungssysteme,
Zugangssysteme usw. geht.  Überhaupt wird die Informatik von einer Merhheit
als die „Wissenschaft von der \emph{systematischen} Darstellung, Speicherung,
Verarbeitung und Übertragung von Informationen, besonders der automatischen
Verarbeitung mithilfe von Digitalrechnern“ (Wikipedia) verstanden.  Auch
gewisse einschlägige Professionen wie etwa der \emph{Systemarchitekt} genießen
unter IT-Anwendern hohe Wertschätzung.

Die Bedeutung des Systembegriffs reicht allerdings weit über den Bereich der
Informatik hinaus -- er ist grundlegend für alle Ingenieurwissenschaften und
als \emph{Systems Engineering} mit der ISO/IEC/IEEE-15288 Norm „Systems and
Software Engineering“ auch Gegenstand internationaler Normierungs- und
Standardisierungsprozesse.  Mehr noch spielt der Systembegriff auch bei der
Beschreibung komplexer natürlicher und kultureller Prozesse -- etwa im Begriff
des \emph{Ökosystems} -- eine zentrale Rolle.

Mit dem \emph{Semantic Web} rückt die Bedeutungsanalyse digitaler Artefakte in
den Mittelpunkt, die in letzter Instanz Sprachartefakte sind und damit
ebenfalls in direktem Zusammenhang zu einem sinnvoll zu entfaltenden
\emph{Systembegriff} stehen als Grundlage jeden Verständnisses konkreter
Systeme.

Mit dem Schlagwort \emph{Nachhaltigkeit} werden schließlich komplexe
gesellschaftliche Abstimmungsprozesse angesprochen, mit denen vielfältige
Informations- und Bewertungsprobleme einhergehen. Hierbei ist die Fähigkeit
der beschreibenden Abgrenzung, Entwicklung und Steuerung von sogenannten
Systemen auf bzw. über verschiedene Governance-, Raum- und Zeitebenen hinweg
von großer Bedeutung.

\begin{quote}
  \textbf{Ziel des Seminars} ist es, ein besseres Verständnis für diese
  Vielfalt von Systembegriffen zu gewinnen und dabei die Zugänge
  \emph{verschiedener Systemtheorien} als Gegenstand einer
  \emph{Systemwissenschaft} zu analysieren.
\end{quote}

\subsection{Inhaltliche Abgrenzung}

Das Seminar ist ein Einführungskurs in die Systemwissenschaft auf Master-Ebene
und thematisiert deren Entwicklung im Laufe der Zeit, Verzweigung von
Ansätzen, Schlüsselbegriffen und Konzepten.  \emph{Systemwissenschaft} wird
hier als übergeordneter Ausdruck für ein Feld verwendet, zu dem zahlreiche
Gelehrte aus den verschiedensten Disziplinen wie Anthropologie, Biologie,
Chemie, Ökologie, Ökonomie, Mathematik, Physik, Psychologie, Soziologie und
andere beigetragen haben. Entwicklungen wie Kybernetik, Chaostheorie oder
Netzwerkanalyse und -wissenschaft können als Teil von Systemwissenschaft oder
zumindest stark verwandt mit ihr angesehen werden.  Einige Zweige der
Systemwissenschaft gelten in Deutschland sogar als neue Wissenschaftsbereiche
mit eigenen Rechten wie Synergetik oder Komplexitätswissenschaft.

Diese Entwicklungen haben neue Möglichkeiten für eine verbesserte Analyse und
Entscheidungsfindung in wissenschaftlichen, geschäftlichen und politischen
Bereichen eröffnet. Wir stellen jedoch täglich fest, dass in komplizierten
Situationen, insbesondere in der Politik und in der Wirtschaft, einfache und
direkte Entscheidungsfindungsprozesse nach wie vor überwiegen, was zu einer
Zunahme negativer Entwicklungen führt, wenn die ursprünglich beabsichtigten
Wirkungen nicht eintreten. Jede unerwartete Nebenwirkung oder Gegenreaktion,
die die Maßnahmen unbrauchbar machen, sind ein klares Indiz dafür, dass die
grundlegenden mentalen Modelle der Akteure unvollständig waren und breitere
systemische Korrelationen vernachlässigt wurden. Das Systemdenken ist daher
von besonderer Bedeutung für den Übergang zu einer nachhaltigeren
Gesellschaft.

\subsection{Methodik}

In diesem Seminar sollen die historische Entwicklung der Systemwissenschaft
(in Teilen) verfolgt sowie relevante Grundbegriffe studiert werden.  Wir
halten uns dabei an kein spezifisches Modell (wie z.B. \emph{Systemdynamik}),
sondern entwickeln ein tieferes Verständnis für die Systemwissenschaft und für
eine spezifische Art des „Systemdenkens“, mit der Nachhaltigkeitsprobleme
erfolgreicher angegangen werden können. Dies erreichen wir durch das Lesen und
Diskutieren von wissenschaftlichen Arbeiten und Buchkapiteln.

Von den Studierenden wird erwartet, dass sie sich aktiv am Seminar beteiligen
durch Seminardiskussionen, Präsentationen, schriftliche Ausarbeitungen und
nicht zuletzt durch Lesen. Die Kursteilnehmer werden angeregt und
aufgefordert, einen eigenen Zugang zum Thema Nachhaltigkeit zu entwickeln.

\subsection{Curriculare Einordnung}

Das Seminar ist Teil des Vertiefungsmoduls \emph{Semantic Web}, zu dem weiter
eine Vorlesung \emph{Nachhaltigkeit und Digitale Skills} sowie ein
Online-Praktikum mit dem Minsker TRIZ-Trainer gehören. Im Seminar wollen wir
uns dem für die TRIZ-Methodik zentralen Begriff des \emph{Systems} nähern und
dazu verschiedene Quellen auswerten. 

\subsection{Plan des Kurses}

Das \emph{Semantic Web} erweitert das Web, um Daten zwischen Rechnern
einfacher austauschbar und verwertbar zu machen; so kann beispielsweise der
Begriff „Bremen“ in einem Webdokument um die Information ergänzt werden, ob
hier ein Schiffs-, Familien- oder der Stadtname gemeint ist. Diese
zusätzlichen Informationen explizieren die unstrukturierten Daten. Zur
Realisierung dienen Standards zur Veröffentlichung und Nutzung
maschinenlesbarer Daten (insbesondere RDF).

Das ist eine \emph{sehr technizistische} Sichtweise, der nicht
berücksichtigt, \emph{warum} diese Unterscheidungen überhaupt relevant
sind. In diesem allgemeineren Sinne \textbf{befassen sich Semantic Web
Technologien mit der werkzeuggestützten Schärfung der Bedeutung von
Begriffen in konkreten Kontexten}.

In diesem Kurs stehen die Voraussetzungen und Bedingtheiten im Vordergrund,
die mit der Schärfung der Bedeutung von Begriffen in konkreten Kontexten
verbunden sind. Wir werden dies auf dem Kontext der Nachhaltigkeitsdebatte
genauer studieren. Mit \emph{technischen Fragen des Semantic Web} werden wir
uns allenfalls am Rande beschäftigen.

Der Kurs ist als \textbf{interdisziplinäres akademisches Lehrangebot}
konzipiert, an dem Kolleginnen und Kollegen aus drei Bereichen beteiligt sind
-- Lydie Laforet und Sabine Lautenschläger vom IIRM, Ken Pierre Kleemann
(Philosohpie) und Hans-Gert Gräbe (Informatik).

\textbf{Akademisch} bedeutet, dass wir miteinander -- besonders im Seminar --
auf Augenhöhe verhandeln wollen und werden. Es geht um rationale
Argumentationen auf einem wissenschaftlichen Niveau, also um \emph{Argumente}
und nicht um apodiktische Wahrheiten.

Der Kurs besteht aus drei Teilen:
\begin{itemize}
\item Die \textbf{Vorlesung} (do 11-13 Uhr). Dort werden grundlegende
  Begriff\-lichkeiten wie Technik, Nachhaltigkeit im Kontext der bürgerlichen
  Gesellschaft, digitales Universum, RDF Basics, Ontologien (im Sinne der
  Informatik), Storytelling, Daten, Information, Wissen sowie kooperatives
  Handeln im digitalen Zeitalter genauer entwickelt.
\item Das \textbf{Seminar} „Systemwissenschaft“ (di 15-17 Uhr). Hier werden
  auf dem Hintergrund des vielfach überladenen Systembegriffs
  Begriffsbildungsprozesse in ihrer theoretischen wie praktischen Dimension
  studiert. Die Herangehensweise der interdisziplinären Partner wird sich
  dabei unterscheiden, wie bereits in der Vorbereitung des Seminars deutlich
  wurde. Während für die Kolleginnen am IIRM die \emph{eigenen Praxen der
    Einbindung} in konkrete sozio-politische Prozesse um die Sicherung von
  Nachhaltigkeit im Vordergrund stehen, geht es dem Informatiker stärker um
  die \emph{sozio-technischen Bedingtheiten} von Begriffsbildungsprozessen als
  Kern von Semantic Web Technologien und dem Philosophen um die \emph{Dynamik
    von Begriffsbildungsprozessen} als solchen auf der Basis
  praktisch-performativer Erfahrungen, wie sie die Kolleginnen vom IIRM und
  auch die Studierenden mitbringen.
\item Dabei werden wir immer wieder auf widersprüchliche Situationen stoßen,
  die genauer sprachlich zu analysieren sind, um sinnvolle Lösungen zu
  finden. TRIZ als (nicht nur) Erfindungsmethodik bietet hierfür ein
  umfassendes Instrumentarium an. Im \textbf{Praktikum} steht die Vermittlung
  grundlegender Fertigkeiten im Umgang mit dieser Methodik im Mittelpunkt,
  wobei der Minsker TRIZ-Trainer als Online-Kurs zum Einsatz kommt.  Darüber
  hinaus werden wir einige Präsenztermine des Praktikums (di 17-19 Uhr)
  nutzen, um sowohl TRIZ-Grundlagen als auch technischere Semantic Web Aspekte
  in dem Umfang zu besprechen, wie es sich als notwendig herausstellen wird.
\end{itemize}

\section{Systembegriff in der Theorie dynamischer Systeme (Gräbe)}

Literatur: \cite{Prigogine1993}, Zusatzliteratur: \cite{Jantsch1992},
\cite{Jooss2017}

\subsection{Fragestellungen der Theorie dynamischer Systeme}
\begin{quote}  
  Ein \emph{dynamisches System} ist eine abgegrenzte zeitabhängige
  Funktionseinheit, die durch ihre Signaleingänge und Signalausgänge in einer
  Wechselwirkung mit der Umwelt steht. 

  Das System hat mindestens einen Signaleingang und einen Signalausgang und
  reagiert zu einem bestimmten Zeitpunkt auf ein beliebiges Eingangssignal mit
  einer bestimmten zeitlichen Reaktion als Ausgangssignal. (Wikipedia)
\end{quote}
Das System kann in der Regel Informationen über vorherige Ereignisse durch
interne Strukturtransformation speichern und entwickelt so ein „Gedächtnis“.
\begin{quote}
  Das Verhalten dieser Systeme kann linear, kontinuierlich nichtlinear,
  diskontinuierlich nichtlinear, zeitinvariant, zeitvariant und
  totzeitbehaftet sein. Dies gilt für Eingrößen- und Mehrgrößensysteme.
  (Wikipedia)
\end{quote}

\subsection{Erste Beispiele}

Beispiele im homogenen Gravitationsfeld (aus der Wikipedia)
\begin{itemize}
\item[(1)] Pendel. Einfacher mathematischer Zusammenhang mit festem
  Bezugspunkt führt zu „einfachem“ Verhalten.
\item[(2)] Gekoppelte Pendel. Kopplung von zwei Zusammenhängen nach (1) führt
  zu einer Reihe qualitativ verschiedener Kopplungsphänomene (Mitschwingen,
  Gegenschwingen, Schwebung).
\item[(3)] Doppelpendel.  Kopplung mit bewegtem Endpunkt als zweiter
  Anfangspunkt führt bereits zu chaotischen Trajektorien, da über die Kopplung
  in das zweite System Energie aus dem ersten eingetragen wird (bei kleinen
  Ausschlägen synchronisieren sich die Pendel nach einere Weile).
\item[(4)] Magnetisches Pendel. Pendel mit drei punktförmigen Attraktoren,
  Energieeintrag hängt von den Entfernungen des Pendelendes von den drei
  Magneten ab. Im Ergebnis pendelt das Ende längere Zeit um jeweils einen der
  Magneten, bis es chaotisch zu einem anderen Magneten wechselt.
\end{itemize}

Beispiele mit gravitativer Wechselwirkung (auch Wikipedia)
\begin{itemize}[noitemsep]
\item Zweikörperproblem
\item Dreikörperproblem und das Kolmogorow-Arnold-Moser-Theorem. Dieses
  besagt, dass fast alle Trajektorien quasiperiodisch sind, dazwischen aber
  immer wieder kompliziertere Trajektorien liegen. Beispiel: Saturnringe. 
\end{itemize}
Das sind bereits -- notwendigerweise reduktionistische -- Beschreibungsformen
der Wirklichkeit: Etwa Pendel mit und ohne Dämpfungsglied.

Aber: Sinnvolle Reduktionen von Beschreibungsformen \textbf{verbessern} unsere
Einsicht in die Zusammenhänge der Welt. Hätte Galileo Galilei diese
Denkmethodik nicht angewendet, wäre ihm niemals aufgefallen, dass Eisen und
Feder gleich schnell fallen, weil dies der praktischen Erfahrung widerspricht
und erst nach einem Abstraktionsprozess deutlich wird. Dann kann es, unter
Herstellung entsprechender idealer Bedingungen, aber auch im Experiment
überprüft werden. 

\begin{itemize}[noitemsep]
\item Nicht alles, was wie Chaos aussieht, muss auch Chaos sein:\\
  \url{https://i.redd.it/zr7tet9mdfl01.gif}
\end{itemize}

\subsection{Grenzzyklen und Attraktoren}

\begin{itemize}[noitemsep]
\item Grenzzyklen
\item Attraktor als stabile zeitinvariante Lösung des entsprechenden
  Differentialgleichungs-Systems 
\end{itemize}

\emph{Beispiel:} Die Attraktoren des Magnetpendels sind die drei stabilen
Endlagen, also drei Punkte im Phasenraum.

Weitere Phänomene:
\begin{itemize}[noitemsep]
\item Hysterese. Beispiel: Temperaturregelung einer Heizungsanlage
\item Räuber-Beute-Zyklen (Wikipedia), Lotka-Volterra-Gleichungen,
  Lotka-Volterra-Regeln
\end{itemize}

Zur Bedeutung „stabiler“ zyklischer Prozesse in der Natur.

Wir sind in der Lage, solche sich \emph{näherungsweise} wiederholenden Muster
in natürlichen Prozessen (d.h. Attraktoren) wahrzunehmen, also auch unabhängig
von der Mathematik eine solche Reduktionsleistung zu vollbringen.

Frage: Wie kompliziert können solche Attraktoren werden?

\emph{Beispiel:} Der Lorenzattraktor.
  
Achtung, im Gegensatz zum Magnetpendel entsteht das Bild \emph{nicht} durch
Bahnverfolgung, sondern stellt wirklich den Attraktor als \emph{globales}
Artefakt dar, als invariante Lösung des (recht einfachen, allerdings
nichtlinearen) Systems aus drei Differenzialgleichungen.

Es geht noch komplizierter: \emph{Seltsame Attraktoren} als „Endzustand eines
dynamischen Prozesses, dessen fraktale Dimension nicht ganzzahlig und dessen
Kolmogorov-Entropie echt positiv ist. Es handelt sich damit um ein Fraktal,
das nicht in geschlossener Form geometrisch beschrieben werden kann“.
(Wikipedia)

\textbf{Damit ist der Trajektorienbegriff der klassischen Physik für derartige
  Phänomene nicht mehr anwendbar}. Damit greift zugleich die klassische
Interpretation des „Schmetterlingseffekts“ zu kurz, denn sie geht von der
Existenz einer Trajektorie aus, längs derer (Mono)-Kausalität vermittelt wird.

\subsection{Systeme auf multiplen Zeitskalen}

Ein wichtiger Ansatz ergibt sich für Systeme, deren Dynamiken auf
verschiedenen Zeitskalen ablaufen. Man kann dann methodisch als weiteren
Abstraktionsschritt zunächst die Dynamiken auf den einzelnen Zeitskalen
untersuchen und später in einem erweiterten Modell die Wechselwirkungen
zwischen den Zeitskalen hinzunehmen. Massiv neue Phänomene ergeben sich
bereits bei der Betrachtung von \emph{zwei} Zeitskalen, was als \emph{Mikro-
  und Makroevolution} bezeichnet wird. Hier wird es in der Wikipedia bereits
dünn.

\begin{itemize}[noitemsep]
\item Beispiel: Doppelpendel als Pendel, aber der Pendelkörper hat selbst noch
  eine innere Dynamik.

  Das Obersystem prägt dem Untersystem durch Energieeintrag eine gemeinsame
  Dynamik auf. Obwohl das Doppelpendel eigentlich chaotisch ist, ist das
  System damit (final) \emph{nicht} chaotisch, sondern verhält sich (bei
  kleinen Ausschlägen) wie ein einfaches Pendel mit Masse im Schwerpunkt.
\item In der Literatur als „Versklavungseffekt“ bekannt und besonders in
  methodisch schlecht fundierten soziologischen Betrachtungen als
  Verbalargument verbreitet.
\item Siehe aber \url{https://de.wikipedia.org/wiki/Netzwerkforschung}
\end{itemize}

\emph{Projektionsphänomene:} Entstehung von Singularitäten bei Reduktion: Die
glatte räumliche Trajektorie $\{(t,t^2,t^3)|t\in\mathbbm{R}\}$ geht bei
Projektion in die $y$-$z$-Ebene in die Kurve $\{(t^2,t^3)|t\in\mathbbm{R}\}$
über. Die Kurve wird durch die Gleichung $y^3=z^2$ beschrieben und hat im
Ursprung eine Singularität (eine Spitze).

Welche Probleme treten beim Zusammensetzen von (verstandenen) Mikroevolutionen
von Teilsystemen zu einem Verständnis der Dynamik auf der Makroebene auf?

\subsection{Immersiver und submersiver Systembegriff}

Komplexere Relationen von Systemen $S_1$ und $S_2$ innerhalb eines Obersystems
$S$ kann man als \emph{Einbettung} der beiden Teilsysteme in das Obersystem
beschreiben.

Mathematische Formulierung der Fragestellung: 
\begin{quote}
  Gegeben sind die beiden Systeme $S_1$ und $S_2$ (als Instanzen einer
  mathematischen Kategorie, zum Beispiel Mengen oder Vektorräume).
  
  Gesucht sind für jedes denkbare Obersystem $S$ geeignete Abbildungen $f_1:
  S_1 \rightarrow S$, $f_2: S_2 \rightarrow S$, die diese „Einbettung“
  realisieren.
\end{quote}
„Einbettung“ steht hier für beliebige Morphismen in der gegebenen Kategorie. 

\emph{Frage:} Gibt es für diese Konstellation ein \emph{universelles}
kategorielles Objekt, d.h. ein in dieser Kategorie allein aus $S_1$ und $S_2$
konstruierbares universelles $U$ samt universellen Abbildungen $p_1: S_1
\rightarrow U$, $p_2: S_2 \rightarrow U$, so dass sich für \emph{jedes} Tripel
$(f_1, f_2, S)$ die obige Konstellation als
\begin{gather*}
  f_1 = f \circ p_1: S_1 \rightarrow U \rightarrow S,\quad f_2 = f \circ p_2:
  S_2 \rightarrow U \rightarrow S
\end{gather*}
für ein geeignetes $f = f_1\oplus f_2: U \rightarrow  S$ schreiben lässt?

$U$ heißt in dem Fall \emph{direkte Summe} und man schreibt $U = S_1\coprod
S_2$.

\paragraph{Mathematische Kategorien.}
Die meisten mathematischen Modelle bewegen sich in derartigen konkreten
Kategorien, zum Beispiel der Kategorie der Mengen, der Vektorräume, der
Faserbündel, der algebraischen Varietäten usw.

Jede solche Kategorie zeichnet sich dadurch aus, dass dort die Begriffe
\emph{Objekt} und \emph{Morphismus} eine klare Bedeutung haben.  Morphismen
zwischen Vektorräumen sind zum Beispiel operationstreue Abbildungen, also
lineare Abbildungen, die sich für endlichdimensionale Vektorräume durch
Matrizen beschreiben lassen.

Nicht in jeder Kategorie existieren solche universellen Objekte.

\emph{Anmerkung:} Die Konstruktion lässt sich leicht auf endlich viele $S_i$
und sogar auf unendlich viele $S_i, i\in I$, verallgemeinern, und so ist es in
der Mathematik auch gemeint.

\paragraph{Kategorie der Mengen.}
In dieser Kategorie existieren direkte Summen $U$ sowohl für endliche als auch
unendliche Indexmengen $I$. Dies ist gerade die \emph{disjunkte Vereinigung}
der Mengen $S_i$.

Die Abbildungen $p_i$ sind gerade die Einbettungen $p_i: S_i \rightarrow U$
der Teilmengen in deren disjunkte Vereinigung.

Die Abbildung $f: U \rightarrow S$ ergibt sich wie folgt: Für jedes $a\in U$
existieren genau ein $i$ und ein $a'\in S_i$ mit $a=p_i(a')$. Setze
$f(a)=f_i(a')$.

Ist $|S_1| = a$, $|S_2| = b$, so ist $|S_1\coprod S_2| = a+b$.  Das Ganze ist
nicht mehr als die Summe seiner Teile.

\paragraph{Kategorie der Vektorräume.}
Auch in dieser Kategorie existieren direkte Summen $U$ sowohl für endliche als
auch unendliche Indexmengen $I$.

Die Abbildungen $p_i$ sind gerade die Einbettungen $p_i: S_i \rightarrow U$
auf die $i$-te Koordinate von $U$ (alle anderen Komponenten gleich dem
Nullelement des Vektorraums).

Die Abbildung $f: U \rightarrow S$ ergibt sich wie folgt: Jedes $a\in U$ lässt
sich eindeutig als (endliche!) Linearkombination von Basisvektoren $e_a$
darstellen, von denen die Bilder $f(e_a)$ bekannt sind. Setze nun $f$ linear
fort.

Ist $|S_1| = a$, $|S_2| = b$, so ist $|S_1\coprod S_2| = a \cdot b$.  Das
Ganze scheint mehr als die Summe seiner Teile zu sein. Ich komme darauf
zurück.

\paragraph{Ein zentrales TRIZ-Prinzip ist Prinzip 13 der Funktionsumkehr.}
Wenden  wir dieses hier an, drehen alle Pfeile um, und schauen, was wir
erhalten. Das Verfahren ist in der Mathematik weit verbreitet und heißt in
diesem Kontext „Übergang zur dualen Kategorie“.

Gesucht sind also nun $f_1: S_1 \leftarrow S$, $f_2: S_2\leftarrow S$, d.h.
Möglichkeiten, $S_1$ und $S_2$ aus einem Obersystem $S$ durch „Projektion“ zu
gewinnen.

Gibt es für diese Konstellation ein universelles kategorielles Objekt,
d.h. ein universelles $U$ und universelle Abbildungen $p_1: S_1 \leftarrow U$,
$p_2: S_2 \leftarrow U$, so dass sich für jedes Tripel $(f_1, f_2, S)$ die
obige Konstellation als
\begin{gather*}
  f_1 = p_1 \circ f: S_1 \leftarrow U \leftarrow S,\quad f_2 = p_2 \circ f:
  S_2 \leftarrow U \leftarrow S
\end{gather*}
für ein geeignetes $f = f_1\otimes f_2: S \rightarrow U$ schreiben lässt.

$U$ heißt in dem Fall \emph{direktes Produkt} und man schreibt $U = S_1\prod
S_2$. 

Wie ändert sich dabei die Perspektive auf den Systembegriff?

\paragraph{Kategorie der Mengen.}
In dieser Kategorie existieren direkte Produkte $U$ sowohl für endliche als
auch unendliche Indexmengen $I$. Dies ist gerade das \emph{karthesische
  Produkt}.

Die Abbildungen $p_i$ sind gerade die Projektionen $p_i : U \rightarrow S_i$
des Produkts auf die einzelnen Komponenten.

Die Abbildung $f : S \rightarrow U$ ergibt sich wie folgt: Für jedes $a\in S$
ist $f(a) = (f_i(a))\in U$.

Ist $|S_1| = a$, $|S_2| = b$, so ist $|S_1\prod S_2| = a\cdot b$.

Das Ganze ist deutlich mehr als die Summe seiner Teile, der größte Teil der
„Information“ ist relationaler Natur.

\paragraph{Kategorie der Vektorräume.}
Auch in dieser Kategorie existieren direkte Produkte $U$ sowohl für endliche
als auch unendliche Indexmengen $I$.

Für endliche Indexmengen unterscheiden sich die direkte Summe und das direkte
Produkt von Vektorräumen (auf den ersten Blick) nicht.

Das direkte Produkt besteht aber nur aus Vektoren, die nur an endlich vielen
Stellen von null verschiedene Einträge haben, das direkte Produkt ist das
volle kartesische Produkt.

Ist der Grundkörper abzählbar (etwa $K=\mathbbm{Q}$) und $I$ auch, so bleibt
die direkte Summe abzählbar, das direkte Produkt ist aber bereits
überabzählbar.

Auch hier enthält also das direkte Produkt deutlich mehr Information als die
direkte Summe.

\paragraph{Submersive und immersive Systemtheorien.}
Systemtheorien machen selten einen Unterschied zwischen diesen beiden
Zugängen.  Zur Unterscheidung der Zugänge bezeichnet man Systemtheorien, in
denen das erste Modellierungsprinzip dominiert, als \emph{immersive}
Systemtheorien. Man erkennt sie daran, dass ihre Konstruktionen wesentlich auf
Einbettungen (Immersionen) aufbauen.

Systemtheorien, die auf dem zweiten Modellierungsprinzip aufbauen, bezeichnet
man als \emph{submersive} Systemtheorien.  Man erkennt sie daran, dass ihre
Konstruktionen wesentlich auf Projektionen (Submersionen) aufbauen und damit
auf Prozessen gestaffelter Komplexitätsreduktion.

Die Theorie dynamischer Systeme ist eine submersive Systemtheorie.

\subsection{Emergente Phänomene}

„Das Ganze ist mehr als die Summe seiner Teile“. Dies trifft \emph{nur} auf
den submersiven Systembegriff zu, ist allerdings ein wesentliches
Konstitutionsprinzip komplexer Systemzusammenhänge.  Dazu folgende Themen (an
Hand der entsprechenden Wikipedia-Einträge):

\begin{itemize}[noitemsep]
\item Nichtlineare Systeme und Phasenübergänge.
\item Selbstorganisation in dissipativen Strukturen
  \begin{itemize}[noitemsep]
  \item Rayleigh-Bénard-Konvektion (Bénardzelle)
  \item Belousov-Zhabotinsky-Reaktion
  \end{itemize}
\item Dissipative Strukturen
\item Temperatur als emergentes Phänomen
\item {Entropie} und {Enthalpie}.
\item Die Erde als dissipatives System
\end{itemize}

\section{Einführung in Systemwissenschaft, Nachhaltigkeit und Allgemeine
Systemtheorie (Lautenschläger)}

Literatur: \cite{Bertalanffy1950}, \cite{Mele2010}, \cite{Binder2013}

Ziel dieses Termins war es u.a., eine Liste von Begriff\-lichkeiten
aufzustellen, die in Systemtheorien immer wieder auftreten und für das
Verständnis der Seminarinhalte zentral sind.

Im Seminar wurde das Thema in zwei Vorträgen von Sabine Lautenschläger und
Lydie Laforet vorgestellt.

Im folgenden Text werden einige Aspekte der Theorie dynamischer Systeme (TDS)
mit den von Lautenschläger und Laforet vorgetragenen Systemtheorieansätzen
abgeglichen und damit zugleich einige Punkte der TDS genauer ausgeführt.

\subsection{Bertalanffys Allgemeine Systemtheorie}

Bertalanffy entwickelt in \cite{Bertalanffy1950} zunächst die Grundlagen der
TDS im Verständnis jener Zeit.  Der Bezugstext steht damit ganz am Anfang
einer stürmischen Entwicklung der TDS in den 1960er und 1970er Jahren, die zu
fundamental neuen Einsichten in die Vielfalt von Formen der Lösungen
gewöhnlicher Differentialgleichungssysteme geführt haben.  Bereits in diesem
Gebiet\footnote{In den Gleichungen werden nur zeitabhängige Ableitungen
  zugelassen, keine partiellen Ableitungen nach auch noch anderen Parametern,
  das Gebiet der \emph{partiellen Differentialgleichungen} wird also noch
  nicht betreten.} finden sich erstaunliche Phänomene wie der Lorenzattraktor,
deterministisches Chaos, das Ende des Trajektorienbegriffs und fraktale
Gebilde. Mit partiellen Differentialgleichungen kommen noch
Solitonen\footnote{Auf dieses Phänomen bin ich in meinem Seminar nicht
  eingegangen, obwohl diese Strukturen, die in vielen Systemen partieller
  Differentialgleichungen als Lösungen auftreten, zu einem vollkommen neuen
  Verständnis des Welle-Teilchen-Dualismus führen. Siehe dazu
  \url{https://de.wikipedia.org/wiki/Soliton}. } hinzu. \cite{Bertalanffy1950}
vermittelt also nur eine erste Ahnung möglicher Phänomene. Die mathematischen
Betrachtungen verwenden allein Taylorreihen und beschränken sich damit auf
Phänomene nahe einer Gleichgewichtslage, können also mathematisch auf
Fließgleichgewichte (ohne wesentlich vereinfachende Annahmen) nicht einmal
angewendet werden.

Seine wissenschaftstheoretischen Überlegungen fußen auf der Analogie
entsprechender mathematischer Beschreibungsformen in verschiedenen
Wissenschaftsgebieten\footnote{Komplexe Systemtheorie stellt die Adäquatheit
  derartiger Beschreibungen heute selbst in Frage.} und stellen damit nach
meinem Verständnis auf \emph{methodologische} Ähnlichkeit von Zugängen und
\emph{nicht} auf Isomorphie von Strukturen (so Lautenschläger) ab. Dass
Bertalanffys Zugang \emph{deduktiv} sei, kann sich damit auch maximal auf den
mathematisch-deduktiven Kern seiner Argumentation beziehen, nicht aber auf die
weitergehenden wissenschaftstheoretischen Beobachtungen, bzw. dies wäre noch
genauer zu belegen.

\subsection{Der Raumbegriff der TDS}

Der Raumbegriff der TDS entwickelt sich aus dem physikalischen Begriff des
\emph{Phasenraums}. So „lebt“ ein klassisches Vier-Teilchen-System in einem
12-dimensionalen Phasenraum, der durch die $4\times 3$ Raumkoordinaten
aufgespannt wird. Derartige Phasenräume dienen zunächst der Koordinatisierung
der Bewegungsgleichungen, allerdings sieht bereits die Physik in solchen
Koordinatenabhängigkeiten einen Mangel, da die Gesetze unter
Koordinatentransformationen invariant sein müssen, also letztlich
koordinatenfreie Beschreibungen mehr Einsicht in bestehende Zusammenhänge
vermitteln. Damit steht zugleich die Frage, invariante geometrische Strukturen
in solchen höherdimensionalen Phasenräumen zu beschreiben.

Derartige Fragen sind Gegenstand zum Beispiel der algebraischen Geometrie oder
der Differentialgeometrie. In diesen Beschreibungen (der invarianten
geometrischen Gebilde) treten ihrerseits Räume auf, die sich etwa im Konzept
der \emph{Vektorbündel} „materialisieren“ als \emph{Sprache}, um geometrische
Eigenschaften der betrachteten invarianten Strukturen zu beschreiben (wie
Fasern, Keime, Schnitte, Obstruktionen zur Fortsetzbarkeit von Schnitten,
Homologieklassen als Strukturen derartiger Obstruktionen usw.).

Im Bereich der Analysis wird der Raumbegriff weiter verallgemeinert zu
unendlich-dimensio\-nalen Banach- und Sobolev-Räumen, in denen sich gewisse
mathematische Konzepte (etwa das Lebesgue-Integral) überhaupt erst entfalten
lassen für Situationen, wo man mit „klassischen“ Lösungen nicht mehr
weiterkommt.  Theorien (wie etwa der Banachsche Fixpunktsatz) lassen sich
überhaupt erst auf der Basis derart verallgemeinerter Raumbegriffe konsistent
entwickeln.

\subsection{Steady State und Fließgleichgewichte}

Diese Begriffe entwickeln sich später zum Begriff des \emph{Attraktors}
weiter.  Zugleich wird erkannt, dass derartige Attraktoren extrem komplexe
Gestalt haben können, womit eine Unterscheidung zu chaotischem Verhalten
allein auf phänomenologischer Ebene schwierig wird.  Zugleich wird die Rolle
auch \emph{negativer Attraktoren} erkannt.  Derartige Strukturen und
Strukturbildungsprozesse sind typisch für dissipative Prozesse fern von
Gleichgewichtszuständen, die durch einen gewissen Durchsatz von Materie und
Energie getrieben werden. Der Durchsatz von Information spielt dabei keine
Rolle\footnote{Siehe dazu etwa noch einmal
  \url{https://de.wikipedia.org/wiki/Dissipative_Struktur}.}. Ich komme unten
auf diese Frage zurück.

\subsection{Komplexe und komplizierte Systeme}

Diese Unterscheidung habe ich überhaupt nicht begriffen. Sicher kann man einen
solchen Unterschied nicht an der Zerlegbarkeit eines technischen Artefakts
(„ein Auto ist kompliziert, nicht aber komplex“) festmachen, da ein
entsprechender Technikbegriff noch deutlich hinter dem des VDI
\cite{VDITechnik} zurückbliebe, der zum System wenigstens noch „Herstellung“
und „Verwendung“ des Artefakts (oder -- dort bereits deutlich --
„Sachsysteme“) rechnet.

Eine solche Unterscheidung lässt sich nach meinem Verständnis ausschließlich
an den Beschreibungsmethodiken festmachen, die etwa im Potsdamer Manifest
\cite{VDW2005} als „mechanisch-materialistisch“ und „geistig-lebendig“
unterschieden werden.  Damit kommen wir aber sofort auf grundlegende Fragen,
welche Technik- und Wissenschaftsverständnisse überhaupt nur Grundlage für
„Nachhaltigkeit“ sein können und welchen Anteil das Wert-Nutzen-Denken des
homo oeconomicus oder auch nur des homo faber an der aktuellen Krise unserer
fossil basierten Produktionsweise hat.

Carlowitz hat vor 250 Jahren wenigstens noch über eine nachhaltige
Bewirtschaftung der nachwachsenden Ressource „Holz“ raisonniert\footnote{Dass
  Carlowitz' Probleme eng mit der aufkommenden kapitalistischen
  Produktionsweise zusammenhängen und vergleichbare Probleme der
  Bewirtschaftung von Infrastrukturen vorher mit den lokalen Allmendegesetzen
  stabil prozessiert werden konnten, hat Elinor Ostrom klar gezeigt, siehe
  etwa \cite{Stollorz2011}. }. Unsere gesamte Technik und Wissenschaft hat
sich seither rasant weiterentwickelt, allerdings auf der Basis \emph{fossiler}
Rohstoffe, die sich definitiv \emph{nicht} in so kurzen Zeiten regenerieren
wie sie verbraucht werden.  Die damit verbundenen grundlegenden Probleme habe
ich bereits in der 2. Vorlesung („Peak Oil? Peak Everything!“) angeschaut.
Siehe dazu auch \cite{Davis2008}, \cite{Graebe2012}.

\subsection{Informationsbegriff}

„Komplexe Systeme sind lernfähig“ (Laforet). Lernfähigkeit setzt nach meinem
Verständnis 1) Reflexionsfähigkeit und 2) Selbstreflexionsfähigkeit voraus.
Ich denke nicht, dass der Begriff „komplexes System“ derart eingeengt werden
sollte.  Insgesamt sind wir bei diesem Ansatz bei Informationstheorien auf dem
Stand der 1970er Jahre, etwa \cite{Steinbuch1969}\footnote{„Geschichte ist die
  uns überlieferte Information über frühere Versuche, die Zukunft zu
  gestalten.“ (ebenda, S. 5)}, die Klaus Fuchs-Kittowski \cite{KFK2002} in der
Unterscheidung zwischen Kybernetik 1. und 2. Ordnung noch einmal resümierte.
Dieser Ansatz wurde bereits Ende der 1990er Jahre in Debatten zwischen Janich,
Capurro, Fleissner, Hofkirchner u.a. fundamental kritisiert. Dazu etwa
\cite{Janich2006}, \cite{Capurro1996}, \cite{Capurro1998}, \cite{Capurro2002},
\cite{Klemm2003}.

\section{Zum Verhältnis von Systembegriff und Wirklichkeit (Kleemann)}

Grundlegendes Problem menschlicher Verhandlungs- und Entscheidungsstrukturen
ist die Frage, wie wir die Vielfalt der Sichten auf die Wirklichkeit („Welten“
in Termini der Vorlesung) mit der Einheit der Wirklichkeit abgleichen
können. „Das, was wirklich geschieht“ ist uns nicht direkt sprachlich
zugänglich, sondern nur über kooperative Sprachformen, die Erwartetes mit
Gewesenem abgleichen können. Siehe dazu auch die
Anmerkungen\footnote{\url{http://www.dorfwiki.org/wiki.cgi?HansGertGraebe/SeminarWissen/2019-10-24}}
zu einem anderen Seminar im Dorfwiki. In diesem Sinne gilt „Welt als
Wirklichkeit \emph{für uns} ist Wirklichkeit im Prozess begriff\-licher
Erfassung“ (Vorlesung). Und in diesem Sinne sind die Begriffe \emph{Welt} und
\emph{Wirklichkeit} im weiteren Text zu verstehen.

Mit dem Systembegriff wird versucht, diese Nahtstelle von Sichten und
Wirklichkeit selbst in eine Sprachform zu bringen und damit den Begriff
\emph{Ganzheitlichkeit} zu entwickeln. Kleemann hat in seinen Ausführungen
diese Versuche über die letzten 300 Jahre, also im Wesentlichen im Kontext
bürgerlicher Gesellschaftsverhältnisse, nachgezeichnet. Nach einer
motivierenden Vorbereitung, auf die ich aus systematischen Gründen erst später
eingehe, hat er fünf Entwicklungsschritte identifiziert, in denen sich der
Ganzheitlichkeitsbegriff im betrachteten Zeitraum entwickelt hat.

\paragraph{1.}
Am Übergang des 17. zum 18. Jahrhundert herrschte folgende Sichtweise (etwa
Leibniz) vor: Die Ganzheit der Welt (hier als \emph{Wirklichkeit}, diese
Unterscheidung wurde damals so noch nicht getroffen) kann überhaupt nur aus
einer Innenperspektive erfasst werden, denn die Welt bewegt sich aus sich
heraus und in sich selbst; es gibt kein \emph{Außen}. Damit wird der Beweis
göttlicher Schöpfung als Anspruch verworfen, da dieser Gedanke einer weiteren
Entwicklung der technischen Möglichkeiten der Menschheit im Wege stand.
Dieser Ansatz findet sich übrigens bereits in der Archimedes zugeschriebenen
Position, dass er nur einen festen Punkt außerhalb der Welt bräuchte, um jene
aus den Angeln zu heben.

\paragraph{2.}
Ab Ende des 18. Jahrhunderts: Die Ganzheit der Welt wird als Einheit der
Wirklichkeit postuliert, die aber nur durch Beschreibungsformen
praktisch-planerisch zugänglich ist. Deshalb geht es um die Ganzheit und
Geschlossenheit der Beschreibungsformen. Um den nicht hintergehbaren
Widerspruch zwischen Beschreibungsformen und Wirklichkeit in die
Beschreibungsformen aufzunehmen, werden die Begriffe \emph{System} (mit einem
Geschlossenheitsanspruch) und \emph{Organismus} (als prinzipiell
unvollständige Beschreibungsform von Teilen der Wirklichkeit) geschieden.
Einem (konstruktiven) Technikbegriff ist jene systemische Welt zugänglich,
aber noch nicht die organismische. Es werden jedoch Erfahrungen aus jener
mechanistisch-technischen Welt auf die Beschreibungsformen jener „Organismen“
übertragen mit entsprechenden Folgen auch für ein Menschenbild, siehe etwa
(als frühes Werk der Periode) \emph{Der Mensch als Maschine} von Julien Offray
de la Mettrie.

\paragraph{3.}
Ab Ende des 19. Jahrhunderts können etwa energetische Experimente mit
\emph{Organismen} durchgeführt, damit die Beschreibungsformen von Organismen
selbst einer rational-kritischen Würdigung unterzogen und mit Mitteln der
mechanistisch-technischen \emph{Welt der Systeme} analysiert werden. Neben der
spekulativ-induktiven Methode der Verallgemeinerung von Beobachtungen
entwickelt sich eine symbolisch-deduktive Methode, in der Logik und Mathematik
als Komplettierungsinstrumente für Theoriegebäude mit Ganzheitsanspruch in
Stellung gebracht werden. Diese \emph{geschlossenen Theorien} begründen

\begin{enumerate}
\item[(a)] eine neue argumentative Tradition des Verhältnisses von Induktion
  und Deduktion (einen Begriff von „Science“ in der im angelsächsischen
  Sprachraum verbreiteten Bedeutung) und
\item[(b)] eine Aufspaltung in Einzelwissenschaften, deren Vertreter den
  intern stehenden (deduktiven) Ganzheitsanspruch an das jeweilige
  Theoriegebäude gern mit dem alten (induktiven) Ganzheitsanspruch einer
  „Welterklärung“ verwechseln (Naturphilosophie, Empiriokritizismus).
\end{enumerate}

\paragraph{4.}
In der ersten Hälfte des 20. Jahrhunderts führen diese zwei Linien zum Schisma
in \emph{Science} und \emph{Humanities}. Kleemann verfolgte vor allem die
Linie \emph{Science}, in der sich die interessanteren Entwicklungen bzgl. der
Sprachformen vollzogen, in denen sich der Widerspruch zwischen dem
Geschlossenheitsanspruch von Theorie und der Ganzheit und Einheit der
Wirklichkeit entwickelt.

Hier ist zunächst der Versuch zu nennen, diesen Widerspruch durch einfache
Identifizierung der Pole zu lösen: Die Versuche (Russell, Hilbert, Bernays
u.a.) zu zeigen, dass mit dem Geschlossenheitsanspruch der Theoriebildung die
Ganzheitlichkeit der Wirklichkeit prinzipiell sprachlich eingefangen werden
kann. Dieser Versuch scheitert aber mit Kurt Gödel bereits an der ersten
ernsthaften Frage: Ist das zu entwickelnde theoretische Instrumentarium
geeignet, den gestellten Geschlossenheitsanspruch in der Anwendung auf sich
selbst einzulösen?  Die verblüffende Antwort lautet nicht nur „Nein“, sondern
die Antwort kann mit den Mitteln jenes Theorieansatzes sogar \emph{bewiesen}
werden, ist also nicht spekulativen, sondern deduktiven Typs.

Auf der anderen Seite formen sich Bereiche (Einzelwissenschaften), in denen
sich je spezifische Balancen zwischen spekulativen und deduktiven Ansätzen
herausbilden, der Geschlossenheitsanspruch also eine je innerdisziplinäre
sozialisierungsbasierte Institutionalisierung (Fachlogik) erfährt. Diese
Fachlogiken, die T.S. Kuhn als \emph{Paradigmen} bezeichnet, stehen ihrerseits
aber in dialektischen Entwicklungsprozessen bis hin zu Brüchen (Kuhn
untersucht derartige Paradigmenwechsel intensiv).

Als dritte Entwicklungslinie jener Zeit verwies Kleemann auf die beginnende
\emph{Technisierung} von Science im Sinne einer technischen
Werkzeugunterstützung von Versuchsaufbaustrukturen bis hin zu repetitiven
Vorgängen innerhalb jener Sprachformen einer mathe\-matisch-deduktiven
Argumentation. Turing greift ältere derartige Ansätze (Rechenmaschinen von
Leibniz und Pascal, die „Analytical Engine“ von Babbage und Lady Lovelace) im
zunächst theoretischen Konzept der Turingmaschine (1936) auf, das in der
\emph{Turing-Bombe}, der Entschlüsselung der Chiffrierung der deutschen Enigma
bereits im Vorcomputerzeitalter einen ersten Höhepunkt praktischer Anwendung
hat.

Die Turingmaschine ist zugleich eine „Gödelmaschine“, denn sie setzt den
Gödelschen Unvollständigkeitsansatz maschinell um: Das unendliche Eingabeband
führt zu einer unendlichen Abfolge innerer Zustände, repetitive
Zustandsstrukturen sind -- im Gegensatz zur engeren Klasse der Kellerautomaten
-- an repetitive Inputs gebunden.

\paragraph{5.}
Seit den 1960er Jahren setzt sich jene instrumentelle Untersetzung
mathematisch-deduk\-tiver Ansätze in Kybernetik, Regelungssystemen, KI,
Automatisierungstechniken usw. fort. Die instrumentelle Untersetzung von
Automatisierungstechniken ist allerdings deutlich älteren Datums --
mechanische Regelwerke existieren seit Tausenden von Jahren, der Einsatz
komplizierter, mechanisch fundierter Prozess-Steuerungen begleitet die
Automatisierungstechnik seit den Anfängen der industriellen Produktionsweise
Mitte des 19. Jahrhunderts, Zuses Z1 (1937) verwendete noch komplett eine
derartige Technologie und selbst Zuses Z4 (1945) war nach dem Übergang zu
einer Technologie mit elektro-mechanischen Relais nur halbherzig
„elektrifiziert“.

Dem gesamten ingenieurtechnischen Konzept eines \emph{Stands der Technik}
liegt ein solcher mathe\-matisch-deduktiver Geschlossenheitsanspruch zu
Grunde, der mit Blick auf die prinzipielle Orientierung von Technik am Lösen
von Problemen aber interdisziplinär ist und damit die mathematisch-deduktiv
begründeten Geschlossenheitsansprüche jeder Einzelwissenschaft nur zu einem
sozio-praktisch begründbaren Geschlossenheitsanspruch zusammenführen kann. Mit
der fortschreitenden Durchdringung unserer Produktionsweise mit
wissenschaftssprachlich fundierten Praxen und fortschreitender
Technologisierung reproduziert sich damit der dialektische Widerspruch
zwischen der Vielfalt der (nun instrumentell hochgradig aufgeladenen) Welten
(der Einzelwissenschaften) und der Einheit der Wirklichkeit auf neuem Niveau.

Die Frage ist keineswegs nur von akademischem Interesse, denn in einer hoch
technisierten Gesellschaft, in der zweckgerichtetes instrumentelles Handeln
die Grundform praktischen Tuns ist, steht immer die Frage, was ist Subjekt und
was Objekt oder -- in Termine von TRIZ -- was ist Werzkeug und was
Produkt. Gestaltung betrifft Menschen, womit jene (in \emph{dieser}
Handlungslogik) immer auch Objekt und Produkt von Handeln sind. Auch TRIZ
perpetuiert diesen Ansatz, der in der \emph{Handlungsplanung} einen äußeren
Standpunkt einnimmt, um dann im \emph{Handlungsvollzug} „die Welt aus den
Angeln zu heben“ wie einst Archimedes. Wir haben in der Grundfrage über die
fünf Stufen hinweg eigentlich nichts gewonnen und bewegen uns \emph{in dieser
  Frage} weiter auf der Ebene spekulativer Gesellschaftstheorien, wie Kleemann
am Beispiel von Talcott Parsons' AGIL-Ansatz genauer ausgeführt hat.

Kleemann kondensierte seine Ausführungen in fünf Problemen, denen sich jede
Systemtheorie stellen muss, wenn sie nicht in den Verdacht geraten will, nur
soziopolitische Legitimation spezifischer Interessenkonstellationen zu sein:

\begin{itemize}[noitemsep]
\item Problem 1: Was ist innen und außen?
\item Problem 2: Der Systemaufbau. Was ist Input und Output?
\item Problem 3: Entwicklung eines tragfähigen Begriffs von Information als
  systemische Grundlage
\item Problem 4: Was ist dann Nachhaltigkeit?
\item Problem 5: Die politische Dimension
\end{itemize}

Das Ringen um einen tragfähigen Systembegriff ordnet sich damit ein in das
Ringen um die Herstellung von Sprachfähigkeit in den Gestaltungs- und
Entscheidungsprozessen der bürgerlichen Gesellschaft, auf die Kleemann im
ersten Teil seiner Ausführungen (allerdings nicht unter einer derart
expliziten Überschrift) einging. Auf dem Weg der Stärkung der
symbolisch-deduktiven Basis dieser Sprachfähigkeit als Grundlage der
wissenschaftlich-tech\-nisch verfassten modernen Produktionsweise geht es auf
einer \emph{ersten Ebene} um die Begründung der Bedeutungen von Begriffen. Auf
einer \emph{zweiten Ebene} geht es um die Bündelung von Begriffen zu praktisch
bedeutsamen Systemen (Ontologien) als sprachliche Komponente bewährter und
institutionalisierter Praxen und Verfahrensweisen. Das Verhältnis zwischen
beiden Ebenen ist ein synergetisches (wie im Konzertbeispiel in der ersten
Vorlesung besprochen): Die verfügbaren Begriffe begrenzen und ermöglichen
Praxen auf Systemebene, umgekehrt entwickeln sich Begriffe im Kontext
systemischer Praxen weiter. In der Vorlesung wird dies auf folgenden Punkt
gebracht: Bedeutung \textbf{ist} der Gebrauch von Begriffen.  Begriff und
System stehen damit in einem reflexiven Verhältnis, welches unsere
Handlungsmacht definiert.

Diese Handlungsmacht wollen wir für eine Transformation der Gesellschaft in
Richtung einer nachhaltigen Produktionsweise einsetzen. Die Vielfalt der
Welten, die beim Wechselverhältnis von Ebene 1 und 2 noch kein Problem ist,
wird nun aber zum Problem, denn mit dem Nachhaltigkeitsanspruch tritt das
dialektische Wechselverhältnis von Vielfalt der (sprachlich konstituierten)
Welten und der Einheit der Wirklichkeit unmittelbar auf die Tagesordnung. Es
gibt keinen externen Standpunkt, von dem aus sich ein nachhaltiger Umgang mit
der Wirklichkeit instrumentell erzwingen ließe. In diesem Sinne müssen wir
offen sein dafür, dass Nachhaltigkeitsfragen auch zu einer wesentlichen
Änderung dessen führen, mit welchen Begriffen und Systemen wir an jenes
dialektische Verhältnis herangehen. Wir müssen nicht nur lernen, das Denken in
Kreisläufen in unser instrumentelles Vorgehen einzubauen, sondern das
instrumentelle Vorgehen selbst zu einem Vorgehen in Kreisläufen
transformieren.

In diesem Sinne stehen auch Begriffe/Systeme einerseits und Nachhaltigkeit
andererseits in einem synergetischen Verhältnis. Billiger ist das Semantic Web
nicht zu haben.

\section{Systembegriffe in der Praxis (Gräbe)}

Literatur: \url{https://wumm-project.github.io/2019-08-07}

In den bisherigen Seminaren wurde eine Vielfalt von Kontexten betrachtet, in
denen ein Systembegriff verwendet wird.

In der \emph{Theorie Dynamischer Systeme} ging Gräbe vor allem auf
mathematische Beschreibungsformen von Modellen ein, die sich wesentlich auf
zwei Zeitebenen -- einer Mikro- und einer Makroevolution -- entfalten.

In den von Laforet und Lautenschläger vorgestellten Theorieansätzen ging es um
die Verbindung zwischen komplexen Beschreibungsformen und Handlungsvollzügen,
an denen eine Vielzahl von Akteuren beteiligt ist.  Bereits dabei wurde
deutlich, dass es schwierig ist, einen Systembegriff allein aus den
Beschreibungsformen zu entwickeln.

Im Beitrag von Kleemann wurde dieser Gedanke noch einmal vertieft und in den
historischen Entwicklungskontext eines \emph{Ganzheitlichkeitsanspruchs}
gestellt, in dem die Differenz zwischen den Erwartungen (kodiert in den
Beschreibungsformen) und den Erfahrungen (aus den Handlungsvollzügen)
praktisch prozessiert werden kann. In den Ausführungen wurde deutlich, in
welch engem Zusammenhang die jeweils historisch-konkreten \emph{Formen} jenes
Prozessierens mit dem konkreten \emph{Stand der Technik} stehen. Mit dem
aktuellen \emph{digitalen Wandel} eröffnen sich gerade auch hier vollkommen
neue Möglichkeiten, die in einer Debatte um das \textbf{Semantic Web} nicht
nur in ihrer technischen Dimension auszuloten sind.

Im Gegensatz zu anderen Seminaren war diesmal als Grundlage keine akademische
Arbeit ausgewählt worden, sondern der Zusammenschnitt einer Diskussion unter
TRIZ-erfahrenen Personen über das Verhältnis theoretischer und praktischer
Dimensionen eines Systembegriffs, der in der TRIZ-Methodologie eine wichtige
Rolle spielt. Der Systembegriff taucht dort etwa im \emph{Systemoperator} oder
bei den \emph{Gesetzen der Evolution technischer Systeme} auf, ist aber
andererseits nur unscharf gegen Begriffe wie Funktion, Komponente, Element,
Produkt oder Objekt abgegrenzt. Genau um die Problematik einer solchen
Abgrenzung ging es in jener Diskussion und auch in der Diskussion im Seminar.

Das Ergebnis unserer Diskussion lässt sich wie folgt zusammenfassen. Der
Systembegriff dient der Abgrenzung eines Beschreibungs- und Handlungsraums, in
dem die Umsetzung eines gewissen Bündels von Zwecken planerisch-beschreibend
und auch im Handlungsvollzug konzentriert ist.  Die Abgrenzung erfolgt
einerseits nach dem \emph{Relevanzkriterium} und damit submersiv, andererseits
nach dem \emph{Beeinflussungskriterium} und damit immersiv. Letzteres wurde
besonders strittig diskutiert, aber zum Ende als \emph{Einbettung in
  vorhandene Praxen} erkannt. Diese Praxen treten einerseits als vorgefundene
institutionalisierte „äußere“ Bedingungen (Obersystem in der
TRIZ-Terminologie), andererseits als vorgefundene technische Mittel
(Komponenten in der TRIZ-Terminologie) in Erscheinung. Ob eine solche Trennung
auf dem Hintergrund des in der Vorlesung entwickelten Technikbegriffs sinnvoll
ist, sei dahingestellt.

Das Begreifen einer derartigen \emph{begriff\-liche Weiterentwicklung} (in der
Vorlesung heißt es dazu „Welt als Wirklichkeit für uns ist Wirklichkeit im
Prozess begriff\-licher Erfassung“) vorgefundener Bedingungen kann sich
prozessual am Konzertbeispiel aus der Vorlesung orientieren, besser aber noch
am Konzept einer \emph{Software aus Komponenten}, nach dem heute moderne
IT-Systeme entworfen und realisiert werden. Derartige aus Komponenten
zusammengesetzte Systeme sind generell der zentrale Ansatz modernen
ingenieur-technischen Handelns, und die Informatik musste sich lange fragen
lassen, warum ein solcher Ansatz dort keine Rolle spielte. Hier sind in den
letzten 10 Jahren klare Weiterentwicklungen zu verzeichnen.

\emph{Komponenten} gehen in solche Systeme vor allem mit der von ihnen
bereitgestellten Funktionalität ein. Gleiches gilt für den dissipativen
Durchsatz, der die Strukturbildungsprozesse innerhalb des Systems antreibt,
wenn so etwas zu berücksichtigen ist. Komponenten sind selbst wieder Systeme,
wenn es darum geht, die \emph{Bereitstellung} solcher Funktionen genauer zu
beschreiben. Systembildung ist damit auf der einen Seite submersive Reduktion
von Komplexität. Diese Perspektive berücksichtigt allerdings zunächst nur die
\emph{Aufbauorganisation} des Systems. In der \emph{Ablauforganisation} müssen
sich die Abläufe in den Komponenten mit den Abläufen im System koordinieren,
was Quelle mannigfacher Restriktionen ist.

Der wesentliche eigene Beitrag auf Systemebene ist die Organisation des
Zusammenspiels der Komponenten auf eine solche Weise, dass die verfolgten
Zwecke erreicht werden. Dies wird in der spezifischen Interpretationsleistung
des Konzertbeispiels ebenso deutlich wie in der spezifischen Leistung des aus
Komponenten zusammengesetzten IT-Systems.

Die hohe zusätzliche Leistung, die bei letzterem in der Entwicklung einer
\emph{Systemarchitektur} liegt und damit die zusätzliche Unterscheidung
zwischen \emph{Systemtemplate} (Klasse) und \emph{Systeminstanz} (Objekt)
nahelegt, bleibt weiter auszuloten, zumal sich dabei Systembildungsprozesse
ihrerseits in System-Komponenten-Dichotomien auf einer höheren
Abstraktionsebene (etwa auf der Ebene von Geschäftsprozessen) einordnen
lassen.

\section{Organisation in lebenden Systemen (Laforet)}

Literatur: \cite{Mingers1989}, \cite{Ulanowicz2009}

\subsection{Vorbemerkungen}

Strukturiertes Handeln geht von der Grundannahme aus, dass die Wirklichkeit
zwar gelegentlich chaotisch erscheint, aber selbst strukturiert ist, womit
eine grundlegende epistemische Frage darin besteht, Beschreibungsformen zu
finden, mit denen diese Strukturiertheit adäquat erfasst werden kann.
Zentraler Punkt ist dabei die Frage nach Beschreibungsformen für relative
Stabilität sowie deren Entstehungs- und Auflösungsbedingungen.

Im Seminar versuchen wir auszuleuchten, welche Stellung ein Begriff
\textbf{System} bei diesen Versuchen in verschiedenen wissenschaftlichen
Zusammenhängen spielt. In den bisherigen Seminaren hatten wir den
Systembegriff als Beschreibungsfokussierung identifiziert, mit der konkrete
Phänomene durch „Reduktion auf das Wesentliche“, also durch einen submersiven
Zugang, einer Beschreibung zugänglich sind. Die Reduktion richtete sich auf
mehrere Aspekte
\begin{itemize}[noitemsep]
\item Abgrenzung des Systems nach außen gegen eine \emph{Umwelt}, Reduktion
  dieser Beziehungen auf Input/Output-Beziehungen.
\item Abgrenzung des Systems nach innen durch Zusammenfassen von Teilbereichen
  als \emph{Komponenten}, deren Funktionieren auf eine „Verhaltenssteuerung“
  über Input/Output-Be\-ziehungen reduziert wird.
\item Reduktion der Beziehungen im System selbst auf „kausal wesentliche“
  Beziehungen.
\end{itemize}
Eine grundlegende Einsicht war, dass Systembeschreibungen ähnlich dem
Konzertbeispiel aus der Vorlesung im Sinne einer „Wirklichkeit im
Prozess begriff\-licher Erfassung“ stets auf bereits vorgefundene
Beschreibungen aufsetzen. Die mit einer Systembeschreibung verbundene
Reduktion setzt in diesem Sinne bestehende Beschreibungsformen auf drei
Ebenen voraus, die mit der Weiterentwicklung der Beschreibung des
Systems selbst (explizit oder implizit) weiterentwickelt werden sollen:
\begin{itemize}[noitemsep]
\item[1.] Eine wenigstens vage Vorstellung über die Input/Output-Leistungen
  der Umgebung.
\item[2.] Eine deutliche Vorstellung über das innere Funktionieren der
  Komponenten.
\item[3.] Eine wenigstens vage Vorstellung über Kausalitäten im System selbst,
  also eine der detaillierten Modellierung vorgängige, bereits vorgefundene
  Vorstellung von Kausalität im gegebenen Kontext.
\end{itemize}
Die Punkte 1 und 2 können ihrerseits in systemtheoretischen Ansätzen für die
Beschreibung der „Umwelt“ (hierfür ist allerdings die Abgrenzung eines
Obersystems in einer noch umfassenderen „Umwelt“ erforderlich) sowie der
Komponenten (als Untersysteme) entwickelt werden, womit die Beschreibung von
\textbf{Koevolutionsszenarien} wichtig wird, die ihrerseits für die Vertiefung
des Verständnisses von Punkt 3 relevant sind.

\subsection{Lebende Systeme}

Im Seminar ging es zunächst um die Frage, ob es sinnvolle Kriterien gibt,
„lebende“ Systeme zu charakterisieren. Im Vergleich mit dem Phänomen der
Bénard-Zellen zeigte sich, dass viele der diskutierten Abgrenzungskriterien
nicht greifen, die Diskussion von „Lebendigkeit“ eher auf den Begriff „Offenes
System“ führt, für das -- im Sinne \textbf{dissipativer Systeme} -- nicht nur
Input/Output-Funktionalitäten, sondern konkrete Input/Output-Durchsätze eng an
innere Systemdynamiken gekoppelt sind. Damit sind Fragen der inneren Struktur
aber nicht nur an die \emph{Funktionalität} von Input/Output gebunden, sondern
hängen auch mit quantitativen Durchsatzraten und spezifischen
Resonanzfrequenzen zusammen.

Das bedeutet allerdings, dass für die Beschreibung von Systemen nicht nur
deren struktureller Aufbau von Bedeutung ist, sondern auch
Prozess-Charakteristika des Betriebs selbst zu berücksichtigen sind.

Strukturierungen in der Zeit sind auch aus physikalischen Systemen wie dem
Pendel -- als \textbf{Resonanzen} -- gut bekannt. Dort hängen derartige
zeitliche Strukturierungen (Resonanzfrequenzen) eng mit räumlichen
Ausdehnungen (etwa der Pendellänge) zusammen. Zeit erscheint in solchen
Systemen aber nicht als lineare Zeit, sondern als \textbf{Eigenzeiten}
(Frequenzen), in denen sich Systemzustände (näherungsweise)
wiederholen. Derartige Strukturmomente von Systemen spielten in den bisherigen
Betrachtungen keine Rolle und werden auch im Ansatz in \cite{Ulanowicz2009}
nicht berücksichtigt, wenn allein \emph{Flüsse} $T_{ij}$ zwischen
Systemkomponenten betrachtet werden, nicht aber deren zeitliche
Strukturierung.  Eigenzeiten, deren systemische Quelle noch zu identifizieren
ist, spielen eine große Rolle in der Taktung sowohl biologischer („biologische
Uhr“) als auch technischer (Taktfrequenzen) Systeme und damit im koordinierten
Zusammenspiel solcher Systeme.

\subsection{Autopoiesis}

In \cite{Mingers1989} wird dargestellt, wie historisch mit dem Begriff
\textbf{Lebendigkeit} versucht wurde, eine Klasse von Systemen näher
einzugrenzen. Diese Beschreibungsversuche gehen von der epistemischen Annahme
aus, dass jene Systeme durch ein „geheimnisvolles“ inneres Agens angetrieben
werden, das Quelle der relativen Stabilität, der Strukturierung und der
Reproduktionsfähigkeit des Systems ist. Mit dem Begriff der
\textbf{Autopoiesis} wird dieser Gedanke der „Selbsterschaffung und
-erhaltung“ von Maturana und Varela auf einen größeren Bereich von Systemen
ausgedehnt, die vergleichbare prozessuale Charakteristika aufweisen.

Eine solche Unterscheidung ist problematisch, da auch bereits
Bénardzellen ähnliche Charakteristika (Stabilität, Struktur,
Reproduktion) aufweisen, aber klar \emph{auch} durch externe Größen
beschrieben werden können. Neben einer „autopoietischen“ Beschreibung
-- der Energiedurchsatz treibt das System an, die Konvektionszellen haben
den „inneren Antrieb“ immer größer zu werden, werden daran jedoch
durch die „äußere Konkurrenz“ anderer Bénardzellen gehindert, so dass
sich „auf magische Weise“ eine „optimale“ Zellengröße einstellt --
ist eine einfache physikalische Beschreibung möglich: Die
„Optimalität“ hängt mit den räumlichen Charakteristika des Gefäßes auf
ähnliche Weise zusammen wie die Pendelfrequenz eines Pendels mit dessen
Pendellänge. Das „innere Agens“ wird zur Beschreibung also nicht
benötigt.

Gleiches gilt für das Phänomen „Wetter“. Die Herausbildung von Hoch- und
Tiefdrucksystemen wird durch den Sonnenwind als Energie- und Partikelstrom
angetrieben, Entstehung und Drehrichtung der entsprechenden atmosphärischen
Wirbel kann durch die Corioliskraft im Gravitationsfeld der rotierenden Erde
beschrieben werden. Wir haben also in erster Näherung eine ähnliche
Beschreibung der beobachteten Strukturierungen wie im Fall der
Bénardzellen. Allerdings wird dieses System wesentlich durch einen weiteren
Materiedurchsatz bestimmt -- die Aufnahme und Abgabe von Wasser und die damit
verbundenen Energieströme als Verdunstungs- und Kondensationswärme. Noch immer
ist kein „inneres Agens“ in Sicht, die Beschreibungsformen führen aber bereits
auf Modelle mit sehr komplexem (kompliziertem?) Verhalten. Moderne
Wettermodelle sind noch deutlich umfangreicher, deren Geneseprozess als
„Fortschreiben bereits vorgefundener Modellvorstellungen“ klar zu erkennen.

Die Beobachtungen von Elsasser \cite{Elsasser1981}, zitiert in
\cite{Ulanowicz2009}
\begin{itemize}[noitemsep]
\item Es gibt keine Regeln für die Biologie, die den Kräftegesetzen der Physik
  ähneln, und
\item Ökosysteme sind voller einzigartiger Events, die nicht mit bekannten
  statistischen Tools behandelt werden können,
\end{itemize}
erscheinen mit Blick auf das Phänomen „Wetter“ zumindest zweifelhaft,
da wir es dabei 
\begin{itemize}[noitemsep]
\item mit einem komplexen Wechselspiel primär physikalischer Gesetze zu tun
  haben,
\item die (aus denselben Gründen der kombinatorischen Variantenexplosion wie
  bei Elsasser) zu lokal einzigartigen Wetterphänomenen führen, die dennoch
  mit klaren Komponentenkonzepten wie „Hoch“ und „Tief“ („Organisation“ bei
  Maturana/Varela) sowie Instanziierungen, Dynamiken und Interdependenzen von
  Ausprägungen derartiger Komponenten („Strukturen“ bei Maturana/Varela) ohne
  ein „inneres Agens“ beschrieben werden können.
\end{itemize}
Im Gegenteil, das Postulat eines solchen „inneren Agens“ weist eher auf
Defizite der Modellierung hin, dass wesentliche Zusammenhänge der durch
Input/Output-Ströme getriebenen Dynamik (noch) nicht erfasst sind.

\subsection{Autokatalyse}

Ein weiteres wichtiges Konzept in \cite{Ulanowicz2009} ist das der
Autokatalyse.  Dieses Konzept spielt auch in der Theorie Dynamischer Systeme
eine wichtige Rolle, etwa als chemische Autokatalyse in der
\emph{Belousov-Zhabotinski-Reaktion}.  Derartige autokatalytische Prozesse
spielen im Stoffwechsel „lebender“ Materie eine herausragende Rolle -- von der
noch aus der Schule bekannten Beschreibung des \emph{Energiestoffwechsels}
einer Zelle über hormonelle Austauschprozesse zwischen verschieden
spezialisierten Zellen in mehrzelligen Organismen bis hin zu autokatalytischen
Prozessen zwischen Zellgruppen (Organen), an deren Vermittlung selbst Zellen
(Blutzellen, T-Zellen, Lymphzellen) beteiligt sind. Wir erkennen an den
wenigen Beispielen bereits eine Verschränkung autokatalytischer Vorgänge auf
verschiedenen Zeitskalen, womit sich die argumentative Einführung in
Maturana/Varela am Beispiel von Prozessen \emph{innerhalb} einer Zelle in
\cite{Mingers1989} als sehr speziell erweist.

Natürlich steht die Frage, ob die genannten Prozesse auf jeder dieser Ebenen
als „autokatalytisch“ durchgehen. Zumindest im Sinne positiver
Feedback-Kreisläufe wie in \cite{Ulanowicz2009} ist dies zu bejahen.
Allerdings bleiben dort wesentliche Charakteristika autokatalytischer Prozesse
ausgeblendet, die in der Theorie Dynamischer Systeme eine wesentliche Rolle
spielen:
\begin{itemize}[noitemsep]
\item[1.] Autokatalytische Prozesse haben eine \textbf{Eigenzeit} und sind
  damit sowohl Quelle eigener Taktraten als auch Phänomenen der Resonanz und
  Dissonanz gegenüber getakteten Input/Output-Beziehungen unterworfen.
\item[2.] Positive Feedback-Kreisläufe führen -- für sich genommen -- in der
  Regel zu exponentiellem Wachstum.  Stabilität autokatalytischer Prozesse ist
  also nur dann zu erklären, wenn auch der begrenzende Faktor identifiziert
  wird, der sich meist aus den Input/Output-Beziehungen ergibt, in welche das
  autokatalytische System eingebunden ist.

\item[3.] Diese externen Dynamiken, welche das Wachstum autokatalytischer
  Systeme begrenzen, sind oft selbst Teil eines autokatalytischen Systems.
\end{itemize}
Zu Punkt 2 ist zu bemerken, dass exponentielles Wachstum insbesondere aus
linearen Differentialgleichungen resultiert, in denen der Zuwachs proportional
zur bereits vorhandenen Substanz ist. Nichtlineare Differentialgleichungen
können auch zu Systemen führen, deren Größe durch innere Parameter bestimmt
sind. 

Die im Punkt 3 thematisierte Verschränkung von Mikro- und Makroevolution wird
in der Theorie Dynamischer Systeme intensiv untersucht, insbesondere wenn die
Zeitskalen der Eigenzeiten von Mikro- und Makrosystem weit auseinander liegen.
Dann kann auf kleinen Zeit\-skalen der durch das Makrosystem induzierte
Durchsatz als Input/Output bei der Analyse der Dynamik der Mikrosysteme als
konstant betrachtet werden.  Umgekehrt kann auf der Makroskala davon
ausgegangen werden, dass sich Volatilitäten auf der Ebene der Mikrosysteme
„ausmitteln“ und damit keine Bedeutung haben, das Mikrosystem als sich auf
einer Attraktorlage bewegend angesehen werden und dessen Verhalten damit
„deterministisch“ modelliert werden kann.

\section{Resilienz (Laforet, Lautenschläger)}

Literatur: \cite{Holling2000}, \cite{Walker2004}, Zusatzliteratur:
\cite{Brand2007}

\subsection{Stadt als System}

An dieser Fragestellung ist besonders deutlich zu sehen, was „Wirklichkeit für
uns als Wirklichkeit im Prozess begriff\-licher Erfassung“ bedeutet. Eine
\textbf{konkrete Stadt} ist zunächst eine \emph{Lebenswirklichkeit}, die als
„Wirklichkeit für uns“ bereits konzeptionell (im Begriff „Stadt“) grob
abgegrenzt ist. Diese Lebenswirklichkeit wird als strukturiert wahrgenommen,
was in diesem Fall (eines kulturell überformten Systems) nicht nur eine
epistemische Annahme ist, sondern auf einer Praxis aufsetzt, die auf der Basis
bereits vorhandener strukturierter Beschreibungsformen die Wirklichkeit nach
diesem Vorbild zu strukturieren versucht.

Oben wurden für Systembeschreibungen die folgenden \textbf{drei
  Reduktionsdimensionen} als „Reduktion auf das Wesentliche“ thematisiert:
\begin{itemize}[noitemsep]
\item[1.] Abgrenzung des Systems nach außen gegen eine \emph{Umwelt},
  Reduktion dieser Beziehungen auf Input/Output-Beziehungen.
\item[2.] Abgrenzung des Systems nach innen durch Zusammenfassen von
  Teilbereichen als \emph{Komponenten}, deren Funktionieren auf eine
  „Verhaltenssteuerung“ über Input/Output-Be\-ziehungen reduziert wird.
\item[3.] Reduktion der Beziehungen im System selbst auf „kausal wesentliche“
  Beziehungen.
\end{itemize}
Eine konkrete Stadt wie Leipzig ist ein „System von Systemen“ in dem Sinne,
dass es im Heute vielfältige vorgefundene Beschreibungs- und auch
Handlungssysteme gibt, die in einem Konzept „Stadt Leipzig als System“ als
\emph{Komponenten} zu betrachten sind, deren Beschreibungen im Sinne von (2)
in die Beschreibung des „Systems Stadt Leipzig“ als über ihre
Input/Output-Beziehungen einer Verhaltenssteuerung zugänglich eingehen.
Ebenso ist die Abgrenzungsdimension (1) von außen verständlich als -- in
erster Näherung -- der nicht weiter beeinflussbare Input/Output aus landes-
und bundespolitischer Ebene, über den das „System Stadt Leipzig“ Objekt
externer Verhaltenssteuerung ist. Weniger klar ist die Reduktion der
Beziehungen (3) -- hier geht zentral die Frage ein, welche Beziehungen als
„wesentlich“ zu betrachten sind. Dies ist eng mit dem \emph{Zweck} der
Systembeschreibung zu verbinden. Dies kann (für Leipzig) zum Beispiel der
praktische Erstellungs- und Abstimmungsprozess zum Integrierten
Stadtentwicklungskonzept
INSEK2030\footnote{\url{https://www.leipzig.de/bauen-und-wohnen/stadtentwicklung/stadtentwicklungskonzept-insek/}}
als politisch vereinbarter Vision der Stadtentwicklung sein.

Liegt eine größere Zahl solcher Beschreibungen konkreter „Städte als System“
vergleichbarer Strukturierung vor, so kann zur Modellierung eines
\textbf{Templates} „Stadt als System“ übergegangen werden, um die Erfahrungen
der bisherigen konkreten Systemmodellierungen zu systematisieren und damit
zukünftige Stadtmodellierungen zu erleichtern.  Technisch gesehen ist eine
solche \emph{Templatisierung} mit einer \emph{Standardisierung} und damit
einer \emph{Institutionalisierung} der konkreten Modellierungsprozesse
verbunden. Damit sind zugleich die divergenten und die konvergenten Phasen der
Modellierungsaktivitäten in Fortschreibung bereits existierender
Beschreibungsformen thematisiert.  Eine solche vergleichende Betrachtung der
Modellierungen verschiedener Stadtsysteme wird erst möglich, wenn zu einem
\emph{Obersystem} übergegangen wird. Erst in einem solchen vergleichenden
Kontext verschiedener Praxen können die dafür erforderlichen sprachlichen
Mittel entwickelt werden.

Dies korrespondiert mit einem der TRIZ-Gesetze der Entwicklung Technischer
Systeme, der \emph{Tendenz des Übergangs zum Obersystem}.

\subsection{System von Systemen}

In den bisherigen Debatten war der Fokus auf ein genaueres Verständnis des
Begriffs \emph{System} gerichtet, der als Reduktion von Komplexität in den
drei oben angeführten Richtungen betrachtet werden kann. Da in diesem
Verständnis Komponenten eines Systems selbst wieder Systeme sind, liegt die
Betrachtung eines Systems als „System von Systemen“ nahe, wie es in
\cite{Holling2000} thematisiert wird. Wesentliches Reduktionskriterium für
Beziehungen (3) sind in solchen Systemen spezifische Eigenzeiten und
Eigenräume wie in den Abbildungen 1--3 in \cite{Holling2000} dargestellt.

Der ebenda für einen solchen epistemischen Prozess geprägte Begriff der
\textbf{Panarchie} bedarf aber einer weiteren Präzisierung durch Offenlegung
und Diskussion der epistemischen Annahmen, die Hollings Argumentation zu
Grunde liegen.

\paragraph{a)}
Die \textbf{erste epistemische Annahme} betrifft die Auswahl vorgefundener
Systeme (Systemkomponenten), die zu einem neuen System zusammengefasst
werden. Dazu heißt es in \cite[S. 392]{Holling2000} „semi-autonomous levels
are formed from the interactions among a set of variables that share similar
speed (and, we would add, geometric/spatial attributes)“. Hinzuzufügen aus
einer früheren Diskussion in unserem Seminar ist der Umstand, dass diese
Komponenten darüber hinaus in einem „autokatalytischen“, positiven
Feedbackverhältnis stehen müssen, das allein Quelle von Wachstum und damit
Entwicklung sein kann.  Derartige Verhältnisse werden in anderen Kontexten als
\textbf{synergetisch} bezeichnet. Die Beschreibung der zentralen Dynamik eines
solchen synergetischen Verhältnisses (in welchem das Ganze mehr als die Summe
seiner Teile ist) ergibt sich aus dem \emph{dynamischen Prozess} der
Wechselwirkung der Systemkomponenten, wie er in den Abbildungen 4 und 5 in
\cite{Holling2000} dargestellt ist, während die Systemkomponenten selbst in
diese Beschreibung allein durch ihre Input/Output-Charakteristika eingehen,
die innere Dynamik der Komponenten aber, welche die Input/Output-Leistungen
garantieren, außer Betracht bleibt. Die Systembeschreibung ist damit eine
\textbf{Fiktion} im Sinne des Gebrauchs dieses Begriffs in der Vorlesung.

Die Erklärung der Leistung eines Systems ergibt sich unter diesen
epistemischen Voraussetzungen vor allem aus dem \emph{Zusammenspiel} der
Komponenten. Dieses \emph{Zusammenspiel} bestimmt dann auch wesentlich
Eigenzeiten und Eigenräume des Systems.

Setzt man dieses epistemische Konzept rekursiv zusammen, dann spielen
Resonanzen und Dissonanzen der Dynamiken der Systemkomponenten eine wesentlich
zentralere Rolle als rein quantitative Durchsatzraten.  Insbesondere steht die
Frage, in welchem Umfang sich in synergetischen Bindungen die Eigenzeiten von
Komponenten ändern können, um resonante Kopplungen herzustellen. Im
Handexperiment zum Doppelpendel habe ich einen solchen Anpassungseffekt
bereits vorgeführt. Ich möchte hier ein weiteres Beispiel aus der Ökonomie
anführen, das sich am Konzept der Kapitalumschlagzeiten als Eigenzeit einer
systemischen ökonomischen Aktivität im heutigen Wirtschaftskontext
orientiert. Die Kapitalumschlagzeit eines Straßenhändlers, der morgens auf dem
Großmarkt einkauft und abends die Reste unverkäuflichere Ware entsorgt,
beträgt einen Tag. Gleichwohl ist er von Obersystemen gezwungen, monatliche
Bilanzen (für den Steuerberater) und Jahresbilanzen (für das Finanzamt)
anzufertigen, was dem Händler weitere Aktivitäten aufzwingt, deren Eigenzeiten
Vielfache der Kapitalumschlagzeit sind. Die Sache ist also einfach, wenn die
Eigenzeiten auf Systemebene ein (gemeinsames) Vielfaches von Eigenzeiten der
Komponenten sind. Ist dies nicht der Fall (etwa bei Kapitalumschlagzeiten von
mehreren Jahren bei Investitionen), muss das System seinen Komponenten
spezielle Synchronisationsleistungen zur Verfügung stellen, wie dies im
konkreten Fall von Investitionen mit Abschreibungsmodalitäten geschieht, mit
denen langfristig investiv gebundene Kapitale in die kurzfristigeren
jährlichen Umschlagzeiten integriert werden, die das Finanzamt triggert. Das
Risiko von Störungen (zu frühe faktische Außerdienststellung der Investition)
bleibt dabei in der Systemkomponente (dem Unternehmen) gekapselt, das dafür
ein angemessenes positives Feedback (den Profit) erhält, um das System
insgesamt im Bereich synergetischer Kopplung zu halten.

Der Ansatz ist insoweit epistemisch geschlossen, als eine solche Betrachtung
Eigenzeiten und Eigenräume von bereits vorgefundenen Systemen und
Systembeschreibungen voraussetzt und die Modellierungsregel postuliert, dass
nur Komponenten mit vergleichbaren (in dem oben prototypisch skizzierten
weiten Verständnis) Eigenzeiten zu einem neuen System zusammengefasst werden,
das Ausgrenzungkriterium (1) also \emph{diesen} Punkt der Modellierung
dominant beachtet.

In einem solchen Ansatz ist zugleich klar, dass die \emph{Eigenzeit} des
Systems deutlich größer ist als die seiner Systemkomponenten, da die
Systemkomponenten gerade \emph{nicht} mit ihrer eigenen internen Dynamik,
sondern nur mit den durch diese Dynamik reproduzierten
Input/Output-Charakteristika, ihrem „Verhalten“, in die Modellierung des
Systems eingehen.

Geht man weiter davon aus, dass diese epistemische Annahme der
Komplexitätsreduktion auch den meisten Formen kooperativer Selbstreflexion
wenigstens höher entwickelter Tierarten zu Grunde liegt und diese damit
versuchen, ihre kooperativen Praxen im Takt solcher Eigenzeiten aktiv zu
synchronisieren, so schlägt die epistemische Annahme in eine Bedingtheit der
Möglichkeiten kooperativen Handelns „autonomer Agenten“ um. Die Dynamik der
Wirklichkeit entfaltet sich als „sich selbst erfüllende Prophezeiung“ des
Handelns autonomer Agenten, das als Spannungsfeld zwischen begründeten
Erwartungen und erfahrenen Ergebnissen ein Zukunftsfeld vorstrukturiert und
damit Handlungsräume aktiv gestaltet.

\paragraph{b)}
Die \textbf{zweite epistemische Annahme} betrifft die Beschreibungsformen der
Systemdynamik selbst. \cite{Holling2000} geht von längeren Phasen stabiler
Entwicklung bis hin zu „konservativem Verhalten“ (Phasen $r$ und $K$) und
kürzeren Phasen kompletten Systemumbaus (Phasen $\Omega$ und $\alpha$) aus als
„normale Systementwicklung“, die dort als \textbf{adaptiver Zyklus} bezeichnet
wird.

Die genaue beschreibungstechnische Basis bleibt im Dunkeln, die Ausführungen
in \cite{Walker2004} weisen darauf hin, dass hier eher auf eine Beschreibung
der unmittelbaren Bewegung im Phasenraum orientiert wird, die aus
irgendwelchen Gründen mit \emph{Störungen} (disturbances) aufgeladen ist.
Bilder wie die Abbildungen 1a und 1b in \cite{Walker2004} werden allerdings
dem Umstand nicht gerecht, dass wir grundsätzlich dissipative Prozesse weitab
von Gleichgewichtslagen beschreiben wollen.  Diese Differenz zwischen „zwei
Bedeutungen des Resilienzbegriffs“ wird in \cite{Brand2007} klar
herausgearbeitet. In einem solchen „Kontext zweiter Art“ kann ebenfalls mit
Begriffen wie „Potenzialbassins“ gearbeitet werden, allerdings nur, wenn die
systemischen Rückstellkräfte in der Nähe von „Fließgleichgewichten“ -- also
systemischen Attraktoren -- modelliert werden.

\begin{wrapfigure}[17]{r}{.4\textwidth}
  \begin{tikzpicture}[scale=.65,line width=1pt,transform shape]
    \node (A0) at (0,10) {};
    \node (A4) at (3,8.5) {};
    \node (A1) at (5,7) {};
    \node (A1a) at (7,6.5) {};
    \node (A2) at (0,5) {};
    \node (A3) at (7,0) {};
    \node (A5) at (6.5,4) {};
    \node (A6) at (6,.5) {};
    \draw plot [smooth] coordinates {(A0) (A4) (A1) (A2) (A6) (A3)};
    \node[draw=red] at (4.2,9.2) [rectangle] {$r$-Phase};
    \draw[<->] (2.3,8.1) -- (3.1,7.7) ;
    \node[draw=red] at (6.2,7.5) [rectangle] {$K$-Phase};
    \draw[<->] (4.7,6.5) -- (5.5,6.5) ;
    \node[draw=red] at (8.2,6.5) [rectangle] {$\Omega$-Phase};
    \draw[->] (6.7,5.9) -- (6.6,5.3) ;
    \node[draw=red] at (8,3.8) [rectangle] {$\alpha$-Phase};
    \draw[->] (6.2,3.4) -- (6.1,2.8) ;
    \draw[<->] (5.2,.3) -- (6.2,-.2) ;
    \draw[fill=green] (A4) circle (6pt) ;
    \draw[->,dashed] plot [smooth] coordinates {(A1) (A1a) (A5) (A6)};
    \draw[fill=green] (A1) circle (6pt) ;
    \draw[fill=green] (A1a) circle (6pt) ;
    \draw[fill=green] (A5) circle (6pt) ;
    \draw[fill=green] (A6) circle (6pt) ;
    \node[draw=red,fill=white] at (7.5,0.9) [rectangle] {neue $r$-Phase};
  \end{tikzpicture}
\end{wrapfigure}
Eine solche Modellierung geht epistemisch davon aus, dass die realweltliche
systemische Dynamik beschreibungstechnisch in einen dominanten „verstandenen“
Basisteil und einen marginalen „unverstandenen“ Rest zerlegt werden kann. Für
den Basisteil lässt sich eine genauere Dynamik beschreiben, die das System in
der Nähe eines Attraktors hält (durch entsprechende Rückstellkräfte, die
„Störungen“ dämpfen, allerdings nur in der Nähe des Attraktors beschreibbar
rückstellend wirken). Damit kann in der Systembeschreibung erläutert werden,
warum sich im System kleine Störungen nicht aufschaukeln, sondern „absorbiert“
werden. Beim Absorbieren von Störungen bewegt sich der Referenzpunkt des
Systems auf dem Attraktor, das System „entwickelt sich“ (Phase $r$), wenn dies
die lokalen Bedingungen für den Referenzpunkt auf dem Attraktor zulassen. Ein
„konservierender Zustand“ (Phase $K$) tritt ein, wenn die Rückstellkräfte
(conservation forces) das System stets auf denselben Referenzpunkt auf dem
Attraktor zurückführen, etwa, weil auf dem Attraktor eine gewisse
\emph{Extremlage} erreicht ist. Da das System auf dem Attraktor nur noch
eingeschränkte Entwicklungsmöglichkeiten in der lokalen Umgebung des
Referenzpunkts hat, können sich Störungen so weit aufschaukeln, dass der
Bereich der Wirkung der Rückstellkräfte verlassen wird und das System in einen
Zustand höherer Dynamik (Phase $\Omega$) übergeht. Im Rahmen eines
\emph{adaptiven Zyklus} wird allerdings angenommen, dass das System relativ
rasch zu einem anderen Ort auf dem Attraktor findet und so -- ggf. auf Kosten
innerer Umbauprozesse in den Systembeziehungen oder in den Systemkomponenten
-- sich selbst und damit seine nach außen ins Obersystem exportierte
\emph{Funktion} stabilisiert. Die „Katastrophe“ bleibt lokal begrenzt.

Ähnliche 4-Phasen-Dynamiken (Frühling, Sommer, Herbst, Winter) sind aus der
Theorie der \emph{K-Wellen} (Kondratjew-Zyklus) bekannt und spielen auch in
Schumpeters Ansatz einer „creative destruction“ sowie bei den von T.S. Kuhn
untersuchten Paradigmenwechseln in der Wissenschaft eine Rolle. Das Verhalten
kann auch kurz als „evolutionäre Entwicklung des Systems durch revolutionären
Umbau seiner Beziehungen und Komponenten“ charakterisiert werden und ist ein
wesentlicher Mechanismus, wie die systemisch lokale Begrenztheit von
Umbauprozessen beschrieben werden kann.

\paragraph{c)}
Die \textbf{dritte epistemische Annahme} betrifft das Zusammenspiel der
Dynamiken von System und Systemkomponenten (bzw.  Obersystem und System), also
die Beschreibungsform der Verschränkung von Mikro- und Makroevolution auf den
jeweiligen kurzwelligen (Systemkomponenten) und langwelligen (System) Skalen.

In diesem Punkt können die Modellvorstellungen in \cite{Holling2000} nicht
überzeugen, da ein weitgehendes Nebeneinander der Dynamiken (Abbildungen 1-3,
6) bzw. nur ein loser Einfluss (Abbildung 7) postuliert wird. Der Ansatz der
Theorie Dynamischer Systeme geht von einer deutlich intensiveren Verschränkung
sowohl auf der Ebene der stoff\-lichen Flüsse als auch der Takt\-raten
aus. Beide (sowohl die quantitativen Charakteristika des Materiedurchsatzes
als auch dessen Taktung) sind wesentliche Charakteristika der
\textbf{Schnittstelle} zwischen dem System und dessen Komponenten -- in die
Modellierung des Systems geht sie in die Beschreibung der Systemdynamik ein,
in die Modellierung der Systemkomponenten als die
Input/Output-Charakteristika, welche die innere Dynamik der Komponente
antreiben. Wir haben damit eine \textbf{vierte Reduktionsdimension}
identifiziert, die für eine beschreibungstechnische Entkopplung der Dynamiken
von System und Komponenten von Bedeutung ist. Erst \emph{nach} einer solchen
Reduktion lassen sich Ebenen von Systemen wie bei \cite{Holling2000} sinnvoll
voneinander scheiden, in denen das System auf der nächst höheren Ebene
angesiedelt ist im Vergleich zu seinen Komponenten.

Holling verweist auf Simons Argumentation, „that each of the levels of a
dynamic hierarchy serves two functions. One is to conserve and stabilize
conditions for the faster and smaller levels; the other is to generate and
test innovations by experiments occuring within a level“
\cite[S. 393]{Holling2000}, und betont, dass die zweite Funktion die
entscheidende für das Funktionieren „adaptiver Zyklen“ sei. Die eher
anthropomorphe Beschreibung („Experimentieren“) dieser zweiten Funktion habe
ich oben auf ein klares Verständnis in der Sprache der Beschreibungen des
Systemverhaltens relativ zu seinem Attraktor zurückgeführt. Mit der ersten
Funktion wird behauptet, dass jedes System die Kapazität hat, stabilisierend
auf die Dynamiken der kurzwelligeren Systemkomponenten zu wirken -- dieser
Effekt wird in der TDS auch als \emph{Versklavungseffekt} bezeichnet. Dies
wird allerdings durch die „panarchial connections“ (Abbildung 7)
konterkatiert, in der nicht nur eine Wirkung „remember“ im Sinne dieser
„ersten Funktion“ postuliert wird, sondern auch eine weitere Funktion
„revolt“, mit welcher die Systemkomponenten Einfluss auf die Adaptivität des
Systems nehmen.

Die epistemische Kausalität der Beschreibungslogik des Ansatzes „System von
Systemen“ geht von längerwelligen großräumigeren Dynamiken zu kurzwelligeren
kleinräumigeren Dynamiken und ist deshalb wenig sensitiv für den zuletzt
beschriebenen Stabilisierungseffekt in der umgekehrten Richtung. Im Kontext
meines Handexperiments zum Doppelpendel hatte ich aber darauf hingewiesen,
dass etwa die Bemühungen eines Kindes, eine Schaukel durch rhythmische
Schwerpunktverlagerungen in Schwingung zu versetzen, genau einen solchen
Einfluss demonstriert. Auch das Konzept „Abschreibungen“ weist in die
umgekehrte Richtung einer Integration langwelliger Phänomene in Systeme mit
kürzeren Eigenzeiten. Wir haben überhaupt \emph{nur} über derartige
Mechanismen heute die Möglichkeit, die großräumigen Probleme, die unsere
Wirtschaftsweise planetar verursacht, in den Griff zu bekommen.

\subsection{Resilienz und Panarchie}

\cite{Brand2007} betont, dass der Begriff \textbf{Resilienz} in den Debatten
vielfach überladen ist. Im Kontext der Betrachtung dissipativer Systeme („the
second kind of resilience to which we refer in this text“ \cite{Brand2007})
werden allein 10 Ansätze aus der Literatur aufgelistet, mit denen Resilienz
„with respect to the degree of normativity“ gefasst wird. Resilienz ist damit
ein sehr problematischer Begriff, insoweit vorn bereits „Normativität“
hineingesteckt wird, die hinten als Option politischen Handelns wieder
herauskommen soll. Ein solcher Beschreibungsansatz muss nicht verkehrt sein,
wenn er mit dem Henne-Ei-Problem, das er in sich trägt, sauber -- also
dialektisch im Sinne der Weiterentwicklung vorgefundener Verhältnisse --
umgeht.

Resilienz ist dabei im Sinne von „Widerständigkeit“ zu verstehen, die
einerseits die Fähigkeit zu relativer Stabilität, aber andererseits die
Fähigkeit auch zu radikalem Umbau umfasst, ohne die Qualität der
Input/Output-Beziehungen in der eigenen Umwelt sowohl quantitativ als auch
qualitativ in Frage zu stellen. So heißt es in \cite{Walker2004} „resilience
is the capacity of a system to absorb disturbance and reorganize while
undergoing change so as to still retain essentially the same function,
structure, identity, and feedbacks.“ Resilienz ist in diesem Sinne also die
(zunächst qualitative) \emph{Fähigkeit} eines Systems, seine
Input/Output-Charakteristika als funktionale Rolle in Obersystemen auch unter
Bedingungen eines möglichen systeminternen Umbaus beizubehalten. Eine halbe
Seite vorher wurde noch eine „major distinction between resilience and
adaptability, on the one hand, and transformability on the other“
thematisiert, so dass möglicherweise auch noch zwischen einer
„transformability“ im Kontext von Resilienz und einer solchen jenseits dieses
Kontexts zu unterscheiden ist.  Letztere wäre dann Teil eines umfassenderen
Umbauprozesses auch in höheren „levels“, wie sie Holling etwa mit seinem
„Sternchenbeispiel“ (Abbildung 9) thematisiert.

Zunächst steht allerdings die Frage, ob Resilienz eine allein qualitativ
fassbare \emph{Fähigkeit} oder doch eine, dann auch quantifizierbare
„Kapazität“ ist. Für Holling sind es zunächst die drei zu quantisierenden
Eigenschaften adaptiver Zyklen (also von Systemen in unserem Verständnis)
\begin{itemize}[noitemsep]
\item inhärentes Änderungspotenzial (wealth),
\item interne Kontrollfähigkeit (controllability) und
\item adaptive Kapazität oder Resilienz (adaptive capacity),
\end{itemize}
mit denen ein Maß für eine \emph{potenzielle} Widerständigkeit (Resilienz?)
definiert werden soll, welches einem einzelnen System als globale Zahl (?)
zugeordnet werden kann.

Wir hatten bereits oben gesehen, dass derart generalisierende „emergente“
Parameter aus Sicht der TDS wenig aussagefähig sind, wenn nicht das Potenzial
der lokalen Dynamik um den Referenzpunkt des aktuellen Systemzustands auf dem
Attraktor mit in die Betrachtung einbezogen wird. Das geschieht in
\cite{Holling2000} mit einem 4-Phasen-Modell, womit „Widerständigkeit“ aber
schon abhängig vom aktuellen Systemzustand wird. Eine Quantifizierung von
Widerständig\-keit in einem solchen Verständnis ist insbesondere nach einer
Reorganisation des Systems in einer $\alpha$-Phase vollkommen neu zu
bestimmen, da sich der Referenzpunkt des Systems auf dem Attraktor dann weit
entfernt vom ursprünglichen Referenzpunkt befindet.

Fasst man Resilienz als substantiviertes Adjektiv, so steht mit aller Schärfe
die einfache Frage, \emph{wovon} es denn eine Eigenschaft sei.  Um sich dieser
Frage zu nähern, führt Holling den Begriff der \textbf{Panarchie} als
„representation of a hierarchy as a nested set of adaptive cycles“ ein, für
die er feststellt, dass „the functioning of those cycles and the communication
between them determines the sustainability of a system.“ Wir sind also mit
einem noch einmal anderen Systembegriff konfrontiert, der diesmal bereits die
gesamte Panarchie umfasst sowie der Frage, in welchem Verhältnis die hier
möglicherweise synonym gebrauchten Begriffe \emph{resilience} und
\emph{sustainability} stehen. Zur Klarheit der Begriffe bezeichne ich im
Weiteren diese Abstraktionsebene der Betrachtungen, die etwa in den
Abbildungen 7 und 9 in \cite{Holling2000} eine Rolle spielt, als
\emph{Panarchie} -- in \cite{Brand2007} auch als \emph{Metasystem} bezeichnet
-- und reserviere die Bezeichnung \emph{System} für die bisher eingenommene
Abstraktionsebene, in der „adaptiver Zyklus“ für die Beschreibung der Dynamik
dieses Systems auf einer hohen Abstraktionsebene steht.

Wie bereits oben ausgeführt ist eine solche \emph{Panarchie} als
Beschreibungsform ein Zusammenspiel von Systembeschreibungen auf verschiedenen
„Ebenen“ (levels), die im Sinne der oben beschriebenen \emph{vierten
  Reduktionsdimension} durch entsprechende API-Strukturen „kommunikativ“
verbunden sind, deren epistemische Charakteristik sich in der Dichotomie von
Spezifikation und Implementierung klassischer API-Strukturen der Informatik
wiederfindet. Damit ist aber eine Panarchie keineswegs ein „nested set of
adaptive cycles“, sondern -- wie bereits oben genauer erläutert -- eine
Beschreibungsstruktur ineinander greifender adaptiver Zyklen, die nach
demselben epistemischen Reduktionsschema rekursiv ausgeführt sind und auch zu
einer realweltlichen Strukturierung führen, wenn kooperative Subjekte diese
Beschreibungsformen als Basis ihrer Handlungsvollzüge nehmen.

Entsprechend werden in \cite{Walker2004} vier „crucial aspects of resilience“
genannt:
\begin{itemize}[noitemsep]
\item \emph{Latitude:} Wie weit sind kritische Punkte (threshold) voneinander
  entfernt?
\item \emph{Resistance:} Wie „widerständig“ ist ein System gegen Veränderung,
  wie „schnell“ bewegt es sich im Phasenraum?
\item \emph{Precariousness:} Wie nahe ist der Systemzustand am nächsten
  kritischen Punkt?
\item \emph{Panarchy:} Welche „cross-scale interactions“ haben Einfluss auf
  das System?
\end{itemize}
Diese Begriffe lassen sich in der TDS deutlicher in der Sprache systemischer
Eigenzeiten und Eigenräume fassen.

Mit den beiden Pfeilen „revolt“ und „remember“ in Abbildung 7 von
\cite{Holling2000} wird suggeriert und auf S. 402 explizit postuliert, dass
der Begriff Resilienz überhaupt nur unter Betrachtung von drei konsekutiven
„Ebenen“ einer Panarchie gefasst werden kann (was im Übrigen sehr genau mit
dem Systemoperator von TRIZ zusammenpasst). Das ist allerdings unzutreffend,
da „revolt“ und „remember“ auch als nichtfunktionale Eigenschaften der
Schnittstellen zwischen den Ebenen als Teil einer sinnvoll ausgeführten
Reduktion entlang der oben thematisierten vierten Reduktionsdimension und
damit als Teil der Systembeschreibung selbst gefasst werden können (zu der ja
die Spezifikation der Schnittstellen des Gesamtsystems sowie der
Schnittstellen der Systemkomponenten gehört).

Die weiteren Fragen in den drei Aufsätzen, in welchem Umfang Resilienz durch
konkrete menschliche Akteure praktisch steuernd beeinflusst werden kann, lässt
sich erst auf der Basis eines ausreichend entfalteten Technikbegriffs sinnvoll
diskutieren. In \cite{Holling2000} werden drei zusätzliche „features“
Vorausschau, Kommunikation und Technologie aufgeführt, mit denen „human
systems“ (was das auch immer ist, ich bevorzuge den Begriff „kooperative
Subjekte“, denn nur solche sind handlungsfähig) ihren Einfluss auf die Frage
der Resilienz von Systemen erhöhen können, insbesondere auf „Charakter und
Lage der Variabilität innerhalb der Panarchie“, um damit „das Potenzial der
Panarchie selbst drastisch zu erhöhen“ (S. 401). Damit wandert der
Resilienz-Begriff (falls es sich noch immer um diesen handelt) von der Ebene
der Systeme zum „Metasystem Panarchie“. Die drei neuen „features“ liegen
allerdings irgendwie quer zu den drei Dimensionen „gesellschaftliches
Verfahrenswissen“, „institutionalisierte Verfahrensweisen“ und „privates
Verfahrenskönnen“ des Technikbegriffs der Vorlesung. Auf jeden Fall sind sie
Teil eines weiteren einer Beschreibung zugänglichen Systems -- des Systems der
Reproduktion des Wissens der Menschheit. Diese Fragen werden in drei weiteren
Aufsätzen \cite{Stollorz2011}, \cite{Helfrich2011}, \cite{Dobusch2011}
aufgegriffen.

\section{Organisation in komplexen adaptiven Systemen (IIRM)}

Literatur: \cite{Ashby1958}, \cite{Boisot2011}; Zusatzliteratur:
\cite{Holland2006}, \cite{Jacobasch2019}

In der VDI-Richtlinie 3780 wird der Technikbegriff unter folgenden drei
Dimensionen gefasst: 
\begin{itemize}[noitemsep]
\item die Menge der nutzenorientierten, künstlichen, gegenständlichen Gebilde
  (Artefakte oder Sachsysteme),
\item die Menge menschlicher Handlungen und Einrichtungen, in denen
  Sachsysteme entstehen und
\item die Menge menschlicher Handlungen, in denen Sachsysteme verwendet
  werden.
\end{itemize}

Im Begriff \emph{technisches System} (ist dies identisch mit dem hier
verwendeten Begriff des „Sachsystems“?) kommt eine weitere Dimension des
Systembegriffs zum Tragen -- seine praktische Bedeutsamkeit für strukturiertes
Handeln in der bürgerlichen Gesellschaft. Mit dem Ansatz \emph{agentenbasierte
  Systeme} wird diese Dimension in \cite{Holland2006} angerissen, auch wenn
dort unter dem Begriff \emph{classifier system} eine extrem signaltheoretisch
geprägte Theorie entwickelt wird, in der derartigen „Agenten“ nur eine sehr
eingeschränkte Fähigkeit zur Handlungsstrukturierung zugebilligt wird.

Der in der Vorlesung entwickelte Begriff der \emph{Fiktion} als
„gesellschaftlich gestützter, garantierter und aufrecht erhaltener Konsens
einer verkürzenden Sprechweise über eine gesellschaftliche Normalität“ rückt
die Herstellung und Aufrechterhaltung dieser \emph{Normalität} und damit die
gesellschaftlichen Bedingungen des \emph{Funktionierens} technischer Systeme
in den Fokus. Technische Systeme sind damit nicht nur zu \emph{beschreiben},
sondern auch zu \emph{betreiben}.

Der Betrieb eines solchen Systems spielt sich nicht (nur) auf der
Beschreibungsebene ab, sondern greift unmittelbar in Prozesse der realen
Wirklichkeit ein. Er ist damit eingespannt in das allgemeine praktische
Verhältnis zwischen den aus den Beschreibungsformen abgeleiteten
\emph{begründeten Erwartungen} und den in der praktischen Umsetzung
\emph{erfahrenen Ergebnissen}.

Der Ansatz agentenbasierter Systeme geht davon aus, dass die
\emph{Strukturierung praktischen Handelns} der bisher indentifizierten
Beschreibungstechnik einer reduktionistisch konstituierten Strukturierung in
Systeme folgt. Ein solcher Ansatz geht damit davon aus, dass Systeme die
\emph{Form} sind, in welcher institutionalisierte Verfahrensweisen als
wichtiges Element einer sozio-technisch strukturierten Wirklichkeit in
Erscheinung treten. Technische Systeme haben damit \emph{Zwecke} und
\emph{Betreiber}, die als kooperative Subjekte innerhalb bürgerlicher
Verhältnisse verantwortungsbeladen die oben thematisierte \emph{Normalität}
herstellen und damit für die Geschlossenheit „adaptiver Zyklen“ einschließlich
gelegentlich auftretender „Umbauphasen“ im jeweiligen System verantwortlich
sind.

Über die API eines solchen Systems lassen sich also nicht nur mögliche
Differenzen und Probleme der \emph{Modellierung} des Betriebs eines solchen
Systems gegen einen Attraktor und damit die \emph{theoretische} Reaktion auf
\emph{imaginierte} Störungen kommunizieren, sondern ebensolche Erfahrungen aus
der direkten Konfrontation der Systembeschreibung als begründete Erwartungen
mit den real erfahrenen Ergebnissen. Damit rückt der Systembegriff aber näher
an menschliche kooperative Praxen.

Damit werden zugleich die impliziten Voraussetzungen von \cite{Ashby1958}
deutlich, der Systeme voraussetzt, die zu bewusst ausgewählten Reaktionen
fähig sind und darüber entsprechende Kanalkapazitätsberechnungen anstellt, die
hier nicht weiter von Interesse sind, da sie von sehr groben Annahmen über die
innere Struktur jener Beschreibungsformen ausgehen, in der die
Reaktionsfähigkeit systemintern (unter Einbeziehung eines Systemgedächtnisses,
das bei Ashby überhaupt keine Rolle spielt, in der Theorie agentenbasierter
Systeme aber schon) prozessiert wird.

In \cite{Boisot2011} werden weitere Ansätze entwickelt, wie die Fähigkeit
eines solchen Systems in Bezug auf die Bewältigung externer Störungen durch
interne Systemreaktionen beschrieben werden kann. Als grundlegender Ansatz
werden zwei weitgehend unstrukturierte Phasenräume externer Störungen (variety
of stimuli) sowie interner Reaktionen (variety of responses) betrachtet und
mit den Begriffen „Ashby line“ und „adaptive frontier“ zwei grundlegende
Mechanismen postuliert, die zur Überforderung des Regulationspotenzials eines
Systems führen -- die Masse der Störungen übersteigt die Masse der Reaktionen
des Systems (Abbildung 16.1) und die Frequenz der Störungen übersteigt die
Frequenz der Reaktionsfähigkeit des Systems (Abbildung 16.2).

Diese stark spekulativen Ansätze werden in \cite{Mann2019} weiter vertieft und in
den Kontext moderner TRIZ-Entwicklungen gestellt.

\section{Institutionelle Analyse von sozio-ökologischen\\ Systemen
  (IIRM)}

Literatur: \cite{Ostrom2007}, \cite{Anderies2004} 

Während in den ersten Seminarterminen vor allem Systemdynamiken ohne
menschliches Zutun betrachtet wurden (wobei dieses menschliche Zutun auch als
Ergebnis einer „Reduktion auf das Wesentliche“ ausgeblendet worden sein kann),
rückten im letzten Seminar mit dem Ansatz \emph{agentenbasierter Systeme}
erstmals Systeme in den Fokus, deren innere Dynamik eng mit menschlichem
Handeln verbunden ist. Wir haben dabei festgestellt, dass derartige Systeme
zusätzlich mindestens noch mit einer Zweck-Mittel-Perspektive aufgeladen sind.

\subsection{Noch einmal zum Systembegriff}

Derartige Systeme spielen im TRIZ-Kontext als \emph{Technische Systeme}
(klassische TRIZ-Termi\-nologie), \emph{Engineering Systems} oder
\emph{man-made systems} \cite{Souchkov2014} eine zentrale Rolle, wobei
Menschen hier sowohl als Subjekte des Handelns als auch als Objekte der
Systemdynamik in Erscheinung treten. Diese drei begriff\-lichen Ansätze nehmen
dabei ihrerseits Reduktionen der Rolle menschlichen Handelns auf
unterschiedliche Weise vor.

Der Ansatz „Technische Systeme“ -- zusammen mit dem grundlegenden TRIZ-Prinzip
der Idealität, dass eine gewünschte Funktion im Idealfall „von selbst“,
gänzlich ohne System zur Verfügung steht -- betont den äußeren Standpunkt des
Menschen im Design des entsprechenden Systems, während es für Menschen als
Objekte der („Von-Selbst“)-Wirkung keinerlei Entrinnen aus der Systemdynamik
gibt, wie schwierig diese sich für sie auch darstellen mag.

Die Modifikation zum „Engineering System“, wie sie mit dem Ansatz von „Trends
of Engineering System Evolution“ (TESE) in \cite{TESE2018} vorgenommen wird,
rückt den Systembegriff näher an den Technikbegriff der VDI-Richtlinien heran.
Damit wird eine deutlichere Verbindung des Systembegriffs mit menschlichen
Praxen thematisiert, in denen ein Netzwerk von Zweck-Mittel-Verhältnissen
einem Netzwerk von Systembeziehungen gegenübersteht, was genauer
beschreibungstechnisch zu fassen bleibt. Die VDI-Definition rückt dabei die
Zweck-Mittel-Beziehungen in den Vordergrund und bleibt mit dem Ansatz „Menge
von Systemen“ (oder sogar „Gebilden“) vage in der Frage der Beziehungen der
Systeme zueinander, während der TESE-Ansatz die traditionell starke
Strukturierung des Systembegriffs der klassischen TRIZ in „Gesetzen und Trends
der Entwicklung technischer Systeme“, insbesondere in der Betrachtung des
Verhältnisses von System und Obersystem(en), übernimmt, in der Betrachtung
systemübergreifender ingenieurtechnischer Praxen aber noch
Entfaltungsmöglichkeiten hat.

Der Ansatz „man-made systems“ schießt begriff\-lich über das Ziel hinaus, da
er -- wenigstens in der unmittelbaren Bedeutung der Worte -- die
naturgesetzliche Kontextualisierung der Bedingtheiten menschlichen Handelns
bei einer „Reduktion auf das Wesentliche“ ausblendet. In dieser Beschränkung
steht ein solcher Ansatz aber in eigentümlicher Nähe sowohl zu den Bemühungen
um die Formierung eines Konzepts von \emph{Resilienz} als auch zu den
Konzepten, die in \cite{Ashby1958} und \cite{Boisot2011} für eine
Systembewertung auf der Basis rein quantitativer Verhältnisse von
Störungsdruck und Reaktionsvermögen diskutiert wurden.  Auf diesem Hintergrund
sind auch die beiden Aufsätze \cite{Ostrom2007} und \cite{Anderies2004} zu
bewerten. In beiden Aufsätzen geht es darum, recht einfach gehaltenen Ansätze
agenten-basierter Systeme, wie etwa in \cite{Holland2006} zu finden, mit der
realen Komplexität institutionalisierter menschlicher Handlungsvollzüge zu
konfrontieren.  In \cite{Ostrom2007} wird dazu ein komplexes \emph{System von
  Parametern} zur Bewertung sozial-ökologischer Systeme vorgeschlagen.

In \cite{Anderies2004} wird ein \emph{grobes Modell} entwickelt, auf das sich
das Verhältnis von Ressourcen und Ressourcennutzern im Kontext einer
Infrastruktur und deren Bewirtschaftung in solchen kulturell-institutionell
überformten \emph{konkreten} sozial-ökologischen Systemen abbilden und damit
vergleichbar machen lässt.

Streitpunkt unserer Diskussion blieb insbesondere die Frage, ob sich in einem
solchen Ansatz der Übergang von Institutionalisierungsprozessen zu
institutionalisierten Strukturen ausreichend leistungsfähig darstellen lässt
oder hierfür nicht bereits schon zum Obersystem einer umfassenderen
Betrachtung kultureller Entwicklungen übergegangen werden muss.

In der Diskussion spielte auch der Begriff \emph{Ideologie} eine Rolle, was
hier in den Kontext der bisherigen Debatte gestellt und näher erläutert werden
soll.

Wir hatten den Systembegriff als eine beschreibungstechnische „Reduktion auf
das Wesentliche“ in vier (genauer beschriebenen) epistemischen Dimensionen
identifiziert, der in Strukturen, die zu ausgeprägtem Zweck-Mittel-Denken
fähig sind und damit einen Prozess des Abgleichs zwischen Planungs- und
Handlungsstrukturen kennen, auch für \emph{praktisches Handeln} leitend ist
und damit zu einer entsprechenden Strukturierung auch der Wirklichkeit
führt. Im Gegensatz zu den verschiedenen systemischen Beschreibungsstrukturen,
die weitgehend unabhängig voneinander sind -- die Beschreibung der Beziehungen
\emph{zwischen} diesen Systemen, insbesondere zwischen einem System und seinen
Komponenten, ist auf die Beschreibung von Input/Output-Verhalten reduziert --
gibt es in den realweltlichen Strukturierungen massiv „versteckte“ (also der
Reduktion zum Opfer gefallene) Abhängigkeiten auch über Systemgrenzen hinweg.
Diese Abhängigkeiten sind umso gravierender, je ungenauer und inadäquater die
Systemmodellierung ist.

Dies ist bei der Bewertung sozial-ökologischer und sozial-technischer Systeme
zu beachten, deren Mittel- und Ausgangspunkt oft Zweck-Mittel-Verhältnisse
sind, also (meist aus einem Obersystem inferierte) Vorstellungen und
Beschreibungsformen, was erreicht werden \emph{soll}. Eines der Hauptprobleme
der ökologischen Krise besteht darin, dass in den letzten 200 Jahren im Zuge
der industriellen Revolution ein System von sozio-technischen Systemen (vor
allem als \emph{Institutionen}) entstanden ist, das weitgehend dem Paradigma
der „Ausbeutung der Natur“ folgt, in dem also die Zweck-Mittel-Verhältnisse
als Beschreibungsformen eine hohe Dominanz in der Systemmodellierung
haben. Das gilt insbesondere auch für die Beschreibung der
Input/Output-Beziehungen derartiger Systeme, die einmal zur Steuerung der
inneren Stabilität und zum anderen zur Stabilisierung des äußeren Kontexts,
des eigenen „Handlungsraums“ dienen.

Hierbei hat sich ein ganzes Netz von Systemen und Beziehungen zwischen
Systemen aufgebaut, das stärker auf die Konservierung bestehender
Zweck-Mittel-Vorstellungen ausgerichtet ist als auf eine angemessene
Integration realweltlicher Entwicklungen, und auch in der eigenen
Handlungsdimension diesen Konservatismus perpetuiert. Dies gilt nicht nur für
das politische System der DDR in ihrer Endphase, sondern auch für das
Bienensterben, das in \cite{Jacobasch2019} genauer analysiert wird.  Obwohl
hier massive Gefahren für umfassendere Ökosystemstrukturen, die auf Bestäubung
durch Insekten aufbauen, auch beschreibungstechnisch klar auf dem Tisch
liegen, entwickelt ein System selbstbezüglicher Systembeschreibungen um das
Thema „Glyphosat“ herum, das auch wesentlich für die Legitimierung politischer
Entscheidungsprozesse ist, als \emph{Netzwerk von Systemen} ein starkes
Beharrungsvermögen, da genau \emph{jene} Informationen über die
netzwerkinternen Input-/Output-Spezifikationen \emph{nicht} angemessen
kommuniziert werden können und damit für die konkrete Dynamik des Netwzerks
von Systemen ohne Folgen bleibt.

Das „Netzwerk von Systemen“ als eigenes System hat auch unzureichende
Reflexionsstrukturen auf Systemebene und \emph{kann} diese enmergente
Entwicklung nicht systemintern darstellen. Derartige Phänomene sind bei der
Betrachtung der \emph{Robustheit} im Sinne von \cite{Anderies2004} zu
berücksichtigen.

\subsection{Zu Dynamiken sozio-ökologischer Systeme}

Noch einige Anmerkungen zu den in \cite{Anderies2004} zusammengetragenen
Beispielen von Entwicklungspfaden des \emph{groben konzeptionellen Modells}
sozio-ökologischer Systeme mit seinen vier Komponenten
\begin{itemize}[noitemsep]
\item Ressourcen
\item Ressourcennutzer
\item Öffentliche Infrastruktur und
\item Infrastrukturbetreiber.
\end{itemize}

Am Beispiel \emph{Straßenbau} wird zunächst argumentiert, dass ein
wesentlicher Eingriff in die bestehende Struktur in der Regel zu wesentlichen
Veränderungen auch bestehender sozio-kultureller Gleichgewichte führt
(Vereinfachen von Wegzügen, regionaler wirtschaftlicher Abschwung, Umwandlung
einer Region in einen Satelliten in unfassenderen regionalen
Ausdifferenzierungsprozessen).

Am Beispiel der \emph{Bewässerungssysteme in Bali} wird weiter gezeigt, in
welchem Umfang austarierte und historisch gewachsene informelle
vorkapitalistische Systeme der Infrastrukturbewirtschaftung durch eine
Kapitalisierung -- und sei es in einer „grünen Revolution“ -- sowie
Bürokratisierung unter Druck geraten, ohne dass „wissenschaftlich fundierte“
Regulierungsformen auf der Basis formalisierter Modelle auch nur annähernd so
erfolgreich in den Vollzugsformen sind wie die alten soziokulturellen Ansätze.

Insgesamt hat der Wechsel in den ökonomischen Formen großen Einfluss auf die
Leistungs\-fähigkeit von sozio-ökologischen Systemen; diese Systemstrukturen
müssen mit transformiert werden. Damit bestätigt sich auch hier die
\emph{Marxsche These vom Primat der Ökonomie}.

Der Übergang von direkter informeller Regelung in vorkapitalistischen Zeiten
zu indirekten Regulierungen über Geldformen führt schnell zur Intransparenz
von Wirkzusammenhängen und hebelt komplexe Interdependenzlogiken aus. Damit
wird es zunehmend einfach, auch \emph{Betreiberrenten} zu etablieren.
Praktisch wird versucht, dem durch staatliche Infrastrukturüber\-wachung
(„Netzagenturen“) zu begegnen.  Marx sieht das als Ausdruck des Widerspruchs
zwischen den Interessen des Gesamtkapitals und der Einzelkapitale.

Probleme nach dem Übergang zu kapitalistischen Bewirtschaftungsformen von
Infrastrukturen entstehen vor allem dann, wenn sich das
Kosten-Nutzen-Verhältnis durch Private defizitär entwickelt. Dieses Phänomen
ist inzwischen auch gut bekannt aus ÖPP-Projekten und führt zur \emph{Frage,
  ob Staaten pleite gehen können}. Nach kapitalistischer Logik ist in einem
solchen Fall die ökonomische Aktivität abzuwickeln, was die privaten
Gewährsträger dann auch tun, ggf. durch \emph{Insolvenz}. Die Aktivität kann
aber nicht abgewickelt werden, da die öffentliche Infrastruktur benötigt
wird. In der Regel wird dann nach der „öffentlichen Hand“ gerufen -- die Zeche
bezahlt der Steuerzahler -- oder die Leistungen fallen an einen
„Grundversorger“ mit deutlich anderen Preisstrukturen zurück. Typisches
Begleitphänomen ist der \emph{Verfall der Infrastruktur} durch Deinvestment,
bevor es zum Krach kommt.

\emph{Widersprüche} treten auch auf zwischen der (technischen) Betriebslogik
und der (betriebswirtschaftlichen) Betreiberlogik, da hierbei unterschiedliche
Zeit-, Raum- und Kausalhorizonte eine Rolle spielen. Insbesondere kann das
wissenschaftlich-technische \emph{Beschreibungsmodell} als Basis der
Handlungsvollzüge mehr den ökonomischen Wunschvorstellungen folgen als den
realweltlichen kausalen Zusammenhängen und damit zum Zusammenbruch der
realweltlichen Reproduktionsstrukturen führen ($\Omega$-Phase des adaptiven
Zyklus). Da die Wirkung jenseits einer Triggerschranke sehr schnell an Fahrt
gewinnt, spielt vorher oft der Aufbau einer ideologischen \emph{Scheinwelt}
eine Rolle, die trotz entsprechender realweltlicher Signale eigene interne
Stabilisierungsmechanismen in der Unterscheidung von Wesentlichem und
Unwesentlichem entwickelt.

Am Beispiel des Aral-Sees wird weiter thematisiert, dass Nutzen einer
spezifischen Infrastrukturentwicklung für die eine Gruppe deutlich negative
Auswirkungen für andere Gruppen haben kann. Dies ist eine andere
Widerspruchsebene zwischen verschiedenen Interessenlagen und Zielen, die --
wiederum durch entsprechende Hebelwirkungen -- zu wesentlichen Deformationen
(bzw. -- positiv gesprochen -- Transformationen) bestehender großflächiger
ökonomischer Strukturen führen kann. In diesem Sinne ist \emph{jede}
spezifische Bewirtschaftungsform einer öffentlichen Infrastruktur zugleich
eine spezifische Form des Prozessierens widersprüchlicher Interessenlagen und
Ziele.

In \cite{Anderies2004} werden nach \cite{Ostrom1990} acht wesentliche
Strukturmerkmale für die langfristig stabile Bewirtschaftung von
Infrastrukturressourcen postuliert:
\begin{itemize}[noitemsep]
\item[1.] Klar definierte Grenzen sowohl räumlich als auch bzgl. der
  Zugangsbedingungen.
\item[2.] Proportionale Äquivalenz zwischen Nutzen und Kosten. Für jeden
  Einzelnen müssen Nutzen und Kosten in angemessenem Verhältnis stehen.
\item[3.] Beteiligung aller Infrastrukturnutzer an der Weiterentwicklung der
  Nutzungsregeln.
\item[4.] Die Nutzung und Entwicklung der Ressource muss gemeinschaftlich
  überwacht werden.
\item[5.] Graduelle Sanktionen. Verstöße gegen die Regeln müssen mit
  wachsender Schärfe sanktioniert werden.
\item[6.] Es muss Konfliktbewältigungs-Mechanisme geben („local“ und „low
  cost“).
\item[7.] Das Recht der Nutzer auf Bildung eigener Organisationsformen muss
  gewahrt sein.
\item[8.] Mehrebenenprinzip der Organisation von Aneignung, Bereitstellung,
  Überwachung sowie Durchsetzung, Konfliktlösung, Governance.
\end{itemize}

In der Konsequenz wird argumentiert, dass sich nicht nur sozio-ökologische
Systeme in adaptiven Zyklen (mit Umbauphasen) bewegen müssen, sondern auch die
Infrastrukturbewirtschaftungsformen. Gerade der \emph{Umbau} letzterer kann
ein wichtiges Moment der \emph{Robustheit} ersterer sein. Also auch hier:
Evolution des Systems schließt die Option des revolutionären Umbau seiner
Komponenten ein.

\section{Sozio-technische Systeme und\\ Transformationsprozesse (IIRM)}

Literatur: \cite{Geels2007}, \cite{Foxon2009}; Zusatzliteratur:
\cite{Ropohl2009}

Das Seminar beschäftigte sich mit Transformationsprozessen von und in
sozio-technischen Systemen. Dabei wurde mit der Multilevel Perspektive von
\cite{Geels2007} begonnen und ihrer Typologie von Transformationsprozessen.
Zuerst wurde sich das Modell angeschaut und die Verwobenheit der einzelnen
Ebenen diskutiert, welche in Nischeninnovation, sozio-technisches Regime und
sozio-technische Landschaft unterteilt sind. Entscheidend für die Bestimmung
der Falltypologie sind die Einflussverhältnisse von Ebene zu Ebene.
Dementsprechend ergeben sich die vier Transformationsarten
\begin{itemize}[noitemsep]
\item[1.] Transformation,
\item[2.] De-Alignment und Re-Alignment,
\item[3.] Technische Substitution und
\item[4.] Rekonfiguration
\end{itemize}
nicht als einzige Möglichkeit, sondern ihnen müssen die Sonderfälle
\begin{enumerate}[noitemsep]
\item[0.] „Normaler“ Produktionsprozess und 
\item[5.] Disruptiver Wandel
\end{enumerate}
an die Seite gestellt werden.

Im nächsten Schritt ging es um die kritische Verbesserung in \cite{Foxon2009},
welche adaptives Management und Transitionsmanagement verknüpft. Dort ergibt
sich aus dieser Kombination ein Kreislauf adaptiver Managementansätze, welche
über \emph{established context}, \emph{goals strategies}, \emph{evaluation
  indicators} und endgültig zu \emph{collect data monitor} läuft und
Akteursansätze aus der Transitionsarena des Transitionsmanagements
einbezieht. Es ergeben sich so ganz spezifische Verhältnisse, welche in Fragen
der adaptiven Kapazität, der Risikobewertung, der Stakeholdereinbeziehung, der
räumlichen Skalierung, der Führung und der Anregung systematischen Wandels
unterschiedliche Bewertungen und entsprechend unterschiedliche Methoden
provozieren.

Die Diskussion drehte sich um die Belastbarkeit und Verwertung der
kombinierten Frameworks und sechs Probleme wurden eingekreist.
\begin{enumerate}[noitemsep]
\item In \cite {Geels2007} ist nicht klar, wie die $y$-Achse funktioniert,
  insbesondere, da sie als \emph{increasing structure} ausgewiesen ist und
  somit das innere Verhältnis im Regime nicht klar ist, besonders das
  Verhältnis von culture, policy and science.
\item Folglich ist nicht klar, was \emph{Struktur} überhaupt sein soll, da zum
  einen eine Handlungs-Managementperspektive bemüht wird, und zum anderen
  dennoch klassische Strukturzuweisungen erfolgen, die nicht wirklich erklärt
  werden.
\item Das Verhältnis von Nischeninnovation und Regime ist nicht klar. Gibt es
  auch eine gesteuerte Regime-Nischeninnovation?
\item Darauf beziehend ist nicht deutlich, was eine Nischeninnovation
  überhaupt ist; ein Prototyp, eine Verfahrensweise, eine Idee?
\item Im 4. Fall ist nicht klar, warum die $y$-Achse fehlt und von Symbiose
  geredet wird.
\item Die begriff\-liche Verwendung von Transition und Transformation in ihrer
  Synonymisierung ist nicht klar. In anderen Wissenschaftsdisziplinen steht
  \emph{Transition} für eine Veränderung mit Erhaltung von Systemkomponenten
  und \emph{Transformation} für eine echte vollständige Änderung aller Formen
  des Systems und somit aller Komponenten. Die Verwendung ist hier in keiner
  Weise klar.
\end{enumerate}

Positiv konnte dennoch die Zielrichtung gewertet werden, welche Handlungen und
diesen entsprechende politische Verfahrensweisen und Abläufe in den Fokus
nimmt. Generell ließen sich erneut auch hier die klassischen Probleme der
Systemtheorie finden. 
\begin{itemize}[noitemsep]
\item Erstens das \textbf{Ebenenproblem}; es ist entscheidend, auf welcher
  Ebene die Modellierung beginnt und was in den Fokus genommen wird, denn
  Fragen von Außen und Innen, von Input und Output und der Leistungsfähigkeit
  des Systems sind davon abhängig.
\item Zweitens ist das \textbf{Durchsatzproblem} erneut entscheidend. Für die
  Performanz als auch für die Übertragung ist entscheidend, was das System,
  auch als Multilevel Modell, antreibt.
\item Drittens steht das \textbf{Strukturproblem}. Immer wieder werden
  Strukturvorstellungen untergeschoben, welche sich nicht aus der induktiven
  Modellierung ergeben, wie Wirtschaft, Politik, Kultur oder ähnliche
  schwammige Begriffskonzepte.
\end{itemize}

\subsection{Transitionspfade}

In \cite{Geels2007} werden eine Reihe von Transitionspfaden beschrieben, die
in Umbauphasen von Systemen beschritten werden. Damit wird versucht -- ohne
dies allerdings zu explizieren -- etwas Struktur in die in \cite{Holling2000}
weitgehend unverstandene $\Omega$-$\alpha$-Umbauphase zu bringen. Auch
\cite{Geels2007} bleibt dabei weitgehend auf einer phänomenologischen Ebene
stehen und entwickelt wenig Konzeptionelles, gesellschaftliche, ökonomische
und technische Entwicklungen zusammen zu denken. Der Aufsatz geht auch nicht
so weit wie die TRIZ Evolutionsforschung, hier Gesetze oder wenigstens Muster
\emph{explizit} zu formulieren.

Im bisher im Seminar entwickelten Verständnis ist die Notwendigkeit zum
Systemumbau dadurch gegeben, dass die \emph{lokalen} Entwicklungsmöglichkeiten
auf dem Systemattraktor ausgeschöpft sind, weil sich das System durch ständig
fortschreitende „Idealisierung“ in ein lokales Extremum des Attraktors
manövriert hat (Hollings K-Phase), in dem sich externe Störungen aufschaukeln
und das System in einen instabilen Zustand treiben (Hollings $\Omega$-Phase),
aus dem durch Umorganisation (Hollings $\alpha$-Phase) ein neuer, vom
ursprünglichen weit entfernter Referenzpunkt auf dem Systemattraktor
eingenommen wird.

Ein solcher Systemumbau übt einen größeren Stress auf die mit dem System
verbundenen weiteren Systeme (Komponenten im System, Nachbarkomponenten im
Obersystem, allgemeine „unsystematische“ Beziehungen zu anderen Systemen) aus.
In diesem Sinne migrieren systemische Umbauprozesse längs der
Systembeziehungen mehr oder weniger weit durch das Netzwerk der Systeme.

Umgekehrt resultiert der Störungsstress aus anderen, mit dem System kausal
verbundenen Systemen, wobei in den klassischen Ansätzen die Bindungen System
-- Obersystem (bzw. System -- „Umwelt“) sowie System - Komponente in der Regel
separat von allgemeinen Bindungen (etwa zwischen den Komponenten innerhalb
eines Systems bzw. -- dasselbe Bild auf einer anderen Betrachtungsebene --
zwischen Teilsystemen eines Obersystems) betrachtet werden. Wir hatten bereits
festgestellt, dass eine solche „Spezialbetrachtung“ einer Mikro- und
Makroevolution nur bei Beziehungen zwischen Systemen sinnvoll ist, die sich
auf deutlich verschiedenen Eigenzeitskalen bewegen: Für das „schnellere“
System kann das langsamere in erster Näherung als statisch betrachtet werden,
für das „langsamere“ das schnellere als weitgehend störungsfrei und damit
deterministisch oder wenigstens stochastisch, da sich die Störungen des
schnellen Systems auf der Zeitskala des langsameren weitgehend ausmitteln.

Betrachten wir aus einer solchen Perspektive die Argumente aus
\cite{Geels2007} und \cite{Holling2000}, so fällt zunächst der stark
agentenbasierte Ansatz der ersteren Arbeit auf. Agenten gibt es auch bei
Holling, siehe etwa \cite[Tab. 2]{Holling2000}, doch setzt \cite{Geels2007}
mit „agency“, „regime“, „organisation“ und „institution“ den Fokus deutlich
anders. Mit allen vier Begriffen, die weitgehend synonym verwendet werden,
wird auf die Ablauforganisation und nicht die Aufbauorganisation von Systemen
verwiesen, ohne allerdings die betrachteten Systeme in jedem Fall genau zu
umreißen. Eher ergeben sich das System und seine Grenzen in den von uns
identifizierten drei (oder vier) Reduktionsdimensionen von
Beschreibungskomplexität „von selbst“ aus der Bewegung heraus.

In einem solchen „panta rhei“ Ansatz werden \cite[S. 401]{Geels2007}
Störungsquelle und Ort des Umbaus differenziert, was mit den oben noch einmal
entwickelten eigenen Modellansätzen gut harmoniert. Die auf dieser Basis
zunächst entwickelte Typologie \cite[Fig. 2]{Geels2007} entspringt allerdings
einer Empirie, die sich nur schwer auf unseren Modellansatz abbilden lässt,
was dann auch später \cite[S. 402]{Geels2007} eingeräumt wird: „empirical
levels are not the same as analytical levels in MLP“ (multi level
perspective).

Die weiter ins Feld geführten „organisational levels“ -- \emph{individual,
  organizational subsystem, organisation, organisational population,
  organisational field, society, world system} -- konzentrieren sich, wenn
dies mit dem Systembegriff relatiert wird, vor allem auf die
institutionalisierten Strukturen der \emph{Aufbauorganisation} der jeweiligen
Systeme (etwa das „System Gesellschaft“) samt ihrer Luhmannschen „Codes“, in
denen jene Systeme überhaupt sprachlich \emph{in der Lage sind}, über
Störungen zu kommunizieren und wenigstens grob zu entscheiden, ob man es mit
einem „incremental, radical, system or techno-economic“ Typ von Störung aka
„Innovation“ zu tun hat und darauf typangemessen zu reagieren.

Wenn eine „conjuncture of multiple development“ \cite[3.2.]{Geels2007}
bedeutsam ist, so wird die These von der Quelle der Störung in einem
Einzelsystem schon fragil, wenn sich jene Störung im Netzwerk der Systeme
wellenförmig fortpflanzt und so kaum noch zu unterscheiden ist, ob jene
„Welle“ von einer punktförmigen Quelle ausgelöst wurde oder ein emergentes
Phänomen des Netzwerks ist (das ja selbst auch wieder als System betrachtet
werden kann) als resonante Antwort auf eine externe Störung. Dass gerade in
Zeiten tiefgreifender technologischer Umbrüche derartige emergenten Phänomene
in komplexen hierarchisch aufgebauten organisationalen Netzwerken nicht außer
Betracht bleiben können, ist ebenso klar wie theoretisch schwierig zu fassen.

Erschwerend kommt hinzu, dass in derartigen Transitionen drei Sphären
wesentlich interagieren:
\begin{enumerate}[noitemsep]
\item Die Sphäre der Beschreibungsformen (das gesellschaftlich verfügbare
  Verfahrenswissen),
\item Die Sphäre der real existierenden, in Systemen strukturierten
  Wirklichkeit (die institutionalisierten Verfahrensweisen) und
\item Die kooperativen Subjekte (mit ihrem „privaten“ Verfahrenskönnen).
\end{enumerate}
Zwischen den Sphären 1 und 2 bestehen \emph{kausale} $m:n$-Beziehungen, durch
Sphäre 3 werden diese Beziehungen \emph{praktisch} vermittelt.

Die drei „kinds of rules“ (\cite[3.3.]{Geels2007} -- der Begriff „Institution“
wird hier bewusst abgewählt \cite[S. 403, Fußnote 1]{Geels2007}), über welche
eine solche Vermittlung in einem „model of agency“ läuft, werden als Basis
einer gemeinsamen „interpretation of the world“ konkreter kooperativer
Subjekte identifiziert, die sich im \emph{Handeln} jener Strukturen („use
rules“, „rules are not only constraining but also enabling“) bewähren müssen
und befestigt werden. Dies sind die Formen, in denen die \emph{Pragmatik}
zwischen den Sphären 1 und 2 vermittelt und damit \emph{realweltliche
  Begriffsbildungsprozesse} induziert werden bis hin zur „conceptualisation of
sociotechnical landscape that \ldots{} forms an external context that actors
cannot influence in the short run“.

Damit werden die Argumentationen in \cite[Fig. 4 und Table 1]{Geels2007} in
ihrem absoluten Anspruch eines „environmental change“ fragwürdig, da Einträge
wie „low“ und „high“ (Table 1) nur gegen klare Etalongrößen Sinn ergeben, hier
also implizit Eigenzeiten und Eigenräume eines Obersystems als Referenz dienen
(bzw., wenn man sich wie ebenda allein an der Ablauforganisation
interagierender Systeme orientiert, ein solches Referenzsystem erst noch
identifiziert werden muss). Dass jenes „environmental system“ seit wenigstens
10\,000 Jahren als kulturell überformt betrachtet werden muss, sei nur in
Parenthese angemerkt. Eine solche Einhegung wird dann mit den Begriffen
\emph{Frame} und \emph{Closure} \cite[S. 405]{Geels2007} auch versucht, jedoch
auf einem recht simplen Niveau unmittelbar transformierender Wirkung
differierender Wachstumsraten. In anderen Beispielen wird jedoch gezeigt, dass
Ungleichheiten in der Ressourcenverfügung von Akteuren auch oft eingesetzt
werden, um anstehende Transitionen \emph{zu verhindern}. Der emergente Effekt
ist dann mglw. eine sinkende Performanz des Gesamtsystems. Selbst der
beschriebene Wettbewerb auf der Basis differierender Wachstumsraten kann auf
der Emergenz\-ebene des Gesamtsystems gegenteilig wirken, wie etwa Marx mit
seinem Gesetz der fallenden Profitrate argumentiert (egal, ob dieses Gesetz
nun wirklich wirkt oder in einem dissipativen Systemkontext die Argumente
anders zu bedenken sind).

Damit lassen sich die sechs Transitionsmuster P0 bis P5 wie folgt auf Hollings
Modell adaptiver Zyklen abbilden:

\paragraph{P0:}
Das System ist in der r-Phase und kann den Veränderungsdruck aus einer seiner
Komponenten („no external landscape pressure“) absorbieren. Dasselbe bleibt
richtig, wenn der Druck „von außen“ (also von anderen Systemen) kommt und
nicht zu groß wird.

\paragraph{P1:}
Druck von „außen“, kein Druck aus den Komponenten, System beim Verlassen oder
jenseits der K-Phase. Das System kann nur durch Reorganisation der Beziehungen
reagieren. Die Autoren sind weitgehend ratlos, vermischen allerdings auch zwei
Modi:
\begin{enumerate}[noitemsep]
\item Das System ist in der $\alpha$-Phase eigener Umbauprozesse.
\item Das System ist im Übergang in die $\Omega$-Phase.
\end{enumerate}

Das Beispiel (Dänische Hygiene-Transition) ist klar eines für die Dynamik in
der $\Omega$-Phase, dem auf der TRIZ-Seite ein Übergang von einer S-Kurve auf
eine andere entspricht. Wie das geht, versteht man dort allerdings auch
nicht. Das Beispiel folgt dem \textbf{Modell $\Omega_1$:} Das System wird
reorganisiert, die Funktion nach außen bleibt erhalten bzw. wird verbessert.

\paragraph{P2:}
Das System wird zerlegt, seine Komponenten anders reorganisiert. Als typisches
Phäno\-men wird „Vakuum“ diagnostiziert, wie es auch als Machtvakuum beim
Zerfall des Ostblocks zu beobachten war. Das im Text angegebene Beispiel
berücksichtigt nicht, dass sich die neuen Bedingungen (Automobil ersetzt
Transport durch Pferde) bereits länger auch strukturell in den Subsystemen --
„im Schoße der alten Gesellschaft“ -- herausgebildet haben. Im weiteren
Beispiel bleibt die \emph{K-Wellen-Dynamik} um 1890 unberücksichtigt.

\paragraph{P3:}
Der Druck kommt nicht aus der Umgebung, sondern von einzelnen Komponenten. Das
System kann sich selbst so reorganisieren, dass die für die reorganisierten
Komponenten erforderlichen neuen äußeren Bedingungen sichergestellt werden,
ohne die Funktionalität des Systems nach außen aufzugeben. Das
Erklärungspotenzial ist dünn, „avalange change“ und „disruptive change“ als
„landscape pressure“ existieren erstens dauernd als „disturbances“ und sind
zweitens hier nicht kausal, wenn auch möglicherweise triggernd. Im Beispiel
bleibt die Wirkung der K-Welle um 1890 ebenfalls unberücksichtigt. Ebenso
werden für solche Transitionen typische „Marktbereinigungen“ nicht besprochen,
da das produktive Ausrollen der neuen Technologien in größerem Umfang auch
größere Mengen vorgeschossenes Kapital erfordert.

\paragraph{P4:}
Komponenten in $\Omega$-Phase treffen auf ein System in $\alpha$-Phase.
Eigentlich wird die Transition aber aus einer kausal tiefer liegenden
Technologieebene getriggert, die Auswirkungen auf \emph{viele} Komponenten hat
und diese in $\Omega$-Phase bringt, was jedoch vom System in $\alpha$-Phase
(und damit in besonders flexibler r-Phase) aufgefangen werden kann. So auch
das Beispiel.

\paragraph{P5:}
Im Gegensatz zu P4 lassen sich die Änderungen \emph{nicht} im System
auf\-fangen und werden weitergeleitet. Damit werden auch die Beziehungen des
Systems nach außen instabil. Die Autoren sind ziemlich ratlos („sequence of
transition pathways“) und haben auch kein Beispiel zur Hand.

Generell wird angemerkt, dass derart komplexe Prozesse nicht nur nicht
monokausal erklärt werden können, sondern auch die Variablen in mathematischen
Beschreibungsmodellen nicht in abhängige und unabhängige unterteilt werden
können. Deshalb könne man nur von \emph{Entwicklungsmustern} sprechen. Die
weiter referenzierten Prozess-Theorien blenden mit einer Fokussierung auf
Ereignisketten in zeitlicher und kausaler Verkettung allerdings
\emph{strukturelle Momente} weitgehend aus, die sich mit fortgeschrittenen
mathematischen Methoden durchaus auch in komplexer strukturierten Phasenräumen
noch gewinnen lassen.

Giddens' Ansatz der „rules as structures, which are recursively reproduced
(used, changed) by actors“ weist in eine Richtung, in der solche strukturellen
Erkenntnisse mit Beschreibungen von Handlungsvollzugsformen konkreter
kooperativer Subjekte auf verschiedenen Abstraktionsebenen zu kombinieren
wären.

\subsection{Adaptives und transitionales Management \cite{Foxon2009}}

Die im ersten Teil der Anmerkungen diskutierten Transitionspfade haben ein
wesentliches epistemisches Problem -- das Problem des äußeren Standpunkts, von
dem aus Bechreibungsformen entwickelt werden, um Einfluss auf realweltliche
Wandlungsprozesse zu gewinnen.

\cite{Foxon2009} schlägt hier einen komplett anderen Zugang vor, indem diese
Beschreibungs- und Analyseformen von den beteiligten Akteuren (mit
methodischer Unterstützung) selbst entwickelt werden. Der Zugang folgt dennoch
klassischen TRIZ-Methodiken der Modellierung, indem zunächst ein Obersystem
als Kontext der Bestimmung der Zwecke des untersuchten Systems identifiziert
wird, um dann das System selbst genauer zu modellieren. Jene Modellierung wird
aber nicht als externer Prozess verstanden, sondern als Konsensfindung
gemeinsamer Beschreibungsformen der Stakeholder selbst, ohne welche
kooperatives Agieren nicht möglich ist (siehe das Konzertbeispiel). Dieser
Modellierungsprozess wird damit zugleich zum \emph{politischen} Prozess, da
als Ergebnis nicht nur anerkannte Beschreibungsformen erwartet werden, sondern
\emph{institutionalisierte Verfahrensweisen}. Ersteres (anerkannte
Beschreibungsformen) ist zweiterem allerdings vorgängig in dem Sinn, dass
widersprüchliche Anforderungen zunächst artikuliert werden müssen, ehe diese
Widersprüche gelöst werden können. Dies entspricht aber auch den zwei Phasen
des TRIZ-Prozesses (im OTSM-Verständnis).

In einer solchen Modellierung sind zwei dialektische Prinzipien bereits
eingebaut
\begin{enumerate}[noitemsep]
\item die dynamische Weiterentwicklung des Modells selbst längs der
  Differenzen zwischen begründeten Erwartungen und erfahrenen Ergebnissen der
  Vollzugsform -- unter Einbeziehung einer möglichst breiten
  Stakeholder-Landschaft (TRIZ-Trend der Vollständigkeit der Teile des
  Systems) und
\item die Weiterentwicklung der Zwecke im Obersystem, in dem das System selbst
  als Komponente („Stakeholder“) erscheint und dort über seine spezifizierte
  Schnittstelle seinen Beitrag dazu in der Vollzugform einbringen kann.
\end{enumerate}
Ersteres ist Schwerpunkt des Ansatzes \emph{Adaptives Management}, zweiteres
des Ansatzes \emph{Transitionales Management}. In beiden Fällen ist die
Weiterentwicklung der Beschreibungsform Teil der Vollzugsform.

Damit ist \cite{Foxon2009} in gewissem Sinne orthogonal zu \cite{Geels2007},
indem \emph{das Innere} einer Transitionsphase in ein methodisches Gerüst
gebracht wird. Es steht natürlich sofort die Frage, für welche der
Transitionstypen in \cite{Geels2007} dieses methodische Gerüst brauchbar ist
oder ob auch hier wiederum ein Konzept als „Allzweckwaffe“ vorgeschlagen wird.

Beide Ansätze unterscheiden sich weiter in der Strategie der
Komplexitätsreduktion.  Während adaptives Management eine Vielzahl
\emph{verschiedener} funktionaler Parameter in der konkreten Ausprägung im
lokalen Kontext eines \emph{Unikats} betrachtet, erfolgt die Reduktion auf der
Ebene des \emph{transitionalen Managements} auf der Basis eines
\emph{funktionalen Prinzips}, nach dem \emph{gleichartige} funktionale
Parameter gebündelt werden (etwa „Energieversorgung der Zukunft“,
„Wasserreinhaltung“, „Biodiversität“), um dieses Prinzip genauer und besser zu
verstehen. Während zweiteres also mehr der Devise „global denken“ folgt, steht
ersteres in der Perspektive „lokal handeln“.

Ein solches Phänomen der verschiedenen Bündelung hatten wir bereits oben im
Kausalverhältnis der Sphären 1 und 2 (der Beschreibungsformen und der
systemisch strukturierten Wirklichkeit) angetroffen. Dieses Phänomen ist auch
aus der Komponententechnologie \cite{Szyperski2002} gut bekannt -- der
\emph{Zuschnitt} von Komponenten erfolgt unter Bündelung \emph{gleichartiger}
Anforderungen aus \emph{verschiedenen} Quellen, der \emph{Einsatz} von
Komponenten erfolgt durch Bündelung \emph{verschiedenartiger} Funktionalitäten
im \emph{gleichen} Zielsystem. \cite{Szyperski2002} zeigt, dass dies bis hin
zur Ausdifferenzierung von Berufsbildern verfolgt werden kann --
Komponentenentwickler erscheinen im „design for component“ als
Fachspezialisten, Komponentenmonteure im „design from component“ als
Generalisten.

Auch dies hat sein Analogon in der TRIZ-Methodik, wo „global denken“ den
Schritt von der abstrakten Problemstellung zur abstrakten Lösung markiert, die
man im besten Fall bereits als „technische Komponente“ (nach Deployment und
Konfiguration) in konkrete Lösungen einbauen kann, in den meisten Fällen aber
noch eine klare Konkretisierung auf die komplexe und einzigartige
\emph{realweltliche} Problemsituation erforderlich ist. Wir haben also auch
auf dieser Ebene dieselbe Unterscheidung wie die zwischen Komponentenbauern
(„design to component“) und Industrieanlagenbauern („design from component“)
im Technikbereich.

Der Aufsatz bricht damit eine Lanze für die Koevolution von Beschreibungsform
und Vollzugsform in kooperativen Zusammenhängen.  Beides ist nicht
widerspruchsfrei, allerdings kann versucht werden, artikulierte Widersprüche
mit entsprechenden Transitionsstrategien im Netz der Systeme bewusst an eine
solche Stelle zu verschieben, wo sie gelöst werden können.

\section{Systembegriff in der TRIZ-Methodik (Gräbe)}

Literatur: \cite{Koltze2017}

Es wurden noch einmal die verschiedenen Dimensionen gegenübergestellt, die bei
der Fassung des Begriffs \emph{technisches System} zu beachten sind.

\begin{enumerate}
\item Ein Black Box -- White Box Ansatz: Von außen werden technische Systeme
  als Black Box betrachtet, die durch eine Spezifikaton
  (Beschreibungsdimension) und durch spezifikationskonformes Verhalten
  (Vollzugdimension) charakterisiert sind. Sie erfüllen dabei eine
  \emph{primär nützliche Funktion} (PNF, „core concern“) als \emph{Zweck} der
  Existenz des Systems und sind damit im Sinne von \cite{Szyperski2002} als
  \emph{Einheit der Abstraktion} zu betrachten.

  Die White Box korrespondiert zur Implementierung, deren genaues Verständnis
  \emph{Expertenwissen} auf dem Gebiet der PNF erfordert.  Nebenfunktionen
  werden oft aus anderen Komponenten importiert. Siehe dazu das CORBA
  Komponentenmodell.

  Wir haben damit eine \emph{Welt Technischer Systeme} vorliegen, die durch
  Relationen gegenseitiger Dienstbarkeit zusammengehalten wird.
\item Technische Systeme sind aus ihrer Beschreibungsdimension durch drei
  Arten von Komplexitätsreduktion charakterisiert
\begin{enumerate}[noitemsep]
\item Abgrenzung gegen die Bestandteile -- Komponenten
\item Abgrenzung nach außen -- Schnittstelle
\item Reduktion der inneren Beziehungen auf kausal wesentliche -- Zwecke,
  Interessen.
\end{enumerate}
\item Die VDI-Definition des Begriffs \emph{Technik}
\begin{enumerate}[noitemsep]
\item Menge von nützlichen Artefakten und Sachsystemen
\item Prozesse ihrer Herstellung
\item Prozesse ihrer Verwendung
\end{enumerate}
\item Die drei Ebenen unserer Technikdefinition
\begin{enumerate}[noitemsep]
\item Gesellchaftlich verfügbares Verfahrenswissen
\item Institutionalisierte Verfahrensweisen
\item Privates Verfahrenskönnen
\end{enumerate}
als Modi vernünftigen wissenschaftlich-technischen Denkens und Handelns.
\end{enumerate}

Für einen submersiven Systembegriff ergibt sich daraus: Die Relation
Obersystem -- System ist nur eine spezifische Relation von Nachbarsystemen und
vermittelt primär die \emph{Nützlichkeit} (Zweck) eines Systems. Im Sinne des
Trimmens können mehrere Funktionen in einer Komponenten zusammengelegt werden,
dann hat diese Komponente als System auch \emph{mehrere Obersysteme}.

Davon zu unterscheiden sind verschiedene Ebenen der Abstraktion (Baum, Wald;
BWL, VWL), die sich aber durch unterschiedliche raum-zeitliche Ausdehnungen
unterscheiden und damit in einem Verhältnis der Makro- (langwellig) und
Mikroevolution (kurzwellig) stehen. Diese Dfferenzierung erlaubt es, auf der
Makroebene Mikrofluktuationen „auszumitteln“ und dort deterministisches
Verhalten für die Untersuchung der Makroebene vorauszusetzen. Für
Untersuchungen auf der Mikroebene können dagegen die Verhältnisse auf der
Makroebene als weitgehend konstant angenommen werden.

\section{ TRIZ und Systematische Innovationen in komplexen
  Umgebungen (Gräbe)} 

Literatur: \cite{Mann2019}

Ziel dieses Seminartermins war es, genauer zu verstehen, wie
Transformationsszenarien im Kontext der TRIZ-Methodik konzeptualisiert werden.
Zunächst ist dazu zu bemerken, dass das Transformationskonzept in der
OTSM-TRIZ eine relativ zentrale Rolle spielt, denn die Lösung einer
widersprüchlichen Anforderungssituation, die sich in einem systemischen
Kontext ergeben hat, besteht in einer geeigneten Transformation dieses
systemischen Kontexts in einen Zustand, in dem der Widerspruch aufgelöst ist.
Die TRIZ-Methodik hilft dabei, einen solchen Transformationspfad auf
systematische Weise zu finden.

Dieser Ansatz unterscheidet sich in zwei Dimensionen wesentlich von den bisher
betrachteten:
\begin{enumerate}[noitemsep]
\item Es geht um die \emph{praktische} Vollzugsdimension einer solchen
  Transformation.
\item Der Zugang ist problemgetrieben und nicht analysegetrieben.
\end{enumerate}

Letzteres (die Analyse) beginnt mit dem Thema „TRIZ und Business“ (wieder)
eine größere Rolle zu spielen, indem praktische Transitionserfahrungen
analytisch aufgearbeitet und systematisiert werden. Damit nähert sich die
TRIZ-Welt der bisher im Seminar betrachteten Transitionsforschung weiter an,
wobei weiterhin ein wesentlicher Unterschied im Theorie-Empirie-Verhältnis
zwischen beiden Communities besteht.

\cite{Mann2019} ist ein Versuch, auf der Seite der TRIZ-Welt etwas
Theorieboden zu gewinnen. Der TRIZ-Gegenstand wird zunächst wie folgt
charakterisiert: „TRIZ is essentially a distillation of the `first principles'
of problem solving. It was originally developed for complicated technical
problem and opportunity situations and, through ARIZ, has been deeply
optimized for such roles. Increasingly, however, the world has become
dominated by complex, non-technical situations, and in these environments many
of the tools, methods and processes of traditional TRIZ become highly
inappropriate.“ Weiter heißt es auf Seite 2 „Traditional TRIZ was very much
focused on technical problems.  And moreover, the large majority of these
technical problems turned out to be complicated. And so traditional TRIZ
worked. In today's massively inter-connected world, however, it is
increasingly rare that we find ourselves able to `merely' focus on just the
technical problem“. Damit werden die Problemlösekapazitäten von TRIZ als
erfinderisches Wirken in \emph{jungen} Technologien noch einigermaßen korrekt
beschrieben. Dies gilt allerdings schon nicht mehr für die meisten der
heutigen TRIZ-Praxen, die sich auf Problemlösungen (auch ingenieur-technischer
Art) in \emph{funktionierenden unternehmerischen Kontexten} beziehen und damit
neben der Lösung des technischen Problems auch die Implementierung dieser
Lösung im unternehmerischen Kontext im Auge haben müssen.  Damit werden
Systeme aber zu sozio-technischen Systemen, denn Zwecke, Ziele,
Business-Strategien und Interessen geraten ins Blickfeld. Eine solche
Erweiterung des Gesichtsfelds von rein ingenenieur-technischen zu
sozio-technischen Fragestellungen war auch schon Thema der
DDR-Erfinderschulen, die (u.a.) Probleme des massiven
COCOM-Technologieboykotts und entsprechende Importablösungen zu lösen
hatten. Derartige Fragen stehen auch heute im Zentrum wichtiger
TRIZ-Anwendungen, nicht zuletzt im Kontext der Patentumgehung.

Allerdings steht die Frage, ob D. Mann mit seiner eigenen Charakterisierung
der TRIZ-Methodik als „first principles of problem solving“ richtig liegt oder
ob sich diese „first principles“ -- selbst in den theoretischen Grundlagen der
TRIZ-Methodik -- nicht doch über \emph{mehrere} Ebenen der Abstraktion
erstrecken, auch wenn dies in den Texten zur theoretischen Fundierung der
TRIZ-Methodik nur selten genauer ausgeleuchtet wird.

Weiter stellt sich die Frage, ob nicht auch im Management-Kontext
\emph{Techniken des Problemlösens}, oder anders -- institutionalisierte
Verfahrensweisen --, im selben Umfang eine Rolle spielen wie beim Lösen rein
ingenieur-technischer Probleme. In strukturierten Kontexten läuft die
Bestellung der nächsten Stahllieferung samt Rechnungslegung und Fakturierung
sicher genauso ARIZ-artig ab wie eine ingenieur-technische Entscheidung. Es
gibt also wenig Grund, wie in \cite{Mann2019} Managemententscheidungen per se
der Kategorie \emph{kompliziert} oder gar \emph{komplex} zuzuordnen.

Mit dem Bezug auf eine „Theorie komplexer adaptiver Systeme“ (complex adaptive
systems -- CAS) wird der theoretische Bogen zu \cite{Foxon2009} geschlagen,
auch wenn die referenzierte theoretische Basis mit \cite{Snowden2007} dünn
ist. Der Titel jener Referenz fokussiert auf „leader's decision making“ und
nicht wie \cite{Foxon2009} auf partizipative Entscheidungsprozesse (AM) oder
Transitionsmanagement (TM).

Schauen wir uns die Argumente im Einzelnen an. Zunächst wird am Beispiel von
Spulenentwicklungen gezeigt, dass auch in der TRIZ-Welt Analogielösungen an
konkrete Parameterbereiche gebunden sind, deren Grenzen „disruptive“ Lösungen
einfordern, die nur durch Übergang zu anderen physikalisch-technischen
Prinzipien möglich sind. Wir finden also auch in diesem Bereich die r-, K-,
$\Omega$- und $\alpha$-Phasen aus \cite{Holling2000}, wobei TRIZ vor allem in
der Bewältigung von Übergängen seine analytischen Stärken ausspielt, in denen
wohlfeile Kontexte zu trans\-zendieren sind. TRIZ bietet hierfür ein größeres
Arsenal von abstrakten Trends, Mustern und Standards an, um Kontexte gezielt
zu vergrößern und in diesem größeren Kontext Transitionspfade zu
identifizieren.

Wie bereits in der Diskussion um \cite{Geels2007} und \cite{Foxon2009} steht
dabei die Frage, wie allgemeingültig derartige Trends, Muster und Standards
sind. TRIZ erhebt hier einen sehr universalistischen Anspruch, der seine
historischen Gründe haben mag (siehe dazu \cite{Gerovich1996}), aber praktisch
nicht zu rechtfertigen ist. Eine \emph{methodische Kontextualisierung} der
TRIZ-Methodik (wann greifen welche Methoden) ist also angezeigt, und in genau
diese Richtung argumentiert \cite{Mann2019}.  Das dort entwickelte Modell ist
sehr einfach und stellt „Komplexität“ von System und Umwelt auf einer
4-stufigen Skala ins Verhältnis, was hier sofort präzisierend als Verhältnis
von System und Obersystem gefasst werden soll. Mit der „Ashby line“ wird dabei
ein spezifisches Komplexitätskonzept aufgerufen, das wir bereits früher als
problematisch identifiziert hatten, da es auf reine Kanalkapazitäten setzt und
intelligente Kompressions- und Dekompressionstechniken nicht berücksichtigt.

Gleichwohl können die vier Stadien „einfach“, „kompliziert“, „komplex“ und
„chaotisch“ durchaus verwendet werden, um die Kopplung von
Strukturierungsprozessen in System und Obersystem zu besprechen. Der Hinweis
„natural forces act against resilience“ \cite[Fig. 3]{Mann2019} entspricht dem
Übergang von der r- in die K-Phase in \cite{Holling2000} und wird auch ähnlich
begründet: Eine \emph{junge} Technologie ist zunächst wenig verstanden und
deshalb „komplex“. Im Zuge der weiteren Entwicklung werden nicht nur die
Beschreibungsformen präziser, sondern auch die institutionalisierten
Verfahrensweisen. Damit werden typische Einsatzszenarien in typischen
Kontexten einfacher, der Gebrauch der Technologie ist nur noch „kompliziert“.
Mit der Weiterentwicklung zu einer \emph{reifen} Technologie differenziert
sich diese Gebrauchsfähigkeit weiter aus und (dies fällt bei D. Mann in Fig. 3
allerdings unter den Tisch) die eine komplizierte Technologie spaltet sich in
eine Vielzahl von verschiedenen einfacheren technologischen Lösungen für
verschiedene spezifischere Anwendungskontexte.

Als „Quer“-Tendenz (horizontal in Fig. 3) wird der „2. Hauptsatz der
Thermodynamik“ bemüht, um zu begründen, dass sich realweltliche
Kontextualisierungen ändern und damit früher passfähige Lösungen nicht mehr
passen.  Darauf ist angemessen durch Gegenstrategien \cite[Fig. 4]{Mann2019}
zu reagieren. Das „Chaos der Welt“, das hier über den 2. Hauptsatz in die
Betrachtungen eingeführt wurde, seine Quelle aber in der reduktiven Qualität
der Beschreibungsform hat, ist selbst strukturiert und rührt (u.a.) aus
Transitionsprozessen an anderen Stellen der „Welt der Systeme“ her mit
unterschiedlicher Anschlussfähigkeit an die im System selbst anstehenden
Transitionen, wie in der Typologie in \cite{Geels2007} genauer entfaltet.

Die „horizontalen Gegenstrategien“ aus Fig. 4 einer Kontextaufspaltung und die
„vertikalen Strategien“ aus Fig. 3 einer weiteren Vereinfachung und
Standardisierung stehen in engem Bezug zueinander und sind eigentlich nur in
ihrer Gemeinsamkeit aus Vereinfachung der Beschreibungsform (Fig. 3) und
Spezialisierung der Vollzugsform (Fig. 4) als sich gegenseitig bedingend
verständlich. Die „vertikalen Gegenstrategien“ aus Fig. 4 entsprechen dem
TRIZ-Trend 4 des „Übergangs zum Makrolevel“ (TESE) und damit der
Stabilisierung der Rahmenbedingungen der Vollzugsdimension. Beides
(Diversifizierung im System und Stabilisierung der Rahmenbedingungen) hatten
als wichtige Resilienz-Strategien auch bisher eine Rolle gespielt, um
Transitionen in der Welt der Systeme lokal einzugrenzen. Diversifizierung
bedeutet dabei, das System gegen Änderungen des Kontexts robuster zu machen
und damit Umbauprozesse im Obersystem besser auszuhalten. Stabilisierung der
Rahmenbedingungen bedeutet den Übergang auf die nächste Abstraktionsebene, auf
der die \emph{Beziehungen} zwischen System und Obersystem(en) zum Gegenstand
systemischer Gestaltung werden. Eine solche Perspektive bleibt komplett
außerhalb des Betrachtungshorizonts von \cite{Mann2019}. Allerdings wird
„Trend 4“ auch in (TESE) anders verstanden.

\section{Gesetze und Trends der Entwicklung technischer Systeme
  (Gräbe)} 

Literatur: \cite{Goldovsky1983}, \cite{Graebe2019}, \cite{Rubin2019};
Zusatzliteratur: \cite{TESE2018}, \cite{Ropohl2009}

Es wurde die Frage näher untersucht, wie der im Seminar entwickelte
Systembegriff, insbesondere der Begriff eines \emph{technischen Systems}, mit
Systembegriffen harmoniert, die im TRIZ-Umfeld verwendet werden. (TESE) ist
hierfür eine gute Referenz, da die zusammenfassende Darstellung der
„Entwicklungstrends von ingenieur-technischen Systemen“ den Status eines
„durch die MATRIZ autorisierten Lehrbuchs“ hat. Es wird explizit der Begriff
\emph{engineering system} gegenüber dem in der sonstigen TRIZ-Literatur,
besonders auch der russischsprachigen, üblichen Begriff des \emph{technischen
  Systems} verwendet.

Allerdings finden sich weder in (TESE) noch in den anderen Referenzen genauere
Begriffsdefinitionen, was unter einem \emph{technischen System} zu verstehen
sei. In allen Quellen wird auf die Anschauung verwiesen, wobei wir mit der
\emph{Facebookdiskussion} (siehe Abschnitt 5) gesehen hatten, dass diese
„Anschauung“ einen weiten Bereich möglicher Interpretation überdeckt.
Allerdings kommt selbst in jenen Betrachtungen die im Kommentar zu
\cite{Mann2019} aufgeworfene Frage nicht vor, ob \emph{Managementtechniken}
auch in Systemen technischer Art erfasst werden können oder hier mit anderen
Begriff\-lichkeiten zu operieren sei.  Der Rückzug auf „ingenieur-technische
Systeme“ wie in (TESE) verschiebt das Problem nur zur Frage, ob modernes
Management- und Verwaltungshandeln nicht auch eine Ingenieurstätigkeit sei.
Von den Anforderungen an spezifische Kenntnisse theoretischer Grundlagen,
institutionalisierter Verfahrensweisen und algorithmischer Vorgehensweisen
sind diese Tätigkeitsprofile jedenfalls von Ingenieurstätigkeiten in größeren
Unternehmen kaum zu unterscheiden.

Explizite systemtheoretische Ansätze im TRIZ-Umfeld verweisen auf komplexe
Wurzeln in Moskauer philosophischen Kreisen der 1960er bis 1980er Jahre, von
denen offensichtlich auch Altschuller beeinflusst war, als er 1984 die in
(TESE) referenzierte Liste von 8 Gesetzen der Entwicklung technischer Systeme

\begin{enumerate}[noitemsep]
\item Gesetz der Vollständigkeit der Teile eines Systems
\item Gesetz der „Energieleitfähigkeit“ eines Systems
\item Gesetz der Harmonisierung der Rhythmen der Systemteile
\item Gesetz der wachsenden Idealität
\item Gesetz der ungleichmäßigen Entwicklung der Systemteile
\item Gesetz des Übergangs zum Obersystem
\item Gesetz des Übergangs von der Makro- zur Mikroebene
\item Gesetz der wachsenden Stoff-Feld-Interaktionen
\end{enumerate}
formulierte. Bereits an dieser Stelle gehen die Darstellungen in (TESE) und
\cite{Rubin2019} auseinander. Rubin bezieht sich auf eine Liste von 9 Gesetzen,
die Altschuller 1977 in Baku veröffentlicht hat und das weitere
\begin{enumerate}[noitemsep]
\item[9.] Gesetz der Dynamisierung starrer technischer Systeme
\end{enumerate}
enthält, was auch im TRIZ-Prinzip 15 gelistet ist. Die Abgrenzung von Trends,
Standards und Prinzipien ist in der TRIZ aber generell problematisch.

\cite{Goldovsky1983} scheint eine wichtige Referenz zu sein, welche die
Verbindung zwischen den Ansätzen eines „Schöpfertums als exakter Wissenschaft“
(Altschuller) und philosophischen Überlegungen herstellt.  In jener Arbeit
wird der Gesetzesbegriff strapaziert, um systemische Entwicklungslinien auf
verschiedenen Abstraktionsebenen zu charakterisieren, und es wird der Begriff
„technisches System“ in den komplexeren Kontext der Entwicklung
\emph{allgemeiner Systeme} eingebettet. Auf die Frage, ob es sich um Gesetze
oder eher um Trends oder gar nur um Entwicklungsmuster handelt, soll hier
nicht eingegangen werden.

Sowohl \cite{Goldovsky1983} als auch \cite{Rubin2019} bleiben eine genauere
Fassung auch des allgemeinen Systembegriffs schuldig. Goldovsky thematisiert
eine Hierarchisierung der dort formulierten Gesetze in
\begin{enumerate}[noitemsep]
\item Grundlegende Entwicklungsmuster
\item Methodologische Muster der Entwicklung technischer Systeme
\item Gesetzmäßigkeiten der Herstellung arbeitsfähiger technischer Systeme
\item Gesetzmäßigkeiten funktioneller Transformationen technischer Systeme
\item Gesetzmäßigkeiten struktureller Transformationen technischer Systeme
\item Muster der Transformation der Systemzusammensetzung
\end{enumerate}
wobei die formulierten Punkte eher einen metaphysischen Charakter der
Kontextualisierung einer Betrachtungsperspektive haben, und somit doch zur
Schärfung der Begriff\-lichkeit eines „technischen Systems“ beitragen,
insbesondere durch die „methodologischen Muster“ 2.1-2.4.

Diese Hierarchisierung reflektiert in gewisser Weise die Komplexität von
Systemtransformationen und reicht von
\begin{enumerate}[noitemsep]
\item grundsätzlichen Epistemiken von Beschreibungsformen über
\item Anforderungen an Beschreibungsformen (an die Modellierung) technischer
  Systeme,
\item Anforderungen an die Verbindung von Beschreibungs- und Vollzugsformen
  technischer Systeme (Betriebsbedingungen in gegebenem Kontext),
\item Anforderungen an die Lösung von Widersprüchen durch funktionale
  Reorganisation (bei unveränderten Komponenten),
\item Anforderungen an die Lösung von Widersprüchen durch strukturelle
  Reorganisation (auch die Komponenten werden verändert) bis hin zu
\item Anforderungen an systemische Reorganisation.
\end{enumerate}

Sie deckt damit einen Teil der systemischen Reorganisationserfordernisse ab,
die in \cite{Geels2007} identifiziert werden. Es bleibt weiter auszuloten,
welche tieferliegenden Erkenntnisse aus diesen eher metaphysisch formulierten
Mustern zur Bewältigung \emph{realer} Transitionserfordernisse zu gewinnen
sind.

Altschuller selbst teilt seine Gesetze in statische (1-3), kinematische (4-6)
und dynamische (7-8) und postuliert die Gültigkeit der statischen und
kinematischen Gesetze für die Entwicklung auch allgemeiner Systeme, während er
die dynamischen Gesetze 7-8 als zeit- und domänenspezifisch ansieht. Diese
Überlegungen werden in \cite{Rubin2019} weiter detailliert.  Wie in
\cite{TESE2018} werden die Gesetze in eine baumartige Kausalstruktur gebracht
(präziser: in die Struktur eines gerichteten azylischen Graphen) und in einem
zweiten Schritt die Verbindung zu den TRIZ-Standards hergestellt, die als
operationale Ausprägung der jeweiligen Gesetze in der TRIZ-Methodik betrachtet
werden. Von dort wird der Bogen weiter zu ARIZ und der Algorithmisierung der
Methodik geschlagen.

Sowohl die Auswahl der Gesetze als auch die genaue Ausgestaltung der kausalen
Beziehungen unterscheiden sich zwischen der Darstellung von Lyubomirsky und
Litvin selbst in \cite[S. 6]{TESE2018}, Rubins Darstellung der Gesetze nach
Lyubomirsky und Litvin (Abb. 1) und der eigenen Darstellung (Abb. 2). Rubin
diskutiert weiter die Verbindung dieser Gesetze zu einer allgemeinen
Systemtheorie, für die er 12 Gesetze in 4 Blöcken formuliert, was weiter zu
analysieren bleibt.

\section{Wie entwickeln sich technische Systeme}

Literatur: \cite{Geels2007}, \cite{Kohlhase2009}, \cite{Graebe2019},
\cite{Rubin2019}, \cite{Ropohl2009}.

Bleibt die abschließende Frage: Wie weit trägt ein systemtheoretischer Ansatz
überhaupt? Wir hatten eingangs unseres Seminars festgestellt, dass es nicht
\emph{den} systemtheoretischen Ansatz gibt, sondern wir mit einem ganzen
Universum aufeinander bezogener Ansätze konfrontiert sind, was zum Begriff der
\emph{Systemwissenschaft} im Seminartitel Anlass gab. \cite{Ropohl2009} lotet
dieses Problem weiter aus und identifiziert drei wesentlich verschiedene
Ansätze
\begin{enumerate}[noitemsep]
\item das funktionale Konzept eines Systems als „Black Box“,
\item das strukturelle Konzept der Modellierung von Wechselwirkungen zwischen
  Komponenten und
\item das hierarchische Konzept einer System-Umwelt-Beziehung.
\end{enumerate}

Das im Seminar entwickelte Konzept geht mit der Betrachtung der Einheit von
Beschreibungs- und Vollzugsform einen deutlichen Schritt weiter. Die drei von
Ropohl unterschiedenen An\-sätze werden als drei Reduktionsdimensionen von
Beschreibungsformen identifiziert, die in unserem Systembegriff
\emph{gleichzeitig} wirken. Dabei bekommen insbesondere die unspezifischen
Begriffe „Umwelt“ und „Obersystem“ eine genauere Fassung: Umwelt kann in
diesem Beschreibungsansatz nur selbst wieder als System und damit nicht als
Totalität einfließen.  Allerdings kann ein System in einem solchen Verständnis
auf \emph{mehrere} Obersysteme bezogen sein, womit die
System-Obersystem-Beziehung ihren exklusiven Charakter unter den systemischen
Nachbarschaftsbeziehungen verliert. Auf der anderen Seite ist zwischen
Modellierung und Metamodellierung zu unterscheiden, wobei letztere regelmäßig
bedeutsam wird, wenn es um die systemische Fassung von Beschreibungsformen der
\emph{Beziehungen} zwischen Systemen geht.

Letzteres gibt Anlass zu einer Stratifizierung der Wirklichkeit längs der
Begriffsbildungsniveaus der Beschreibungsformen, die Kleemann als prägend für
hoch technisierte Gesellschaften bezeichnet hat. Diese
Beschreibungsstratifizierung als spezifische Form der Komplexitätsreduktion
(„Fiktion“ in meiner Vorlesung) findet ihre Entsprechung in technischen
Schichtenarchitekturen wie etwa im OSI-7-Schichten-Modell.

Systemische Betrachtungen identifizieren auf der Beschreibungsebene Einheit in
der Vielfalt, aus der in der Vollzugsform wieder Vielfalt zurückgewonnen
werden muss. Menschen sind hier zugleich Subjekt und Objekt von Handeln. Die
damit verbundenen Widersprüche sind im Prinzip bewusst gestaltbar, enthalten
aber einen weiteren Stolperstein - Selbstbezüglichkeit. Hier ist Systemtheorie
überfordert und muss in eine Gesellschaftstheorie eingebettet werden, so
Kleemann.  \cite{Foxon2009} hatte mit dem partizipativen Ansatz eines
\emph{adaptive management} in einem Multi-Stakeholder-Kontext eine wichtige
Form einer solchen Einbettung aufgezeigt. Systemtheorie bleibt ein wichtiges
\emph{Instrument des Handelns} in einem solchen Kontext, wenn sie auf vier
wesentliche Punkte ausgerichtet wird:
\begin{enumerate}[noitemsep]
\item Theoriegeladenheit
\item Bewältigung des Ebenenproblems von Beschreibungsformen und
  Begriffsbildungsprozessen
\item Bewältigung des Durchsatzproblems: Durchsatz bestimmt das
  Innenverständnis des Systems, das „kooperative Weltbild“, wie in der
  Vorlesung entwickelt.
\item Ausrichtung auf Transition und Transformation, Resilienz und
  Nachhaltigkeit, Dynamik aller Komponenten und Beziehungen.
\end{enumerate}

Praktische Konsequenzen eines solchen Übergangs zum „Primat des Politischen“:
\begin{enumerate}[noitemsep]
\item Wir brauchen eine Initialphase. Wie kommen wir dahin?
\item Die Initialphase ist mit Vorurteilen beladen. Diese können
  systemtheoretisch-analytisch aufgearbeitet werden.  TRIZ ist hierfür eine
  leistungsfähige Methodik.
\item Was leisten hierbei Frametheorien \emph{praktisch}?
\end{enumerate}

\begin{thebibliography}{xxx}
\bibitem{Anderies2004} John M. Anderies, Marco A. Janssen, Elinor Ostrom
  (2004).  Framework to Analyze the Robustness of Social-ecological Systems
  from an Institutional Perspective. In: Ecology and Society 9 (1),
  18.\\ \url{https://www.ecologyandsociety.org/vol9/iss1/art18/}
\bibitem{Ashby1958} William Ross Ashby (1958).  Requisite variety and its
  implications for the control of complex systems. In: Cybernetica 1:2,
  83--99.\\ \url{http://pcp.vub.ac.be/Books/AshbyReqVar.pdf}
\bibitem{Bertalanffy1950} Ludwig von Bertalanffy (1950). An outline of General
  System Theory, The British Journal for the Philosophy of Science, Volume I,
  Issue 2, 1 August 1950, 134–165.\\
  \url{https://doi.org/10.1093/bjps/I.2.134}
\bibitem{Binder2013} C.R. Binder, J. Hinkel, P.W. Bots, C. Pahl-Wostl (2013).
  Comparison of Frameworks for Analyzing Social-ecological Systems. Ecology
  and Society, 18 (4), 26.  \\
  \url{https://www.ecologyandsociety.org/vol18/iss4/art26/}
\bibitem{Boisot2011} Max Boisot, Bill McKelvey (2011). Complexity and
  Organization-Environment Relations: Revisiting Ashby’s Law of Requisite
  Variety. In: Allen, Peter, Steve Maguire and Bill McKelvey (eds.). The Sage
  Handbook of Complexity and Management, 279--298. 
\bibitem{Brand2007} Fridolin Simon Brand, Kurt Jax (2007).  Focusing the
  Meaning(s) of Resilience: Resilience as a Descriptive Concept and a Boundary
  Object. In: Ecology and Society 12 (1), 23.
  \url{https://www.ecologyandsociety.org/vol12/iss1/art23/}
\bibitem{Capurro1996} Rafael Capurro, Peter Fleissner, Wolfgang Hofkirchner
  (1996). Is a unified theory of information feasible?
  \url{http://www.capurro.de/trialog.htm}
\bibitem{Capurro1998} Rafael Capurro (1998). Das Capurrosche Trilemma.
  \url{http://www.capurro.de/janich.htm}.
\bibitem{Capurro2002} Rafael Capurro (2002). Menschengerechte Information oder
  informationsgerechter Mensch? \url{http://www.capurro.de/gotha.htm}.
\bibitem{Davis2008} Mike Davis (2008). Wer wird die Arche bauen?  Das Gebot
  zur Utopie im Zeitalter der Katastrophen.  Telepolis, 11.12.2008.
\bibitem{Dobusch2011} Leonhard Dobusch, Sigrid Quack (2011). Auf dem Weg zu
  einer Wissensallmende? Argumente Politik und Zeitgeschichte 28--30,
  S. 41--46.
\bibitem{Elsasser1981} W.M. Elsasser (1981). A form of logic suited for
  biology? In Robert Rosen (ed.). Progress in Theoretical Biology, Volume 6.
  Academic Press, p. 23--62.
\bibitem{Foxon2009} T.J. Foxon, M.S. Reed, L.C. Stringer (2009). Governing
  long‐term social–ecological change: what can the adaptive management and
  transition management approaches learn from each other? Environmental Policy
  and Governance, 19 (1), 3--20.\\ \url{https://doi.org/10.1002/eet.496}
\bibitem{KFK2002} Klaus Fuchs-Kittowski (2002). Wissens-Ko-Produktion.
  Verarbeitung, Verteilung und Entstehung von Informationen in
  kreativ-lernenden Organisationen.  Festschrift zum 65. Geburtstag von Klaus
  Fuchs-Kittowski.
\bibitem{Geels2007} Frank W. Geels, Johan Schot (2007). Typology of
  Sociotechnical Transition Pathways. In: Research Policy 36 (2007),
  399–417.\\ \url{https://doi.org/10.1016/j.respol.2007.01.003}
\bibitem{Gerovich1996} Slava Gerovitch (1996). Perestroika of the History of
  Technology and Science in the USSR: Changes in the Discourse. Technology and
  Culture, Vol. 37.1, S. 102--134.
\bibitem{Goldovsky1983} Boris I. Goldovsky (1983). System der
  Gesetzmäßigkeiten des Aufbaus und der Entwicklung technischer Systeme.
  \url{https://wumm-project.github.io/Texts.html}
\bibitem{Graebe2012} Hans-Gert Gräbe (2012). Wie geht Fortschritt? LIFIS
  Online, 12.11.2012.
\bibitem{Graebe2019} Hans-Gert Gräbe (2019). Zur Entwicklung Technischer
  Systeme.  Manuskript. \\ \url{https://wumm-project.github.io/Texts.html}
\bibitem{Helfrich2011} Silke Helfrich, Felix Stein. Was sind Gemeingüter?
  Argumente Politik und Zeitgeschichte 28--30, 9-15.
\bibitem{Holland2006} John H. Holland (2006). Studying complex adaptive
  systems. In: Journal of systems science and complexity, 19 (1),
  1–8.\\ \url{https://link.springer.com/article/10.1007/s11424-006-0001-z}
\bibitem{Holling2000} C.S. Holling (2000). Understanding the Complexity of
  Economic, Ecological, and Social Systems. In: Ecosystems (2001) 4, 390–405.
  \url{https://www.esf.edu/cue/documents/Holling_Complexity-EconEcol-SocialSys_2001.pdf}
\bibitem{Jacobasch2019} Gisela Jacobasch (2019). Bienensterben -- Ursachen und
  Folgen.  Leibniz Online 37 (2019).
  \url{https://leibnizsozietaet.de/bienensterben-ursachen-und-folgen/}
\bibitem{Janich2006} Peter Janich (2006). Was ist Information?
  Frankfurt/Main.
\bibitem{Jantsch1992} Erich Jantsch (1992). Die Selbstorganisation des
  Universums. Vom Urknall zum menschlichen Geist.  Hanser, München.
\bibitem{Jooss2017} Christian Jooß (2017). Selbstorganisation der Materie.
  Verlag Neuer Weg, Essen.
\bibitem{Klemm2003} Helmut Klemm (2003). Ein großes Elend. Informatik
  Spektrum, S. 267--273.
\bibitem{Klix1999} Friedhart Klix, Karl Lanius (1999). Wege und Irrwege der
  Menschenartigen.  Kohlhammer, Stuttgart.
\bibitem{Kohlhase2009} Andrea Kohlhase, Michael Kohlhase (2009). Spreadsheet
  Interaction with Frames: Exploring a Mathematical Practice. In: Carette J.,
  Dixon L., Coen C.S., Watt S.M. (eds). Intelligent Computer
  Mathematics. Proceedings of CICM 2009.  LNCS 5625. Springer, Berlin,
  Heidelberg.\\
  \url{https://kwarc.info/people/mkohlhase/papers/mkm09-framing.pdf}
\bibitem{Koltze2017} Karl Koltze, Valeri Souchkov (2017). Systematische
  Innovation.  2. Auf\-lage, Hanser, München.
\bibitem{Kozhemyako2019} Anton Kozhemyako (2019). Features of TRIZ
  applications for solving organizational and management problems:
  schematization of an inventive situation and working with models of
  contradictions. (In Russisch).\\ \url{https://matriz.org/kozhemyako/}
\bibitem{TESE2018} Alex Lyubomirskiy, Simon Litvin, Sergey Ikovenko et al.
  (2018).  Trends of Engineering System Evolution (TESE).
\bibitem{Mann2019} Darrell Mann (2019).  Systematic innovation in complex
  environments. Proceedings of the TRIZ Summit 2019 Minsk.\\
  \url{https://triz-summit.ru/file.php/id/f304797-file-original.pdf} 
\bibitem{Mele2010} C. Mele, J. Pels, F. Polese (2010). A brief review of
  systems theories and their managerial applications. Service Science,
  2(1--2), 126--135.\\ \url{https://doi.org/10.1287/serv.2.1_2.126} 
\bibitem{Mingers1989} John Mingers (1989). An Introduction to Autopoiesis --
  Implications and Applications. In: Systems Practice, Vol. 2, No. 2,
  1989.\\ \url{https://link.springer.com/article/10.1007/BF01059497}
\bibitem{Ostrom1990} Elinor Ostrom (1990). Governing the commons. The
  evolution of institutions for collective action. Cambridge University Press,
  New York.  ISBN 9780511807763.
\bibitem{Ostrom2007} Elinor Ostrom (2007). A diagnostic approach for going
  beyond panaceas.  Proceedings of the national Academy of sciences, 104(39),
  15181--15187.\\ \url{https://doi.org/10.1073/pnas.0702288104}
\bibitem{Prigogine1993} Ilya Prigogine, Isabelle Stengers (1993). Das Pardox
  der Zeit. Piper, München, Kap. 3--5.
\bibitem{Ropohl2009} Günter Ropohl (2009). Allgemeine Technologie: eine
  Systemtheorie der Technik.  KIT Scientific Publishing.
  \url{https://books.openedition.org/ksp/3007}
\bibitem{Rubin2019} Michail Rubin (2019).  Zum Zusammenhang der
  Entwicklungsgesetze allgemeiner Systeme und der Entwicklungsgesetze
  technischer Systeme. \\ \url{https://wumm-project.github.io/Texts.html}
\bibitem{Snowden2007} David J. Snowden, Mary E. Boone (2007).  A Leader’s
  Framework for Decision Making.  Harvard Business Review, November 2007.
\bibitem{Souchkov2014} Valeri Souchkov (2014). Breakthrough Thinking with TRIZ
  for Business and Management: An Overview.
  \url{https://www.semanticscholar.org}
\bibitem{Steinbuch1969} Karl Steinbuch (1969). Die informierte Gesellschaft.
  Stuttgart, 2. Auflage.
\bibitem{Stollorz2011} Volker Stollorz (2011). Elinor Ostrom und die
  Wiederentdeckung der Allmende. Argumente Politik und Zeitgeschichte 28--30,
  S. 3--8.
\bibitem{Szyperski2002} Clemens Szyperski (2002). Component Software.
  2. Auf\-lage.  Pearson Education.
\bibitem{Ulanowicz2009} Robert E. Ulanowicz (2009). The dual nature of
  ecosystem dynamics.  In: Ecological Modelling 220 (2009),
  1886–1892.\\ \url{https://people.clas.ufl.edu/ulan/files/Dual.pdf}
\bibitem{VDITechnik} VDI-Norm 3780 (2000). Technikbewertung -- Begriffe und
  Grundlagen.  
\bibitem{VDW2005} VDW -- Verein Deutscher Wissenschaftler (2005). „We have to
  learn to think in a new way“. Potsdamer Denkschrift.
\bibitem{Vernadsky1938} V.I. Vernadsky(1997, Original 1936--38). Scientific
  Thought as a Planetary Phenomenon.
  \url{https://wumm-project.github.io/Texts.html}
\bibitem{Walker2004} Brian Walker, C.S. Holling, Stephen R. Carpenter, Ann
  Kinzig (2004).  Resilience, Adaptability and Transformability in
  Social-ecological Systems.  In: Ecology and Society 9 (2).
  \url{https://www.ecologyandsociety.org/vol9/iss2/art5/}
\end{thebibliography}
\end{document}
