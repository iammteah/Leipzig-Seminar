% third part of TRIZ-Modelling after OTSM-TRIZ Slides
\documentclass{beamer}
\usepackage{lsfolien}
\usepackage[english]{babel}
\usepackage[utf8]{inputenc}

\myfootline{System Modelling and Semantic Web -- Spring 2021}{Hans-Gert Gräbe}

\title{Modelling Sustainable Systems\\ and Semantic Web\\[6pt]
  About the Notion of a Technical System
  \vskip1em}

\subtitle{Lecture in the Module 10-202-2309\\ for Master Computer Science}

\author{Prof. Dr. Hans-Gert Gräbe\\
\url{http://www.informatik.uni-leipzig.de/~graebe}}

\date{April 2021}
\begin{document}

\section{Basics}
\begin{frame}{TRIZ and ARIZ}

TRIZ is not just a theory, but proposes a precise algorithmic procedure as a
methodology to be applied.

There exist several variants of this algorithm ARIZ (Algorithm for the
Solution of Inventive Problems), the "official" one is ARIZ-85C, which is
based on a version published by Altschuller in 1985. Others (D. Zobel) see
little progress compared to ARIZ-77 (a version published by Altschuller in
1977) and recommend this somewhat simpler approach.

We use AIPS-2015 (Algorithm for the Correction of Problematic Situations), a
version in the tradition of OTSM-TRIZ, which is also used in the Minsk
TRIZ-Trainer.
\end{frame}

\begin{frame}{TRIZ-Trainer -- the First Stage of the Solution Process}
  
The first stage of the solution process provides an accurate model of the
"system as it is" that needs to be transformed to solve the problem. This
phase consists of three steps
\begin{itemize}
\item [(A)] Contextualise the problem. The system as a black box.
\item [(B)] Analyse and model the structural and procedural organisation of
  the "system as it is" -- the "machine" in the terminology of the
  TRIZ-Trainer.
\item [(C)] Identify and localise the central contradiction, determine the
  operational zone and operational time, i.e. where and when the contradiction
  occurs, and establish possible hypotheses about the causes of the conflict.
\end{itemize}
From these hypotheses a task is formulated, which in the second stage is
analysed in more detail.
\end{frame}

\begin{frame}{TRIZ-Trainer -- the First Stage of the Solution Process}

First section "Clarification of the circumstances":
\begin{itemize}
\item [1.] Identify the system to be examined as a black box and give it a
  "speaking name", from which the semantics of the system can already be
  roughly understood -- what is the "useful product"?
\item [2.] Identify the \emph{main useful function} (MUF) of the system.

  Investigate, if necessary, what \emph{purpose} the system serves in the
  supersystem and, if applicable, determine the throughput required to operate
  the system (input required from the upper system for the functioning of the
  system).
\item [3.] Formulate the existing problem, which prevents the specification
  compliant behaviour of the system in the supersystem -- the "undesired
  effect". 
\end{itemize}
\end{frame}

\begin{frame}{TRIZ-Trainer -- the First Stage of the Solution Process}
  
Second section "Conflict in the system": 
\begin{itemize}
\item[4.] Determine the components of the machine (the structural organisation
  of the system) as well as its mode of operation (the procedural organisation
  of the system).  Often it is sufficient to focus on one of the two
  questions.

  Follow the general structure pattern "energy source, engine, transmission,
  tool, action, object being processed, useful product plus control".
  
  It is important to describe the main useful function (MUF) of the system,
  even if the problem is located in one of its components, because the
  resources used in the system are grouped around the MUF.
\end{itemize}
\end{frame}

\begin{frame}{TRIZ-Trainer -- the First Stage of the Solution Process}
  
\begin{itemize}
\item [5.] This MUF is in some relation to the "effect that cannot be
  completed without problems".

  This effect, as well as its relationship to the MUF, is now to be determined
  more precisely as the core of the conflict to be resolved.

  In this analysis, in particular the place and time of the conflict must be
  determined more precisely in order to prepare for possible later separation
  by time or place as one of the basic methods of resolution.
\end{itemize}
\end{frame}

\begin{frame}{TRIZ-Trainer -- the First Stage of the Solution Proces}
  
Third section "Formulation of a hypothesis": 
\begin{itemize}
\item Through a more detailed analysis of the "causes of conflict", one or
  several hypotheses of general nature are formulated, what measures in the
  sense of the \emph{Ideal Final Result} would solve the problem.
  
  One of these approaches is formulated in more detail as a "task" for the
  second stage of the solution process in order to work on it with suitable
  TRIZ tools.
\end{itemize}
\end{frame}

\begin{frame}{The Ideal Final Result}
  
The \textbf{ideal final result} (IFR) describes the "system as required" as
\emph{target} of the transformation, without initially caring whether the
formulated result can be realised in practice. In the further solution
process, the obstacles to be overcome on the way to the IFR are identified
and, based on the TRIZ methodology, strategies are developed how to overcome
these obstacles in practice.

The IFR is one of the basic concepts of TRIZ. The IFR is an orientation in the
sense of a "concrete utopia", which essentially determines the target corridor
on which the further solution process concentrates in its second stage.
\end{frame}

\end{document}
