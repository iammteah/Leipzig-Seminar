\documentclass{beamer}
\usepackage{lsfolien,enumitem}
\usepackage[english]{babel}
\setlist[itemize]{noitemsep,label={\color{blue}$\rhd$}}
           
\myfootline{Complex Systems and Co-Operative Action -- Spring 2021}{Hans-Gert
  Gr\"abe}

\newcommand{\ueberschrift}[1]{\begin{center}\bf #1\end{center}}

\parskip1em

\title{System Management, Organizations, Systems. General Management Challenges}

\author{Prof. Dr. Hans-Gert Gräbe\\
\url{http://www.informatik.uni-leipzig.de/~graebe}}

\date{April 2021}
\begin{document}

{\setbeamertemplate{footline}{}
\begin{frame}
  \titlepage
\end{frame}}

\section{Systematic Management}
\begin{frame}{Systematic Management}

  (Rebekiah Hill, 2015)

  \emph{Systematic management} is an approach to management that focuses on the
management process rather than on the final outcome. The goals to this
approach to management were:
\begin{itemize}
\item To create specific processes and procedures to be used in job task
  completion.
\item To ensure that organizational operations were economical.
\item To ensure that staffing was adequate for the needs of the organization.
\item To maintain suitable inventory so that the demands of consumers could be
  met.
\item To establish organizational controls. 
\end{itemize}
\end{frame}
  
\begin{frame}{Systematic Management}
  \begin{itemize}
  \item Planning and experience
  \item Dialectical relation of tension between plan and reality
  \item Practical real-world development and \emph{practice of thinking} 
  \end{itemize}
  \vskip2em
  
  \begin{block}{World and Reality}
    \emph{World} is \emph{reality for us} and thus reality in the process of
    conceptual grasping.
  \end{block}
  
\end{frame}

\begin{frame}{Organization}

  \begin{block}{Formal Organization (Wikipedia)}
    
    An organization that is established as a \emph{means for achieving defined
      objectives} has been referred to as a formal organization.\medskip

    Its design specifies how \emph{goals are subdivided and reflected} in
    subdivisions of the organization. Divisions, departments, sections,
    positions, jobs, and tasks make up this work structure.\medskip

    Thus, the formal organization is expected to \emph{behave impersonally} in
    regard to relationships with clients or with its members. [...]\medskip

    A \emph{bureaucratic structure} forms the basis for the appointment of
    heads or chiefs of administrative subdivisions in the organization and
    endows them with the authority attached to their position.  (my emphasis)
  \end{block}
\end{frame}


\begin{frame}{Organization}
  \begin{block}{Informal Organization (Wikipedia)}
   
    [...] The informal organization expresses the personal objectives and
    goals of the individual membership.

    Their objectives and goals may or may not coincide with those of the
    formal organization. [...]
  \end{block}
\end{frame}

\begin{frame}{Organization}

  \ueberschrift{The ORG Ontology}

  \texttt{org:OrganizationalUnit}, \texttt{org:FormalOrganization} and
  \texttt{org:OrganizationalCollaboration} as subconcepts of the concept
  \texttt{org:Organization}.

  \begin{block}{Organization}
    ... represents a collection of people organized together into a community
    or other social, commercial or political structure.\medskip

    The group has some common purpose or reason for existence which goes
    beyond the set of people belonging to it and can act as an Agent.\medskip

    Organizations are often decomposable into hierarchical structures.
  \end{block}
\end{frame}

\begin{frame}{Organization}

  \texttt{org:Organization} is related to \texttt{foaf:Agent}

  ... the class of agents; things that do stuff. A well known sub-class is
  \texttt{foaf:Person}, representing people. Other kinds of agents include
  \texttt{foaf:Organization} and \texttt{foaf:Group}. 

  A \texttt{foaf:Group}

  ... represents a collection of individual agents (and may itself play the
  role of a Agent, i.e. something that can perform actions).

  This concept is intentionally quite broad, covering informal and ad-hoc
  groups, long-lived communities, organizational groups within a workplace,
  etc. ...
\end{frame}

\begin{frame}{Organization}

  While a Group has the characteristics of a Agent, it is also associated with
  a number of other Agents (typically people) who constitute the Group, its
  members. ...  The basic mechanism for saying that someone is to use the
  member property of the Group to indicate the agents that are members of the
  group.
  
  The terms Agent and Group thus introduce self-similar concepts of structures
  that are \emph{capable of action}. This corresponds to the legal
  construction of a \emph{juridical subject} in the sense of the Civil Code
  (BGB) if \emph{responsibility for the consequences of action} is added.

\end{frame}

\begin{frame}{Organization as Socio-Technical Systems}

This corresponds closely with the \emph{system concept in TRIZ}:

  A system (lat. greek "system", "composed", a whole consisting of parts;
  connection) is a set of elements that are interconnected and interact with
  each other, forming a unified whole that possesses properties that are not
  already contained in the constituent elements considered individually.
  (Petrov, 2020)
  
\end{frame}

\begin{frame}{Organization as Socio-Technical Systems}

  A \emph{system} is a set of elements that are in relationship and connection
  with each other and that constitute a well defined unity, an integrity.

  The necessity of the use of the term "system" occurs when it is required to
  emphasize that something is large, complex, immediately not wholly
  comprehensible, but at the same time a unified whole.

  Unlike the notions "set" or "aggregate", the concept of a system emphasizes
  the ordering, the integrity, the regularity of construction, functioning and
  development.  (TRIZ Ontology Project)
\end{frame}
\begin{frame}{Organization as Socio-Technical Systems}

  Ian Sommerville \emph{Software Engineering} also starts with the concept of a
  system and moves from there to the concept of \emph{organization}.\vskip2em
  
\begin{block}{System}
  A system is a meaningful set of interconnected components that work together
  to achieve a specific goal.  
\end{block}
\end{frame}
\begin{frame}{Organization as Socio-Technical Systems}
  \begin{block} {Technical computer-based systems}
    ... are systems that contain hardware and software components, but not
    procedures and processes. ... Individuals and organizations use technical
    systems for specific purposes, but knowledge of that purpose is not part
    of the system.\medskip

    For example, the word processor I use does not know that I am using it to
    write a book.
  \end{block}

\end{frame}

\begin{frame}{Organization as Socio-Technical Systems}

  \begin{block}{Socio-technical systems}
    ... contain one or more technical systems, but beyond that -- and this is
    crucial -- the knowledge of how the system should be used to achieve a
    broader purpose.\medskip

    This means that these systems have \emph{defined work processes},
    \emph{human operators} as integral part of the system, are \emph{governed
      by organizational policies} and are \emph{affected by external
      constraints} such as national laws and regulations.
  \end{block}

\end{frame}

\begin{frame}{Essential Characteristics of Socio-Technical Systems}
\begin{itemize}
\item They have special properties that affect the system as a whole, and are
  not related to individual parts of the system. These special properties
  depend on the system components and the relationships between them. Because
  of this complexity, the system-specific properties can only be evaluated
  when the system is composed.\medskip
\item They are often not deterministic. The behaviour of the system depends on
  the human operators and on other people who do not always react in the same
  way. Also, the operation of the system can change the system itself.
\end{itemize}
\end{frame}

\begin{frame}{Essential Characteristics of Socio-Technical Systems}
\begin{itemize}
\item The extent to which the system supports organizational goals depends not
  only on the system itself. It also depends on the \emph{stability of the
    goals}, the relationships and \emph{conflicts between organizational
    goals}, and how people in the organization \emph{interpret those goals}.
\end{itemize}

In this context, there is a clear shift on the scale of controllability to
movement according to intrinsic laws, which in \textbf{socio-economic systems}
with a large number of stakeholders or even \textbf{socio-ecological systems}
shifts further in the direction of movement according to intrinsic laws
("natural processes").
\end{frame}


\begin{frame}{Technical Systems}

The concept of a technical system in a planning and real-world context is four
times overloaded:\small
\begin{itemize}
\item[1.] as a real-world unique specimen (e.g. as a product, even if the
  unique specimen is a service),
\item[2.] as a description of this real-world unique specimen (e.g. in the
  form of a special product configuration)
\end{itemize}
and for components produced in larger quantities also
\begin{itemize}
\item[3.] as description of the design of the system template (product design)
  and
\item[4.] as description and operation of the delivery and operating
  structures of the real-world unique specimen systems produced from this
  template (as plans of production, quality assurance, delivery, operation and
  maintenance).
\end{itemize}

\end{frame}
  
\begin{frame}{Technical Systems}
The (more general) concept of a system in such a concept has the epistemic
function of (functional) «reduction to the essential». This reduction takes
place in the following three dimensions
\begin{itemize}
\item [(1)] External demarcation of the system against an environment,
  reduction of these relationships to input/output relations and guaranteed
  throughput.
\item [(2)] Internal demarcation of the system by combining subareas as
  components, whose functioning is reduced to a «behavioral control» via
  input/output relations.
\item [(3)] Reduction of the relations in the system itself to «causally
  significant» relationships.
\end{itemize}
\end{frame}

\begin{frame}{Technical Systems}
It is further stated there that such a reductive description service rests on
preexisting (explicit or implicit) description services in three dimensions:
\begin{itemize}
\item [(1)] An at least vague idea about the (working) input/output services of
  the environment.
\item [(2)] A clear idea of the inner function of the components (beyond the
  pure specification).
\item [(3)] An at least vague idea about causalities in the system itself,
  i.e. one that precedes the detailed modeling, an already existing idea of
  causality in the given context.
\end{itemize}
\end{frame}
\begin{frame}{Systems and Resources}\small
One final thought, not yet elaborated: the lofty approach at the beginning of
this presentation, that it is more about „the management processes rather than
the final outcome", is of course only half the truth (a well-known sentence of
a former Chancellor).  When it comes to \emph{reliability} in collaboration,
the specification-compliant outcome of a system (as a black box) is in the
foreground, and the way how this was achieved is minor important.

In a network of systems where one relies on the other, this form of
reliability plays a major role, since a prerequisite for a system to function
in accordance with its specification is not only its internal organisation,
but also that the system's operating conditions are met, which manifests
itself as structured access to the resources required for the work in the form
of a specific throughput of material, energy and information.
  
  
\end{frame}
\end{document}
