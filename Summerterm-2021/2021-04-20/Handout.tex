\documentclass[11pt,a4paper]{article}
\usepackage{ls}
\usepackage[english]{babel}

\title{System Management, Organizations, Systems. General Management
  Challenges}  

\newcommand{\Merksatz}[1]{
  \begin{center}
    \fbox{\fboxsep12pt \fbox{\parbox{.9\textwidth}{\textbf{Remember:} #1}}}
  \end{center}
}

\author{Hans-Gert Gr\"abe}

\date{19 April 2021}

\begin{document}
\maketitle

\section{Once more about the goal of the seminar}

Systematic innovation methodologies such as TRIZ are essentially based on a
better understanding of the development dynamics of corresponding (technical
and non-technical) \emph{systems}.  The results are rooted in engineering
experience from structured processes of planning, implementation and operation
of technical systems. Increasingly, cooperative interdisciplinary
collaboration matters rather than the one brilliant mind that commands
thousand hands. The \emph{socio-technical character} of contradictions is
thereby intensified and opens up new dimensions of contradiction management.

Today, managers face similar challenges when it comes to placing
decision-making processes on a systematic basis, aligning the processes under
control with long term goals, and also achieving the targeted goal corridors.
It turns out that many engineering experiences on structured procedures in
contradictory requirement situations can be transferred to this area, which
has been investigated within the topic "TRIZ and Business" for 20 years.

Nevertheless, experiences and approaches to theories of systematic management
are based more broadly and also have much longer historical traditions.
\emph{In the seminar}, we want to study this field more closely, with special
attention to cooperative approaches in interdisciplinary contexts.

\section{Systematic Management Basics}

"\emph{Systematic management} is an approach to management that focuses on the
management process rather than on the final outcome. The goals to this
approach to management were:
\begin{itemize}[noitemsep]
\item To create specific processes and procedures to be used in job task
  completion.
\item To ensure that organizational operations were economical.
\item To ensure that staffing was adequate for the needs of the organization.
\item To maintain suitable inventory so that the demands of consumers could be
  met.
\item To establish organizational controls." \cite{Hill2015}
\end{itemize}
These points require a \emph{planned} approach, based on a \emph{conceptual
  understanding} of the process landscape in an appropriate explicit form of
description and \emph{intelligible actions}.

The formulated intelligible actions -- the \emph{plan} -- is in
\emph{contradictory tension} with the processes actually taking place: On the
one hand, it has a controlling effect on these practices, on the other hand,
those practices partially resist this control.\enlargethispage{1em}

This difference must be fed back to the planning process as an
\emph{evaluation of experienced results} in order to keep also the divergence
between plan and reality under control.

Relating planning and experience dimension is only possible on a language
level and requires a \emph{system of notions} to accompany the practical
real-world development by a discursive process (as \emph{practice of
  thinking}).

This system of concepts is more stable than the real-world practices, but it
is not static -- it develops together with the practices.

\Merksatz{\emph{World} is \emph{reality for us} and thus reality in the
  process of conceptual grasping.}

These basic considerations are about \emph{processes} and \emph{procedures}
within an \emph{organization}.

\section{Organizations}

What is an organization? Wikipedia distinguises between formal aud informal
organizations. 

\paragraph{Formal organizations.}
"An organization that is established as a \emph{means for achieving defined
  objectives} has been referred to as a formal organization. Its design
specifies how \emph{goals are subdivided and reflected} in subdivisions of the
organization. Divisions, departments, sections, positions, jobs, and tasks
make up this work structure. Thus, the formal organization is expected to
\emph{behave impersonally} in regard to relationships with clients or with its
members. [...] A \emph{bureaucratic structure} forms the basis for the
appointment of heads or chiefs of administrative subdivisions in the
organization and endows them with the authority attached to their position."
(Wikipedia, my emphasis)

See about the "impersonality" also the "automaton" in the quote by Marx in my
first lecture.

\paragraph{Informal organizations.}
"[...] The informal organization expresses the personal objectives and goals
of the individual membership. Their objectives and goals may or may not
coincide with those of the formal organization. [...]" (Wikipedia)

The further explanations in Wikipedia remain weak and contradictory.
Structure-building processes and especially shared conceptual systems also
develop in informal organizations, with exciting new structuring processes
of co-operative action taking place that are of particular interest to us in
the seminar. Wikipedia is a reflection of the weakness of the conceptual basis
in this field.

Also ORG -- the \emph{organization ontology of the W3C} \cite{vocab-org} --
considers \texttt{org:OrganizationalUnit}, \texttt{org:FormalOrganization} and
\texttt{org:OrganizationalCollaboration} as subconcepts of the concept
\texttt{org:Organization} but does not mention informal organizations.  In
their definition an organization
\begin{quote}
  represents a collection of people organized together into a community or
  other social, commercial or political structure. The group has some common
  purpose or reason for existence which goes beyond the set of people
  belonging to it and can act as an Agent. Organizations are often
  decomposable into hierarchical structures.~\cite{vocab-org}
\end{quote}

\texttt{org:Organization} is related to \texttt{foaf:Agent}, 
\begin{quote}\raggedright
  ... the class of agents; things that do stuff. A well known sub-class is
  \texttt{foaf:Person}, representing people. Other kinds of agents include
  \texttt{foaf:Organization} and \texttt{foaf:Group}. \cite{foaf}
\end{quote}

A \texttt{foaf:Group}
\begin{quote}
  ... represents a collection of individual agents (and may itself play the
  role of a Agent, i.e. something that can perform actions).

  This concept is intentionally quite broad, covering informal and ad-hoc
  groups, long-lived communities, organizational groups within a workplace,
  etc. ...
  
  While a Group has the characteristics of a Agent, it is also associated with
  a number of other Agents (typically people) who constitute the Group, its
  members. ...  The basic mechanism for saying that someone is to use the
  member property of the Group to indicate the agents that are members of the
  group.
\end{quote}
The terms Agent and Group thus introduce self-similar concepts of structures
that are \emph{capable of action}. This corresponds to the legal construction
of a \emph{juridical subject} in the sense of the Civil Code (BGB) if
\emph{responsibility for the consequences of action} is added.  

\section{Organizations as Socio-Technical Systems}

While in the Wikipedia definition positions, jobs and tasks are mentioned, but
beyond bureaucracy no people, in this definition an organization is a
"community of people". However, it has a goal that does not result from the
set of goals of the people involved, but is an emergent function of the
organization -- the whole is more than the sum of its parts in the sense that
relational synergy effects are of special importance in such an organization.

This corresponds closely with the \emph{system concept in TRIZ}:
\begin{quote}
  A system (lat. greek "system", "composed", a whole consisting of parts;
  connection) is a set of elements that are interconnected and interact with
  each other, forming a unified whole that possesses properties that are not
  already contained in the constituent elements considered individually.
  \cite{Petrov2020}

  A \emph{system} is a set of elements that are in relationship and connection
  with each other and that constitute a well defined unity, an integrity. The
  necessity of the use of the term "system" occurs when it is required to
  emphasize that something is large, complex, immediately not wholly
  comprehensible, but at the same time a unified whole. Unlike the notions
  "set" or "aggregate", the concept of a system emphasizes the ordering, the
  integrity, the regularity of construction, functioning and development.
  \cite{TOP}
\end{quote}

Ian Sommerville \cite{Sommerville2015} also starts with the concept of a
system and moves from there to the concept of \emph{organization}.
\begin{quote}
  A system is a meaningful set of interconnected components that work together
  to achieve a specific goal.  \cite{Sommerville2015}
\end{quote}
Right after that comes a distinction between technical and socio-technical
systems:

\paragraph{Technical computer-based systems}
are systems that contain hardware and software components, but not procedures
and processes. ... Individuals and organizations use technical systems for
specific purposes, but knowledge of that purpose is not part of the system.
For example, the word processor I use does not know that I am using it to
write a book.

\paragraph{Socio-technical systems}
contain one or more technical systems, but beyond that -- and this is crucial
-- the knowledge of how the system should be used to achieve a broader
purpose.  This means that these systems have \emph{defined work processes},
\emph{human operators} as integral part of the system, are \emph{governed by
  organizational policies} and are \emph{affected by external constraints}
such as national laws and regulations.

Essential characteristics of socio-technical systems:
\begin{enumerate}
\item They have special properties that affect the system as a whole, and are
  not related to individual parts of the system. These special properties
  depend on the system components and the relationships between them. Because
  of this complexity, the system-specific properties can only be evaluated
  when the system is composed.
\item They are often not deterministic. The behaviour of the system depends on
  the human operators and on other people who do not always react in the same
  way. Also, the operation of the system can change the system itself.
\item The extent to which the system supports organizational goals depends not
  only on the system itself. It also depends on the \emph{stability of the
    goals}, the relationships and \emph{conflicts between organizational
    goals}, and how people in the organization \emph{interpret those goals}.
\end{enumerate}

In this context, there is a clear shift 
\begin{center}
  on the scale of controllability to movement according to intrinsic laws,
\end{center}
which in \textbf{socio-economic systems} with a large number of stakeholders
or even \textbf{socio-ecological systems} shifts further in the direction of
movement according to intrinsic laws ("natural processes").

Here, however, the TRIZ principle 25 \emph{Exploit Self-Service Processes}
becomes significant, which counts as the mastery of engineering.  It claims
that the best solution of a task is reached if the aspired goals are realised
"by themselves".

However, this means making the "natural" movement in systems according to
their own laws accessible to the unified expertise in terms of description.

\section{Systems and components}

From \cite{Graebe2020}

Operation and use of technical systems is a central element of today world
changing human practices. For this purpose planned and coordinated action
along a division of labour is necessary, because exploiting the benefit of a
system requires its operation. Conversely, it makes little sense to operate a
system that is not being used. Closely related to this distinction between
definition and call of a function, well known from computer science, is the
distinction between design time and runtime, that is even more important in
the real-world use of technical systems based on the division of labour –
during design time, the principal cooperative interaction is planned, during
the runtime the plan is executed. For technical systems one has to distinguish
the descriptive forms, interpersonally communicated as justified expectations,
and the enforcement forms, interpersonally communicated as experienced
results.

In addition to the description and enforcement dimension, for technical
systems the aspect of reuse also plays a major role. This applies, at least on
the artifact level, but not to larger technical systems – these are unique
specimen, even though assembled using standardized components. Also the
majority of computer scientists is concerned with the creation of such unique
specimens, because the IT systems that control such plants are also unique. In
this work we concentrate especially on such large technical systems and their
parallels to design issues of socio-ecological systems.

The special features of a technical system are therefore mainly in the area of
interplay of components, where one has also to distinguish between the
description form (modeling) and the enforcement form (operation in the context
of the various large-scale technical systems). While in the planning and
modeling phase there still remains open much freedom for changes, the
enforcement form is characterized by significantly higher
inflexibility. Although here too the world is more complicated than getting
caught up in a dichotomy like this – who dares to change a plan which has
already been approved by the high chiefs – we are working with such a concept
of „reduction“ in the following.

This brings together essential elements to serve as basis for a concept of a
technical system, which in a planning and real-world context is four times
overloaded:
\begin{itemize}
\item[1.] as a real-world unique specimen (e.g. as a product, even if the
  unique specimen is a service),
\item[2.] as a description of this real-world unique specimen (e.g. in the
  form of a special product configuration)
\end{itemize}
and for components produced in larger quantities also
\begin{itemize}
\item[3.] as description of the design of the system template (product design)
  and
\item[4.] as description and operation of the delivery and operating
  structures of the real-world unique specimen systems produced from this
  template (as plans of production, quality assurance, delivery, operation and
  maintenance).
\end{itemize}
Technical Systems in such a context are systems whose design and use are
influenced by cooperatively acting people on the basis of the division of
labour, whereby existing technical systems are normatively characterized at
description level by a specification of its interfaces and at enforcement
level by their guaranteed specification-compliant operation.

The same applies to the description form of «natural» systems, which are also
modeled in a structured way as systems of systems – as systems consisting of
components, which in turn are modeled as systems, whose functioning (both in a
functional and operational sense) are presupposed for the currently considered
system level.

The (more general) concept of a system in such a concept has the epistemic
function of (functional) «reduction to the essential». This reduction takes
place in the following three dimensions
\begin{itemize}
\item [(1)] External demarcation of the system against an environment,
  reduction of these relationships to input/output relations and guaranteed
  throughput.
\item [(2)] Internal demarcation of the system by combining subareas as
  components, whose functioning is reduced to a «behavioral control» via
  input/output relations.
\item [(3)] Reduction of the relations in the system itself to «causally
  significant» relationships.
\end{itemize}
It is further stated there that such a reductive description service rests on
preexisting (explicit or implicit) description services in three dimensions:
\begin{itemize}
\item [(1)] An at least vague idea about the (working) input/output services of
  the environment.
\item [(2)] A clear idea of the inner function of the components (beyond the
  pure specification).
\item [(3)] An at least vague idea about causalities in the system itself,
  i.e. one that precedes the detailed modeling, an already existing idea of
  causality in the given context.
\end{itemize}
(1) and (2) can in turn be developed in systems theory approaches to describe
the «environment» and the components (as subsystems), with which the
description of coevolutionary scenarios in turn becomes important for
deepening the understanding of (3).

\section{Systems and resources}

One final thought, not yet elaborated here: the lofty approach at the
beginning of these remarks, that it is more about „the management processes
rather than the final outcome", is of course only half the truth (a well-known
sentence of a former Chancellor).  When it comes to \emph{reliability} in
collaboration, the specification-compliant outcome of a system (as a black
box) is in the foreground, and the way how this was achieved is minor
important.

In a network of systems where one relies on the other, this form of
reliability plays a major role, since a prerequisite for a system to function
in accordance with its specification is not only its internal organisation,
but also that the system's operating conditions are met, which manifests
itself as structured access to the resources required for the work in the form
of a specific throughput of material, energy and information.

\begin{thebibliography}{xxx}

\bibitem{foaf} FOAF Vocabulary Specification (2014).\\ W3C Recommendation.
  \url{http://xmlns.com/foaf/spec/}.
\bibitem{Graebe2020} Hans-Gert Gräbe (2020). Men and their technical systems
  (in German).\\ LIFIS Online, 19 May 2020.
  \url{https://doi.org/10.14625/graebe_20200519}.

  A shorter English version is available at\\
  \url{https://hg-graebe.de/EigeneTexte/sys-20-en.pdf}
\bibitem{Hill2015} Rebekiah Hill (2015).  Systematic Management: Theory,
  Overview.\\{\small
  \url{https://study.com/academy/lesson/systematic-management-theory-lesson-quiz.html}}
\bibitem{vocab-org} The Organization Ontology (2014).\\  W3C Recommendation.
  \url{https://www.w3.org/TR/vocab-org}.
\bibitem{Petrov2020} Vladimir Petrov (2020). Laws and patterns of systems
  development (in Russian). ISBN 978-5-0051-5728-7.
\bibitem{Sommerville2015} Ian Sommerville (2015). Software Engineering.
  Chapter 19 „Systems Engineering“.

  Slide stack available at Sildeshare\\{\small
  \url{https://www.slideshare.net/software-engineering-book/ch19-systems-engineering}   }
\bibitem{TOP} The TRIZ Ontology Project.
  \url{https://wumm-project.github.io/Ontology.html}.
\end{thebibliography}

\end{document}

