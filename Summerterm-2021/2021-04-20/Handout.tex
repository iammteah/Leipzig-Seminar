\documentclass[11pt,a4paper]{article}
\usepackage{ls}
\usepackage[english]{babel}

\title{ISO 9000, Business Modelling and Software Management}  

\author{Hans-Gert Gr\"abe}

\date{18 April 2021}

\begin{document}
\maketitle

\section{Once more about the goal of the seminar}

Systematic innovation methodologies such as TRIZ are essentially based on a
better understanding of the development dynamics of corresponding (technical
and non-technical) systems.  The results are rooted in engineering experience
from structured processes of planning, implementation and operation of
technical systems. Increasingly, cooperative interdisciplinary collaboration
matters rather than the one brilliant mind that commands thousand hands. The
socio-technical character of contradictions is thereby intensified and opens
up new dimensions of contradiction management.

Today, managers face similar challenges when it comes to placing
decision-making processes on a systematic basis, aligning the processes under
control with long term goals, and also achieving the targeted goal corridors.
It turns out that many engineering experiences on structured procedures in
contradictory requirement situations can be transferred to this area, which
has been investigated within the topic "TRIZ and Business" for 20 years.

Nevertheless, experiences and approaches to theories of systematic management
are based more broadly and also have much longer historical traditions.
\emph{In the seminar}, we want to study this field more closely, with special
attention to cooperative approaches in interdisciplinary contexts.

\section{Systematic Management Basics}

"\emph{Systematic management} is an approach to management that focuses on the
management process rather than on the final outcome. The goals to this
approach to management were:
\begin{itemize}[noitemsep]
\item To create specific processes and procedures to be used in job task
  completion.
\item To ensure that organizational operations were economical.
\item To ensure that staffing was adequate for the needs of the organization.
\item To maintain suitable inventory so that the demands of consumers could be
  met.
\item To establish organizational controls." \cite{Hill2015}
\end{itemize}
These points require a \emph{planned} approach, which requires a
\emph{linguistically formulated picture} of the process landscape as a
suitably explicit form of description.

This linguistically formulated image -- the \emph{plan} -- is in
\emph{contradictory tension} with the processes actually taking place: On the
one hand, it has a controlling effect on these practices, on the other hand,
those practices partially resist this control.\enlargethispage{1em}

This difference must be fed back to the planning process as an
\emph{evaluation of experienced results} in order to limit the divergence
between plan and reality.

Relating planning and experience dimension is only possible on a linguistic
level and requires a \emph{system of notions} to accompany the practical
real-world development by a discursive process (as \emph{practice of
  thinking}).

This system of concepts is more stable than the real-world practices, but it
is not static -- it develops together with the practices.

These basic considerations are about \emph{processes} and \emph{procedures}
within an \emph{organization}.

\section{Organizations}

What is an organization? Wikipedia distinguises between formal aud informal
organizations. 

\paragraph{Formal organizations.}
"An organization that is established as a \emph{means for achieving defined
  objectives} has been referred to as a formal organization. Its design
specifies how \emph{goals are subdivided and reflected} in subdivisions of the
organization. Divisions, departments, sections, positions, jobs, and tasks
make up this work structure. Thus, the formal organization is expected to
\emph{behave impersonally} in regard to relationships with clients or with its
members. [...] A \emph{bureaucratic structure} forms the basis for the
appointment of heads or chiefs of administrative subdivisions in the
organization and endows them with the authority attached to their position."
(Wikipedia, my emphasis)

\paragraph{Informal organizations.}
"[...] The informal organization expresses the personal objectives and goals
of the individual membership. Their objectives and goals may or may not
coincide with those of the formal organization. [...]" (Wikipedia)

The further explanations in Wikipedia remain weak and contradictory.
Structure-building processes and especially shared conceptual systems also
develop in informal organisations, with exciting new structuring processes
of co-operative action taking place that are of particular interest to us in
the seminar. Wikipedia is a reflection of the weakness of the conceptual basis
in this field.

Also ORG -- the \emph{organisation ontology of the W3C} \cite{vocab-org} --
considers \texttt{org:OrganizationalUnit}, \texttt{org:FormalOrganization} and
\texttt{org:OrganizationalCollaboration} as subconcepts of the concept
\texttt{org:Organization} but does not mention informal organisations.  In
their definition an organization
\begin{quote}
  represents a collection of people organized together into a community or
  other social, commercial or political structure. The group has some common
  purpose or reason for existence which goes beyond the set of people
  belonging to it and can act as an Agent. Organizations are often
  decomposable into hierarchical structures.~\cite{vocab-org}
\end{quote}

While in the Wikipedia definition positions, jobs and tasks are mentioned, but
beyond bureaucracy no people, in this definition an organisation is a
"community of people". However, it has a goal that does not result from the
set of goals of the people involved, but is an emergent function of the
organization -- the whole is more than the sum of its parts in the sense that
relational synergy effects are of special importance in such an organization.

This corresponds closely with the \emph{system concept in TRIZ}:
\begin{quote}
  A system (lat. greek "system", "composed", a whole consisting of parts;
  connection) is a set of elements that are interconnected and interact with
  each other, forming a unified whole that possesses properties that are not
  already contained in the constituent elements considered individually.
  \cite{Petrov2020}

  A \emph{system} is a set of elements that are in relationship and connection
  with each other and that constitute a well defined unity, an integrity. The
  necessity of the use of the term "system" occurs when it is required to
  emphasize that something is large, complex, immediately not wholly
  comprehensible, but at the same time a unified whole. Unlike the notions
  "set" or "aggregate", the concept of a system emphasizes the ordering, the
  integrity, the regularity of construction, functioning and development.
  \cite{TOP}
\end{quote}

\section{Organizations as Socio-Technical Systems}

Ian Sommerville \cite{Sommerville2015} also starts with the concept of a
system and moves from there to the concept of \emph{organisation}.
\begin{quote}
  A system is a meaningful set of interconnected components that work together
  to achieve a specific goal.  \cite{Sommerville2015}
\end{quote}
Right after that comes a distinction between technical and socio-technical
systems:

\paragraph{Technical computer-based systems}
are systems that contain hardware and software components, but not procedures
and processes. ... Individuals and organisations use technical systems for
specific purposes, but knowledge of that purpose is not part of the system.
For example, the word processor I use does not know that I am using it to
write a book.

\paragraph{Socio-technical systems}
contain one or more technical systems, but beyond that -- and this is crucial
-- the knowledge of how the system should be used to achieve a broader
purpose.  This means that these systems have \emph{defined work processes},
\emph{human operators} as integral part of the system, are \emph{governed by
  organisational policies} and are \emph{affected by external constraints}
such as national laws and regulations.

Essential characteristics of socio-technical systems:
\begin{enumerate}
\item They have special properties that affect the system as a whole, and are
  not related to individual parts of the system. These special properties
  depend on the system components and the relationships between them. Because
  of this complexity, the system-specific properties can only be evaluated
  when the system is composed.
\item They are often not deterministic. The behaviour of the system depends on
  the human operators and on other people who do not always react in the same
  way. Also, the operation of the system can change the system itself.
\item The extent to which the system supports organisational goals depends not
  only on the system itself. It also depends on the \emph{stability of the
    goals}, the relationships and \emph{conflicts between organisational
    goals}, and how people in the organisation \emph{interpret those goals}.
\end{enumerate}


\begin{thebibliography}{xxx}

\bibitem{Hill2015} Rebekiah Hill (2015).  Systematic Management: Theory,
  Overview.\\{\small
  \url{https://study.com/academy/lesson/systematic-management-theory-lesson-quiz.html}}
\bibitem{vocab-org} The Organization Ontology (2014).\\  W3C Recommendation.
  \url{https://www.w3.org/TR/vocab-org}.
\bibitem{Petrov2020} Vladimir Petrov (2020). Laws and patterns of systems
  development (in Russian). ISBN 978-5-0051-5728-7.
\bibitem{Sommerville2015} Ian Sommerville (2015). Software Engineering.
  Chapter 19 „Systems Engineering“.

  Slide stack available at Sildeshare\\{\small
  \url{https://www.slideshare.net/software-engineering-book/ch19-systems-engineering}   }
\bibitem{TOP} The TRIZ Ontology Project.
  \url{https://wumm-project.github.io/Ontology.html}.
\end{thebibliography}

\end{document}

