\documentclass{beamer}
\usepackage{lsfolien,enumitem}
\usepackage[english]{babel}

\myfootline{Complex Systems and Co-Operative Action -- Spring 2021}{Hans-Gert
  Gr\"abe}

\newcommand{\ueberschrift}[1]{\begin{center}\bf #1\end{center}}

\title{Complex Systems and Co-Operative Action \vskip1em}

\subtitle{Research Seminar in the Module 10-202-2309\\ for Master Computer
  Science}

\author{Prof. Dr. Hans-Gert Gräbe\\
\url{http://www.informatik.uni-leipzig.de/~graebe}}

\date{April 2021}
\begin{document}

{\setbeamertemplate{footline}{}
\begin{frame}
  \titlepage
\end{frame}}

\section{Background}
\begin{frame}{Seminar Orientation}

\ueberschrift{The Notion of a System}

The concept of a \emph{system} plays a prominent role in computer science when
it comes to database systems, software systems, hardware systems, accounting
systems, access systems, etc.

In general, computer science is regarded by a majority as the "science of the
\emph{systematic} representation, storage, processing and transmission of
information, especially their automatic processing using digital computers"
(German Wikipedia).  Also certain relevant professions such as the
\emph{system architect} are in high esteem by IT users.
\end{frame}

\begin{frame}{Seminar Orientation}
However, the significance of the concept of system extends far beyond the
field of computer science -- it is fundamental for all engineering sciences
and as \emph{Systems Engineering} with the ISO/IEC/IEEE-15288 standard
"Systems and Software Engineering", it is also the subject of international
standardisation processes.

Even more, the concept of systems also plays an important role in the
description of complex natural and cultural processes -- for instance in the
concept of an \emph{ecosystem}.
\end{frame}

\begin{frame}{Seminar Orientation}
While classical TRIZ focuses strongly on instrumentally feasible engineering
solutions, Systems Engineering\vskip1em 
\begin{quote}
  is an interdisciplinary field of engineering and engineering management that
  focuses on how to design, integrate, and manage complex systems over their
  life cycles. At its core, systems engineering utilizes systems thinking
  principles to organize this body of knowledge. The individual outcome of
  such efforts, an engineered system, can be defined as a combination of
  components that work in synergy to collectively perform a useful function.
  (English Wikipedia)
\end{quote}
\end{frame}

\begin{frame}{Seminar Orientation}

\ueberschrift{Management of Systems' Changes}

In the winter semester 2019/20, we had already studied more intensively
different system concepts and, in particular, examined their application in
complex socio-ecological, socio-economic and socio-technical contexts.

We observed that the central concepts of \emph{transition management} and
\emph{activity management} addressed two different perspectives on structural
change processes.

In the transition management approach, the structural-transitional challenges
are in the foreground, the activity management approach studies the
implementation of structural changes via the actions and co-actions of actors
and stakeholders.
\end{frame}

\begin{frame}{Seminar Orientation}

  \ueberschrift{Management Theories and Contradictions}
\small
In both approaches, however, the focus was on a holistic-structural and
analytical view of a \emph{decision preparation} rather than on practical
procedural management approaches of \emph{decision-making} and decision
implementation in complex and contradictory real-world situations.

The WUMM project (WUMM stands in German for "Widersprüche und
Managementmethoden" -- contradictions and management methods) aims at a better
understanding of such management processes.

Our starting point is TRIZ as a systematic innovation methodology derived from
engineering experience in contradictory requirement situations.

Today, similar demands for an experience-based \emph{systematic} approach are
also addressed in the field of management, which means that engineering
approaches and admissions are also there on the agenda.\vspace*{2em}
\end{frame}

\begin{frame}{Seminar Orientation}\small
With the field of "Business TRIZ", which has been unfolding for about 20
years, this transfer of experience is being actively promoted, embedded in
older management cultures and approaches.

In recent years, co-operative action by differently specialised experts has
become increasingly important.

In such interdisciplinary work contexts, the development of \emph{common
  conceptual systems} of sufficient performance proves to be a difficult
problem that can be supported by digital semantic technologies.

Parallel to these challenges \emph{agile approaches} play a major role in
recent years, not only in the field of management, but also increasingly in
the solution of socio-technical and engineering problems concerning ongoing
co-operative actions in multi-stakeholder contexts -- for example with the
concept of \emph{technical ecosystems}.
\end{frame}

\section{Seminar Themes}
\begin{frame}{Seminar Orientation}

\textbf{In the seminar}, we want to learn more about traditional appoaches to
management theories (F. Taylor, R. Ackoff, P. Drucker, H. Mintzberg) and
relate this to developments in the area of Business TRIZ.

We are particularly interested in the connection between the dialectical
resolution of contradictory requirement situations in the sense of TRIZ
methodology and the emergence of common conceptual and notational worlds as a
result of the application of suitable semantic web technologies.

A special emphasis will be put on the work of the \emph{Methodological School
  of Management} and the Moscow Methodological Circle around
G.P. Shchedrovitsky.
\end{frame}

\section{Seminar Organisation}
\begin{frame}{Seminar Organisation}\small
The seminar is a \textbf{research seminar} in which we jointly explore
different aspects of co-operative action in different management concepts.

With this seminar, we are approaching a topic that is new to us, which offers
the opportunity to participate in a joint academic explorative process on a
basis of equals.

This bears opportunities, but also challenges.  The students are expected to
actively participate in the seminar through seminar discussions, presentations
and last but not least by reading the relevant materials.

For the successful completion of the seminar, a topic has to be presented in
the seminar as discussion leader and a handout of 2--3 pages on the topic has
to be submitted in advance.
\end{frame}

\begin{frame}{Seminar Organisation}\small

The seminar will be held weekly on Tuesdays 9-11 a.m. (Leipzig time)
synchronously online.

Prior to each appointment participants have to study the assigned reading to
be in a position to discuss the problems in the seminar.

The seminar is moderated by a \emph{discussion leader}, who prepares a short
handout of 2--3 pages and makes it available to the participants in advance
\emph{before the seminar} (by Sunday evening).

Students of Leipzig University find more about the seminar in the Saxonian
e-learning platform OPAL -- Course S21.BIS.SIM.  The platform will be used for
organisational purposes only.

The \textbf{primary source for the seminar plan} is the (actual version of
the) file \texttt{Seminarplan.md} in the \texttt{Summerterm-2021} folder of
the github repository \emph{Leipzig-Seminar}.
\end{frame}

\section{Course Structure}
\begin{frame}{Course Structure}
The course includes

\begin{itemize}
\item[$\bullet$] A lecture "Modelling Sustainable Systems and Semantic Web"
\item[$\bullet$] A seminar "Complex Systems and Co-Operative Action"
\item[$\bullet$] A TRIZ practical course.
\end{itemize}
\textbf{Note that the access to the e-learning system used in the TRIZ
  practical course is subject to a fee}. Details can be found in the forum of
the OPAL course.
\end{frame}

\section{The Lecture}
\begin{frame}{The Lecture}
In the \textbf{lecture} \emph{Modelling Sustainable Systems and Semantic Web}
(Thursdays 11-13 a.m.)  important concepts of our previous interdisciplinary
course programme such as
\begin{itemize}
\item[$\bullet$] technology as a unity of socially available procedural
  knowledge, institutionalised procedures and private procedural skills,
\item[$\bullet$] sustainability requirements in systemic concepts,
\item[$\bullet$] digital changes and concepts of semantic web technologies,
\item[$\bullet$] concept and knowledge formation processes,
\item[$\bullet$] cooperative action, network economies and open culture
\end{itemize}
will be developed in more detail.

The lecture and the seminar are not directly related to each other, but
conceptual frameworks developed in the lecture will be heavily present in the
seminar.\vskip3em
\end{frame}

\begin{frame}{Organisational Matters}
These course parts can be taken for credit in various combinations

\begin{itemize}
\item[1)] All three parts as In-depth Module 10-202-2309 (10 CP) "Modelling
  sustainable systems and semantic web".
  \begin{itemize}[noitemsep]
  \item[$\bullet$] \textbf{Prerequisites for examination:} successfully
    completed seminar and practical course.
  \item[$\bullet$] \textbf{Examination:} oral examination (30 min)
  \end{itemize}
\item[2)] Lecture and seminar as Seminar Module 10-202-2312(5 CP) "Applied
  Computer Science".
  \begin{itemize}[noitemsep]
  \item[$\bullet$] \textbf{Prerequisite for examination:} successfully
    completed seminar.
  \item[$\bullet$] \textbf{Examination:} RDF project and home work paper.
  \end{itemize}
\end{itemize}
\end{frame}

\begin{frame}{Organisational Matters}
\begin{itemize}
\item[3)] The practical course alone as Module 10-202-2012 (5 CP) "Current
  Trends in Computer Science".
  \begin{itemize}[noitemsep]
  \item[$\bullet$] \textbf{Prerequisite for examination:} successfully
    completed practical course.
  \item[$\bullet$] \textbf{Examination:} oral examination (30 min)
  \end{itemize}
\end{itemize}
More about this in OPAL \url{https://bildungsportal.sachsen.de/opal} in the
course S21.BIS.SIM.  There, please enrol first in the course and then in the
corresponding group.

You can access OPAL with the data of your studserv account.

\end{frame}

\begin{frame}{Organisational Matters}
You will find a more detailed lecture concept in the github repo
\url{https://github.com/wumm-project/Leipzig-Seminar} in the folder
\texttt{Summerterm-2021}.

\ueberschrift{Data protection}

We follow an Open Culture approach not only theoretically but also practically
and make course materials publicly available. This also applies to the course
materials you have to produce (presentations, seminar papers) as well as to
(annotated) chat sessions of the seminar discussions, in which your names are
also mentioned. \textbf{We assume your consent to this procedure if you do not
  explicitly object}. The discussions themselves are not recorded.

\end{frame}

\begin{frame}{Organisational Matters}

\begin{itemize}
\item[$\bullet$] Lecture: Thursdays 11:15-12:45, synchronous digital
\item[$\bullet$] The Flipped Classroom Concept
\item[$\bullet$] Continuously updated lecture plan and list of references in
  the \texttt{Lecture/README.md} file in the github Repo.  See also the
\item[$\bullet$] Further (mainly organisational) information also in the forum
  of the OPAL course.
\item[$\bullet$] Seminar: Tuesdays 9:15-10:45, synchronous digital
\item[$\bullet$] All events online in the BBB room BIS.SIM,
  \url{https://meet.uni-leipzig.de/b/gra-w2c-fhz-qnp}
\end{itemize}
\begin{center}\LARGE\bf
  Questions ?
\end{center}

See also \texttt{2021-04-13/README.md} for additional information about the
goal of the course. 

\end{frame}

\end{document}
